\chapter{Vowel sounds}
\label{chap:vow}

Un son de voyelle est fait en façonnant l'air comme il quitte la
bouche. Nous utilisons les articulateurs pour façonner l'air -- les
lèvres, la mâchoire, la langue. L'\exEN{anglais} britannique utilise 12 positions de la bouche.

\section{Front Vowels (langue vers l'avant)}
\label{sec:frontvow}

\speechvow{4}


\subsection{\son{iː}}\label{sec:ilong}

Ce son a pour nom technique :

\begin{itemize}
\item \exEN{Close Front Unrounded Vowel.}
\item \exFR{Voyelle frontale non arrondie.}
\end{itemize}

\indicsound

\properukus{https://youtu.be/EuZa9-QbhG8}{https://youtu.be/PIu5WDIco0I}

\begin{enumerate}
\item \exEN{\href{http://www.wordreference.com/enfr/need}{need}} qui s'écrit en phonétique \href{https://en.oxforddictionaries.com/definition/need}{\exPH{niːd}}

  \begin{itemize}
  \item\exEN{I \href{https://youtu.be/p0quLJutRC8}{need} to work every day if I want to improve my level.}
  \item\exFR{Je dois travailler tous les jours si je veux
      améliorer mon niveau.}
  \end{itemize}

  \youglish{need}


\item \exEN{\href{http://www.wordreference.com/enfr/tea}{tea}} qui s'écrit en phonétique \href{https://en.oxforddictionaries.com/definition/tea}{\exPH{tiː}}

  \begin{itemize}
  \item\exEN{Every morning we use to drink \href{https://youtu.be/Euh8dY4EU9o}{tea}.}
  \item\exFR{Tous les matins on a l'habitude de boire du thé.}
  \end{itemize}

  \youglish{tea}

\item \exEN{\href{http://www.wordreference.com/enfr/believe}{believe}}
  qui s'écrit phonétiquement
  \href{https://en.oxforddictionaries.com/definition/believe}{\exPH{bɪˈliːv}}
  
  \begin{itemize}
  \item\exEN{\href{https://youtu.be/GIQn8pab8Vc}{I believe I can fly.}}
  \item\exFR{Je crois que je peux voler.}
  \end{itemize}

  \youglish{believe}
  
\item \exEN{\href{http://www.wordreference.com/enfr/see}{see}} qui s'écrit
  phonétiquement
  \href{https://en.oxforddictionaries.com/definition/see}{\exPH{siː}}
  
  \begin{itemize}
  \item\exEN{What You \href{https://youtu.be/Dpf2yHjBVYM}{See} Is What You Get (\href{https://fr.wikipedia.org/wiki/What\_you\_see\_is\_what\_you\_get}{WYSIWYG})}
  \item\exFR{Ce que vous voyez est ce que vous obtenez.}
  \end{itemize}

  \youglish{see}
  
\end{enumerate}

\subsection{\son{ɪ}}\label{sec:soni}

Ce son a pour nom technique :

\begin{itemize}
\item \exEN{Near-Close Near-Front Unrounded Vowel.}
\item \exFR{à traduire.}
\end{itemize}

\indicsound

\properukus{https://youtu.be/7PpuPMrISVc}{https://youtu.be/Ok_HG-0lNCA}

\begin{enumerate}
\item \exEN{\href{http://www.wordreference.com/enfr/england}{England}} qui s'écrit en phonétique \href{https://en.oxforddictionaries.com/definition/england}{\exPH{ˈɪŋɡlənd}}

  \begin{itemize}
  \item\exEN{Last summer I went to \href{https://youtu.be/QUPBesOdax8}{England}.}
  \item\exFR{L'été dernier je suis allé en Angleterre.}
  \end{itemize}

  \youglish{england}
  
\item \exEN{\href{http://www.wordreference.com/enfr/thin}{thin}} qui s'écrit en phonétique \href{https://en.oxforddictionaries.com/definition/thin}{\exPH{θɪn}}

  \begin{itemize}
  \item\exEN{Usually female top model are \href{https://youtu.be/LekA62H17bo}{thin}.}
  \item\exFR{Habituellement les mannequins féminins sont minces.}
  \end{itemize}

  \youglish{thin}
  
\item \exEN{\href{http://www.wordreference.com/enfr/big}{big}} qui s'écrit phonétiquement \href{https://en.oxforddictionaries.com/definition/big}{\exPH{bɪɡ}}

  \begin{itemize}
  \item\exEN{New York has got a nickname: the \href{https://youtu.be/Jha4OkG-ixw}{big} apple.}
  \item\exFR{New York a un surnom : la grosse pomme.}
  \end{itemize}

  \youglish{big}
  
\item \exEN{\href{http://www.wordreference.com/enfr/which}{which}} qui s'écrit phonétiquement \href{https://en.oxforddictionaries.com/definition/which}{\exPH{wɪtʃ}}

  \begin{itemize}
  \item\exEN{Pick up a word in the list. \href{https://youtu.be/5fR\_\_LXDkRg}{Which} one?}
  \item\exFR{Choisis un mot dans la liste. Lequel ?}
  \end{itemize}

  \youglish{which}
  
\end{enumerate}

\subsection{\son{e} noté aussi \son{ɛ}}\label{sec:sone}

Ce son a pour nom technique :

\begin{itemize}
\item \exEN{Close-Mid Front Unrounded Vowel.}
\item \exFR{à traduire.}
\end{itemize}

\indicsound

\properukus{https://youtu.be/d98t4b3XLjg}{https://youtu.be/xKxV8XfigaE}

\begin{enumerate}
\item \exEN{\href{http://www.wordreference.com/enfr/bed}{bed}} qui s'écrit
  phonétiquement
  \href{https://en.oxforddictionaries.com/definition/bed}{\exPH{bɛd}}

  \begin{itemize}
  \item\exEN{It's time to go to \href{https://youtu.be/urARKkLo6MY}{bed}.}
  \item\exFR{C'est l'heure d'aller se coucher.}
  \end{itemize}

  \youglish{bed}
  
\item \exEN{\href{http://www.wordreference.com/enfr/bread}{bread}} qui s'écrit phonétiquement \href{https://en.oxforddictionaries.com/definition/bread}{\exPH{brɛd}}
  
  \begin{itemize}
  \item\exEN{French people are famous for their \href{https://youtu.be/Ynm9Wrznz4I}{bread}.}
  \item\exFR{Les Français sont célèbres pour leur pain.}
  \end{itemize}

  \youglish{bread}
  
\item \exEN{\href{http://www.wordreference.com/enfr/said}{said}} qui s'écrit
  phonétiquement
  \href{https://en.oxforddictionaries.com/definition/said}{\exPH{sɛd}}

  \begin{itemize}
  \item\exEN{\href{https://www.azlyrics.com/lyrics/beatles/yesterday.html}{Yesterday} you said that same thing.}
  \item\exFR{Hier tu as dit cette même chose.}
  \end{itemize}

  \youglish{said}
  
\item \exEN{\href{http://www.wordreference.com/enfr/friend}{friend}} qui
  s'écrit phonétiquement
  \href{https://en.oxforddictionaries.com/definition/friend}{\exPH{frɛnd}}

  \begin{itemize}
  \item\exEN{\href{https://youtu.be/q-9kPks0IfE}{I'll be there for you} my friend.}
  \item\exFR{Je serais là pour toi mon ami(e).}
  \end{itemize}

  \youglish{friend}

% % \item Mot français qui utilise le même son
% % \label{sec:motfr}
% % \exFR{\href{http://www.wordreference.com/fren/m\%25C3\%25A8re}{mère}} qui s'écrit phonétiquement \href{http://www.larousse.fr/dictionnaires/francais-\exEN{anglais}/m\%25c3\%25a8re/50499}{\exPH{mεr}}
  
\end{enumerate}

\subsection{\son{æ} noté aussi parfois \son{\href{https://en.oxforddictionaries.com/definition/cat}{a}}}\label{sec:sonae}

Ce son a pour nom technique :

\begin{itemize}
\item \exEN{Near-Open Front Unrounded Vowel.}
\item \exFR{à traduire.}
\end{itemize}

\indicsound

\properukus{https://youtu.be/NavmTDkd8Z8}{https://youtu.be/mynucZiy-Ug}

\begin{enumerate}
\item \exEN{\href{http://www.wordreference.com/enfr/bat}{bat}} qui s'écrit
  phonétiquement
  \href{https://en.oxforddictionaries.com/definition/bat}{\exPH{bat}}
  
  \begin{itemize}
  \item\exEN{Have you ever noticed that \href{https://www.youtube.com/watch?v=O24Ui015YXM}{Batman} means the \href{https://youtu.be/24howVwYgHY}{man} who
      is a \href{https://youtu.be/eozL5n2Plmc}{bat}?}
  \item\exFR{As-tu déjà remarqué que Batman signifie l'homme qui
      est une chauve-souris ?}
  \end{itemize}

  \youglish{bat}
  
\item \exEN{\href{http://www.wordreference.com/enfr/cat}{cat}} qui s'écrit
  phonétiquement
  \href{https://en.oxforddictionaries.com/definition/cat}{\exPH{kat}}

  \begin{itemize}
  \item\exEN{What \href{https://youtu.be/7FjChUY0zgQ}{about} Catwoman? Is she a \href{https://youtu.be/eNQazP-wdj4}{cat}?}
  \item\exFR{Qu'en est-il de Catwoman ? Est-elle une chatte ?}
  \end{itemize}

  \youglish{cat}
  
\item \exEN{\href{http://www.wordreference.com/enfr/that}{that}} qui s'écrit phonétiquement \href{https://en.oxforddictionaries.com/definition/that}{\exPH{ðat}} lorsqu'il est considéré comme
un pronom, un déterminant, un adverbe. En revanche, en tant que
conjonction il se prononce parfois différemment
\href{https://en.oxforddictionaries.com/definition/that}{\exPH{ðət}}

\begin{itemize}
\item\exEN{\href{https://youtu.be/HAlz5TiKOCM}{That} house is really big.}
\item\exFR{Cette maison est vraiment grande.}
\end{itemize}

\youglish{that}

\item \exEN{\href{http://www.wordreference.com/enfr/hand}{hand}} qui s'écrit
  phonétiquement
  \href{https://en.oxforddictionaries.com/definition/hand}{\exPH{hand}}

  \begin{itemize}
  \item\exEN{Maradona had scored with his \href{https://youtu.be/KDKBY9FqwQg}{hand} during a famous
      match Argentina versus England.}
  \item\exFR{Maradona avait marqué avec sa main durant un célèbre
      match Argentine contre Angleterre.}
  \end{itemize}

  \youglish{hand}
  
\end{enumerate}

\section{Centre Vowels (langue relativement plate)}\label{sec:centvow}

\speechvow{4}

\subsection{\son{ə} qui se note aussi \son{ɜ} }\label{sec:sonenv}

Ce son a pour nom technique :

\begin{itemize}
\item \exEN{Mid-Central Vowel.}
\item \exFR{Voyelle mi-centrale.}
\end{itemize}

\indicsound

\properukus{https://youtu.be/RVvn6204I_Y}{https://youtu.be/m1mDSUSwNls}

\begin{enumerate}
\item \exEN{\href{http://www.wordreference.com/enfr/ago}{ago}} qui s'écrit
  phonétiquement
  \href{https://en.oxforddictionaries.com/definition/ago}{\exPH{əˈɡəʊ}}

  \begin{itemize}
  \item\exEN{I started to learn English when I was in Middle School
      25 years \href{https://youtu.be/RO4fWbM3WA8}{ago}!}
  \item\exFR{J'ai commencé à apprendre l'\exEN{Anglais} quand j'étais au
      Collège il y a 25 ans.}
  \end{itemize}
  
\item \exEN{\href{http://www.wordreference.com/enfr/today}{today}} qui
  s'écrit phonétiquement
  \href{https://en.oxforddictionaries.com/definition/today}{\exPH{təˈdeɪ}}

  \begin{itemize}
  \item\exEN{\href{https://youtu.be/yCSLK0WCUd8}{Today} is Wednesday.}
  \item\exFR{Aujourd'hui c'est mercredi.}
  \end{itemize}
  
\item \exEN{\href{http://www.wordreference.com/enfr/rhythm}{rhythm}} (attention il y a bien 2 h) qui s'écrit phonétiquement
\href{https://en.oxforddictionaries.com/definition/rhythm}{\exPH{ˈrɪð(ə)m}}

\begin{itemize}
\item\exEN{Did you know that any language has got its own \href{https://youtu.be/XQJVoS3SlX0}{rhythm}?}
\item\exFR{Saviez-vous que chaque langue a son propre rythme ?}
\end{itemize}

\item \exEN{\href{http://www.wordreference.com/enfr/supply}{supply}} qui
  s'écrit phonétiquement
  \href{https://en.oxforddictionaries.com/definition/supply}{\exPH{səˈplʌɪ}}

  \begin{itemize}
  \item\exEN{Do not worry I will always \href{https://youtu.be/qEd6QUbK2Mw}{supply} you with multimedia
      documents, audio links, videos, texts\dots}
  \item\exFR{Ne vous inquiétez pas, je vous fournirai toujours des
      documents multimédias, des liens audios, des vidéos, des textes\dots}
  \end{itemize}
\end{enumerate}

\subsection{\son{ɜː} qui se  note aussi \son{əː} }\label{sec:sonenvlong}

Ce son a pour nom technique :

\begin{itemize}
\item \exEN{Open-Mid Central Unrounded Vowel.}
\item \exFR{Voyelle mi-ouverte centrale non arrondie.}
\end{itemize}

\indicsound

\properukus{https://youtu.be/dweBtpz3gco}{https://youtu.be/Ehn6XixUBKs}

\begin{enumerate}
\item \exEN{\href{http://www.wordreference.com/enfr/bird}{bird}} qui s'écrit
  phonétiquement
  \href{https://en.oxforddictionaries.com/definition/bird}{\exPH{bəːd}}
  
  \begin{itemize}
  \item\exEN{\href{https://genius.com/The-beatles-free-as-a-bird-lyrics}{Free as a bird.}}
  \item\exFR{Libre comme l'air (littéralement : libre tel un
      oiseau)}
  \end{itemize}
  
\item \exEN{\href{http://www.wordreference.com/enfr/turn}{turn}} qui s'écrit
  phonétiquement
  \href{https://en.oxforddictionaries.com/definition/turn}{\exPH{təːn}}
  
  \begin{itemize}
  \item\exEN{\href{https://youtu.be/WLTI2rWAlV4}{Turn} off your TV, actually, you should sell it.}
  \item\exFR{Éteins ta télé, en fait, tu devrais la vendre.}
  \end{itemize}
  
\item \exEN{\href{http://www.wordreference.com/enfr/worse}{worse}} qui
  s'écrit phonétiquement
  \href{https://en.oxforddictionaries.com/definition/worse}{\exPH{wəːs}}

  \begin{itemize}
  \item\exEN{I don't know if watching silly cat videos on YouTube
      is \href{https://youtu.be/JHWhzS0zdOc}{worse} than watching TV, but you won't improve your
      intellectual level by doing so.}
  \item\exFR{Je ne sais pas si regarder des vidéos débiles de chat
      sur YouTube est pire que de regarder la télé, mais tu
      n'augmenteras pas ton niveau intellectuel en le faisant.}
  \end{itemize}
  
\item \exEN{\href{http://www.wordreference.com/enfr/learn}{learn}} qui
  s'écrit phonétiquement
  \href{https://en.oxforddictionaries.com/definition/learn}{\exPH{ləːn}}
  
  \begin{itemize}
  \item\exEN{If you want to \href{https://youtu.be/1xXs7MAsB0w}{learn} English, you need to \href{https://youtu.be/wmCAKUFKZ7Y}{practice}
      the sounds.}
  \item\exFR{Si tu veux apprendre l'\exEN{anglais}, il faut que tu
      pratiques les sons.}
  \end{itemize}
  
\end{enumerate}

\subsection{\son{ʌ}}\label{sec:sonup}

Ce son a pour nom technique :

\begin{itemize}
\item \exEN{Open Mid-Back Unrounded Vowel.}
\item \exFR{Voyelle arrière mi-ouverte non arrondie.}
\end{itemize}

\indicsound

\properukus{https://youtu.be/zUpF0pYoTZ8}{https://youtu.be/_63fTgbG-yQ}


\begin{enumerate}
\item \exEN{\href{http://www.wordreference.com/enfr/cup}{cup}} qui s'écrit
  phonétiquement
  \href{https://en.oxforddictionaries.com/definition/cup}{\exPH{kʌp}}

  \begin{itemize}
  \item\exEN{Do you want a \href{https://youtu.be/pjcOzqxu4JQ}{cup} of tea?}
  \item\exFR{Voulez-vous une tasse de thé ?}
  \end{itemize}
  
\item \exEN{\href{http://www.wordreference.com/enfr/something}{something}}
  qui s'écrit phonétiquement
  \href{https://en.oxforddictionaries.com/definition/something}{\exPH{ˈsʌmθɪŋ}}
  
  \begin{itemize}
  \item\exEN{She does \href{https://youtu.be/UelDrZ1aFeY}{something} special with her voice that I can't
      \href{https://genius.com/The-beatles-something-lyrics}{describe}, but I like it.}
  \item\exFR{Elle fait quelque chose de spécial avec sa voix que
      je ne peux pas décrire, mais j'aime ça.}
  \end{itemize}
  
\item \exEN{\href{http://www.wordreference.com/enfr/fun}{fun}} qui s'écrit
  phonétiquement
  \href{https://en.oxforddictionaries.com/definition/fun}{\exPH{fʌn}}

  \begin{itemize}
  \item\exEN{Some studies have shown that having \href{https://youtu.be/KXJNoC6CuYE}{fun} is the best
      way to learn.}
  \item\exFR{Des études ont montré que s'amuser est le meilleur
      moyen pour apprendre.}
  \end{itemize}
  
\item \exEN{\href{http://www.wordreference.com/enfr/luck}{luck}} qui s'écrit
  phonétiquement
  \href{https://en.oxforddictionaries.com/definition/luck}{\exPH{lʌk}}
  
  \begin{itemize}
  \item\exEN{They wish you good \href{https://youtu.be/LQCY2zL0Jr8}{luck} for your \href{https://youtu.be/o61dD6hwrdM}{learning}.}
  \item\exFR{Ils vous souhaietent bonne chance pour votre
      apprentissage.}
  \end{itemize}
  
\end{enumerate}

\subsection{\son{ɑː} }\label{sec:sonalong}

Ce son a pour nom technique :

\begin{itemize}
\item \exEN{Open Back Unrounded Vowel.}
\item \exFR{Voyelle arrière non arrondie.}
\end{itemize}

\indicsound

\properukus{https://youtu.be/1F47WdIjn5U}{https://youtu.be/R5CY1UniS68}


\begin{enumerate}
\item \exEN{\href{http://www.wordreference.com/enfr/father}{father}} qui
  s'écrit phonétiquement
  \href{https://en.oxforddictionaries.com/definition/father}{\exPH{ˈfɑːðə}}
  
  \begin{itemize}
  \item\exEN{My \href{https://youtu.be/MZDAUbeSwNY}{father} used to tell me that you never waste your
      time when you think.}
  \item\exFR{Mon père avait l'habitude de me dire qu'on ne perd
      jamais son temps à réfléchir.}
  \end{itemize}
  
\item \exEN{\href{http://www.wordreference.com/enfr/arm}{arm}} qui s'écrit
  phonétiquement
  \href{https://en.oxforddictionaries.com/definition/arm}{\exPH{ɑːm}}

  \begin{itemize}
  \item\exEN{We are lucky because we have two \href{https://youtu.be/tlhQghmuMf8}{arms} and two legs;
      sorry if one of them is harmed.}
  \item\exFR{Nous avons la chance d'avoir deux bras et deux
      jambes; désolé si l'un d'eux est blessé.}
  \end{itemize}
  
\item \exEN{\href{http://www.wordreference.com/enfr/dance}{dance}} qui
  s'écrit phonétiquement
  \href{https://en.oxforddictionaries.com/definition/dance}{\exPH{dɑːns}}
  
  \begin{itemize}
  \item\exEN{Would you like to \href{https://youtu.be/aagbeWUDe7w}{dance} with me pretty lady?}
  \item\exFR{Veux-tu danser avec moi jolie demoiselle ?}
  \end{itemize}
  
\item \exEN{\href{http://www.wordreference.com/enfr/half}{half}} qui s'écrit
  phonétiquement
  \href{https://en.oxforddictionaries.com/definition/half}{\exPH{hɑːf}}
  
  \begin{itemize}
  \item\exEN{\href{https://youtu.be/XWamnSNgiCM}{Half} time! That's the right moment to get some drinks!}
  \item\exFR{Mi-temps ! C'est le bon moment pour prendre à boire !}
  \end{itemize}
  
\end{enumerate}

\section{Back Vowels (langue vers l'arrière)}\label{sec:backvow}

\speechvow{4}

\subsection{ \son{uː} }\label{sec:ulong}

Ce son a pour nom technique :

\begin{itemize}
\item \exEN{Close Back Rounded Vowel.}
\item \exFR{à traduire}
\end{itemize}

\indicsound

\properukus{https://youtu.be/qPB0Ajjs7nE}{https://youtu.be/lkM6CKBM2ns}

\begin{enumerate}
\item \exEN{\href{http://www.wordreference.com/enfr/too}{too}} qui s'écrit
  phonétiquement
  \href{https://en.oxforddictionaries.com/definition/too}{\exPH{tuː}}
  
  \begin{itemize}
  \item\exEN{I like to speak English, and you? Me \href{https://youtu.be/RaveinO4\_vs}{too}.}
  \item\exFR{J'aime parler \exEN{Anglais}, et toi ? Moi aussi.}
  \end{itemize}
  
\item \exEN{\href{http://www.wordreference.com/enfr/few}{few}} qui s'écrit
  phonétiquement
  \href{https://en.oxforddictionaries.com/definition/few}{\exPH{fjuː}}
  
  \begin{itemize}
  \item\exEN{\href{https://youtu.be/r3TaGhdqEiA}{Few} people understand the key role of phonetics.}
  \item\exFR{Peu de gens comprennent le rôle clé de la phonétique.}
  \end{itemize}
  
\item \exEN{\href{http://www.wordreference.com/enfr/rule}{rule}} qui s'écrit
  phonétiquement
  \href{https://en.oxforddictionaries.com/definition/rule}{\exPH{ruːl}}
  
  \begin{itemize}
  \item\exEN{\href{https://youtu.be/rStL7niR7gs}{Do you want to rule?}}
  \item\exFR{Voulez-vous diriger ?}
  \end{itemize}
  
\item \exEN{\href{http://www.wordreference.com/enfr/lose}{lose}} qui s'écrit
  phonétiquement
  \href{https://en.oxforddictionaries.com/definition/lose}{\exPH{luːz}}
  
  \begin{itemize}
  \item\exEN{You \href{https://youtu.be/UNcCTgA5lzo}{lose} the game this time, do you want to try again?}
  \item\exFR{Vous avez perdu la partie cette fois, voulez-vous
      essayer à nouveau ?}
  \end{itemize}
  
\end{enumerate}

\subsection{\son{ʊ} }\label{sec:omega}

Ce son a pour nom technique :

\begin{itemize}
\item \exEN{Near-Close Near-Back Vowel.}
\item \exFR{à traduire}
\end{itemize}

\indicsound

\properukus{https://youtu.be/5lOF-zRg8x0}{https://youtu.be/moLTR-dLQQY}

\begin{enumerate}
\item \exEN{\href{http://www.wordreference.com/enfr/good}{good}} qui
  s'écrit phonétiquement
  \href{https://en.oxforddictionaries.com/definition/good}{\exPH{ɡʊd}}
  
  \begin{itemize}
  \item\exEN{Your book is \href{https://youtu.be/o3TQSaqHBtM}{good}.}
  \item\exFR{Votre le livre est bon.}
  \end{itemize}
  
\item \exEN{\href{http://www.wordreference.com/enfr/put}{put}} qui s'écrit
  phonétiquement
  \href{https://en.oxforddictionaries.com/definition/put}{\exPH{pʊt}}
  
  \begin{itemize}
  \item\exEN{\href{https://youtu.be/BSpoa7TsiD0}{Put} your energy in something you like.}
  \item\exFR{Mettez votre énergie dans quelque chose que vous
      aimez.}
  \end{itemize}
  
\item \exEN{\href{http://www.wordreference.com/enfr/would}{would}} qui
  s'écrit phonétiquement
  \href{https://en.oxforddictionaries.com/definition/would}{\exPH{wʊd}}
  
  \begin{itemize}
  \item\exEN{\href{https://youtu.be/wRSNm3pr100}{Would} you like to drink something?}
  \item\exFR{Voulez-vous boire quelque chose ?}
  \end{itemize}
  
\item \exEN{\href{http://www.wordreference.com/enfr/look}{look}} qui s'écrit
  phonétiquement
  \href{https://en.oxforddictionaries.com/definition/look}{\exPH{lʊk}}

  \begin{itemize}
  \item\exEN{\href{https://youtu.be/b4xcpMCPhfE}{Look} at this!}
  \item\exFR{Regarde ça !}
  \end{itemize}
  
\end{enumerate}

\subsection{ \son{ɔː} }\label{sec:oouvert}

Ce son a pour nom technique :

\begin{itemize}
\item \exEN{Open-Mid Back Rounded Vowel.}
\item \exFR{à traduire}
\end{itemize}

\indicsound

\properukus{https://youtu.be/Bc1tCtP2ZSg}{https://youtu.be/pr_KAu-_Hmo}

\begin{enumerate}
\item \exEN{\href{http://www.wordreference.com/enfr/pork}{pork}} qui s'écrit
  phonétiquement
  \href{https://en.oxforddictionaries.com/definition/pork}{\exPH{pɔːk}}
  
  \begin{itemize}
  \item\exEN{Do you eat \href{https://youtu.be/WqTJbyfewzw}{pork}?}
  \item\exFR{Mangez-vous du porc ?}
  \end{itemize}

  \youglish{pork}
  
\item \exEN{\href{http://www.wordreference.com/enfr/law}{law}} qui s'écrit
  phonétiquement
  \href{https://en.oxforddictionaries.com/definition/law}{\exPH{lɔː}}

  \begin{itemize}
  \item\exEN{\href{https://youtu.be/us5CUAsH0U0}{Hackers like to say: code is law.}}
  \item\exFR{Les hackers aiment dire que le code est la loi.}
  \end{itemize}

  \youglish{law}
  
\item \exEN{\href{http://www.wordreference.com/enfr/taught}{taught}} qui
  s'écrit phonétiquement
  \href{https://en.oxforddictionaries.com/definition/taught}{\exPH{tɔːt}}
  
  \begin{itemize}
  \item\exEN{I \href{https://youtu.be/U2BG2\_K2fGk}{taught} you how to write English phonetics yesterday.}
  \item\exFR{Hier je t'ai enseigné comment écrire la phonétique
      \exEN{Anglais}e.}
  \end{itemize}

  \youglish{taught}
  
\item \exEN{\href{http://www.wordreference.com/enfr/thought}{thought}} qui
  s'écrit phonétiquement
  \href{https://en.oxforddictionaries.com/definition/thought}{\exPH{θɔːt}}
  
  \begin{itemize}
  \item\exEN{Tell me your \href{https://youtu.be/8kR-GDbYHhc}{thoughts}.}
  \item\exFR{Raconte-moi tes pensées.}
  \end{itemize}

  \youglish{thought}
  
\end{enumerate}

\subsection{ \son{ɒ} }\label{sec:oa}

Ce son a pour nom technique :

\begin{itemize}
\item \exEN{Open Back Rounded Vowel.}
\item \exFR{à traduire}
\end{itemize}

\indicsound

\begin{center}
  \uks{https://youtu.be/A3l-yWQfIW4}
\end{center}


\begin{enumerate}
\item \exEN{\href{http://www.wordreference.com/enfr/got}{got}} qui s'écrit
  phonétiquement
  \href{https://en.oxforddictionaries.com/definition/got}{\exPH{ɡɒt}}

  \begin{itemize}
  \item\exEN{I \href{https://youtu.be/Bo09BiPb24Y}{got} you. (slang: \href{https://youtu.be/EWRaAbVUkjA}{Gotcha})}
  \item\exFR{Je t'ai eu.} (argot : Gotcha)
  \end{itemize}

  \youglish{got}
  
\item \exEN{\href{http://www.wordreference.com/enfr/watch}{watch}} qui
  s'écrit phonétiquement
  \href{https://en.oxforddictionaries.com/definition/watch}{\exPH{wɒtʃ}}
  
  \begin{itemize}
  \item\exEN{\href{https://youtu.be/qOs8MagOfwg}{Watch} this video carefully.}
  \item\exFR{Regardez attentivement cette vidéo.}
  \end{itemize}

  \youglish{watch}
  
\item \exEN{\href{http://www.wordreference.com/enfr/rob}{rob}} qui s'écrit
  phonétiquement
  \href{https://en.oxforddictionaries.com/definition/rob}{\exPH{rɒb}}

  \begin{itemize}
  \item\exEN{Are you planning to \href{https://youtu.be/X3uZ0Gf104A}{rob} a bank? I discourage you to do
      that.}
  \item\exFR{Êtes-vous en train d'envisager de cambrioler une
      banque ? Je vous déconseille de faire ça.}
  \end{itemize}

  \youglish{rob}
  
\item \exEN{\href{http://www.wordreference.com/enfr/top}{top}} qui s'écrit
  phonétiquement
  \href{https://en.oxforddictionaries.com/definition/top}{\exPH{tɒp}}

  \begin{itemize}
  \item\exEN{\href{https://youtu.be/gPaD513xWOY}{Top} videos are sometime very boring.}
  \item\exFR{Les vidéos de top sont parfois très ennuyeuses.}
  \end{itemize}

  \youglish{top}
  
\end{enumerate}

% \section{Diphthong Vowels}
% \label{sec:diphtong}

% Il y a 7 diphtongues en \exEN{anglais}.

% \subsection{ [eɪ] }
% \label{sec:ei}

% \begin{enumerate}
% \item \exEN{\href{http://www.wordreference.com/enfr/snake}{snake}} qui
%   s'écrit phonétiquement
%   \href{https://en.oxforddictionaries.com/definition/snake}{\exPH{sneɪk}}. Exemple
%   d'utilisation du mot :

%   \begin{itemize}
%   \item\exEN{\href{https://youtu.be/MOltIVdyAHQ}{Snakes} regularly shed their skin.}
%   \item\exFR{Les serpents perdent régulièrement leur peau.}
%   \end{itemize}
  
% \item \exEN{\href{http://www.wordreference.com/enfr/pay}{pay}} qui s'écrit
%   phonétiquement
%   \href{https://en.oxforddictionaries.com/definition/pay}{\exPH{peɪ}}. Exemple
%   d'utilisation du mot :

%   \begin{itemize}
%   \item\exEN{How much would you be able to \href{https://youtu.be/mBuLm5XeF44}{pay} for additional
%       content?}
%   \item\exFR{Combien seriez-vous capable de payer pour du contenu
%       supplémentaire ?}
%   \end{itemize}
  
% \item \exEN{\href{http://www.wordreference.com/enfr/mail}{mail}} qui s'écrit
%   phonétiquement
%   \href{https://en.oxforddictionaries.com/definition/mail}{\exPH{meɪl}}. Exemple
%   d'utilisation du mot :

%   \begin{itemize}
%   \item\exEN{The post office redirected the \href{https://youtu.be/KX1CSSZa1v0}{mail} to my new address.}
%   \item\exFR{Le bureau de poste a fait suivre le courrier à ma
%       nouvelle adresse.}
%   \end{itemize}
  
% \item \exEN{\href{http://www.wordreference.com/enfr/great}{great}} qui
%   s'écrit phonétiquement
%   \href{https://en.oxforddictionaries.com/definition/great}{\exPH{ɡreɪt}}. Exemple
%   d'utilisation du mot :

%   \begin{itemize}
%   \item\exEN{Your content is \href{https://youtu.be/e0qM84DWXzA}{great}!}
%   \item\exFR{Ton contenu est génial !}
%   \end{itemize}
  
% \end{enumerate}

% \subsection{ [ɔɪ] }
% \label{sec:oouverti}

% \begin{enumerate}
% \item \exEN{\href{http://www.wordreference.com/enfr/toy}{toy}} qui s'écrit
%   phonétiquement
%   \href{https://en.oxforddictionaries.com/definition/toy}{\exPH{tɔɪ}}. Exemple
%   d'utilisation du mot :

%   \begin{itemize}
%   \item\exEN{The little boy was delighted with all his \href{https://youtu.be/1qbuZhVUj\_g}{toys}.}
%   \item\exFR{Le petit garçon était enchanté par tous ses jouets.}
%   \end{itemize}
  
% \item \exEN{\href{http://www.wordreference.com/enfr/choice}{choice}} qui
%   s'écrit phonétiquement
%   \href{https://en.oxforddictionaries.com/definition/choice}{\exPH{tʃɔɪs}}. Exemple
%   d'utilisation du mot :

%   \begin{itemize}
%   \item\exEN{Looking at my additional content is your \href{https://youtu.be/qBfeK\_IIHag}{choice}.}
%   \item[{française}] Regarder mon contenu supplémentaire est votre choix.
%   \end{itemize}
  
% \item \exEN{\href{http://www.wordreference.com/enfr/joy}{joy}} qui s'écrit
%   phonétiquement
%   \href{https://en.oxforddictionaries.com/definition/joy}{\exPH{dʒɔɪ}}. Exemple
%   d'utilisation du mot :

%   \begin{itemize}
%   \item\exEN{The music creates a sensation of \href{https://youtu.be/-GjW1pSYgUk}{joy} and playfulness.}
%   \item\exFR{La musique crée une sensation de joie et de gaieté.}
%   \end{itemize}
  
% \item \exEN{\href{http://www.wordreference.com/enfr/oyster}{oyster}} qui
%   s'écrit phonétiquement
%   \href{https://en.oxforddictionaries.com/definition/oyster}{\exPH{ˈɔɪstə}}. Exemple
%   d'utilisation du mot :

%   \begin{itemize}
%   \item\exEN{Inside the \href{https://youtu.be/PVn6b9QQZeM}{oyster}, I found a pearl.}
%   \item\exFR{À l'intérieur de l'huître, j'ai trouvé une perle.}
%   \end{itemize}
  
% \end{enumerate}

% \subsection{ [aɪ] }
% \label{sec:ai}
% \begin{enumerate}
% \item \exEN{\href{http://www.wordreference.com/enfr/my}{my}} qui s'écrit
%   phonétiquement
%   \href{https://dictionary.cambridge.org/dictionary/english/my}{\exPH{maɪ}}. Exemple
%   d'utilisation du mot :

%   \begin{itemize}
%   \item\exEN{\href{https://youtu.be/SMwEkjcEACM}{My} content is made to help you \href{https://www.youtube.com/watch?v=m\_uWS6K-VF8\&list=PL0J5xb8JH3VukoRHgk86Yr9BSVeBewCuZ}{progress in English}.}
%   \item\exFR{Mon contenu est fait pour vous aider à progresser en
%       \exEN{anglais}.}
%   \end{itemize}
  
% \item \exEN{\href{http://www.wordreference.com/enfr/while}{while}} qui
%   s'écrit phonétiquement
%   \href{https://dictionary.cambridge.org/dictionary/english/while}{\exPH{waɪl}}. Exemple
%   d'utilisation du mot :

%   \begin{itemize}
%   \item\exEN{She partied \href{https://youtu.be/8q182kWAhiM}{while} I worked.}
%   \item\exFR{Elle faisait la fête alors que je travaillais.}
%   \end{itemize}
  
% \item \exEN{\href{http://www.wordreference.com/enfr/might}{might}} qui
%   s'écrit phonétiquement
%   \href{https://dictionary.cambridge.org/dictionary/english/might}{\exPH{maɪt}}. Exemple
%   d'utilisation du mot :

%   \begin{itemize}
%   \item\exEN{Hurricanes show us the \href{https://youtu.be/Nqlr35WnqTk}{might} of nature.}
%   \item\exFR{Les ouragans nous démontrent la puissance de la
%       nature.}
%   \end{itemize}
  
% \item \exEN{\href{http://www.wordreference.com/enfr/life}{life}} qui s'écrit
%   phonétiquement
%   \href{https://dictionary.cambridge.org/dictionary/english/life}{\exPH{laɪf}}. Exemple
%   d'utilisation du mot :

%   \begin{itemize}
%   \item\exEN{The author withdrew from public \href{https://youtu.be/zyKGKoGACVk}{life}.}
%   \item\exFR{L'auteur s'est retiré de la vie publique.}
%   \end{itemize}
  
% \end{enumerate}
% \subsection{ [əʊ] }
% \label{sec:enenvomegaenv}

% \begin{enumerate}
% \item \exEN{\href{http://www.wordreference.com/enfr/alone}{alone}} qui
%   s'écrit phonétiquement
%   \href{https://en.oxforddictionaries.com/definition/alone}{\exPH{əˈləʊn}}. Exemple
%   d'utilisation du mot :

%   \begin{itemize}
%   \item\exEN{I experience real \href{https://youtu.be/cnsk7iXFCtY}{joy} when I am alone in nature.}
%   \item\exFR{Je ressens une joie réelle quand je suis seul dans la
%       nature.}
%   \end{itemize}
  
% \item \exEN{\href{http://www.wordreference.com/enfr/goat}{goat}} qui s'écrit
%   phonétiquement
%   \href{https://en.oxforddictionaries.com/definition/goat}{\exPH{ɡəʊt}}. Exemple
%   d'utilisation du mot :

%   \begin{itemize}
%   \item\exEN{Behind a door there is a sports car and behind each of
%       the other two there is a \href{https://youtu.be/pEHWbpy-EpI}{goat}.}
%   \item\exFR{Derrière une porte il y a une voiture de sport et
%       derrière chacune des deux autres il y a une chèvre.}
%   \end{itemize}
  
% \item \exEN{\href{http://www.wordreference.com/enfr/hope}{hope}} qui s'écrit
%   phonétiquement
%   \href{https://en.oxforddictionaries.com/definition/hope}{\exPH{həʊp}}. Exemple
%   d'utilisation du mot :

%   \begin{itemize}
%   \item\exEN{I \href{https://youtu.be/\_pKcv0Fml-A}{hope} you will enjoy your stay.}
%     \item\exFR{J'espère que vous apprécierez votre séjour.}
%     \end{itemize}
    
% \item \exEN{\href{http://www.wordreference.com/enfr/road}{road}} qui s'écrit
%   phonétiquement
%   \href{https://en.oxforddictionaries.com/definition/road}{\exPH{rəʊd}}. Exemple
%   d'utilisation du mot :

%   \begin{itemize}
%   \item\exEN{\href{https://youtu.be/jzmy6iUGDo8}{Body like a back road.}}
%   \item\exFR{Un corps comme une route de retour.}
%   \end{itemize}
  
% \end{enumerate}
% \subsection{ [aʊ] }
% \label{sec:aomega}

% \begin{enumerate}
% \item \exEN{\href{http://www.wordreference.com/enfr/now}{now}} qui s'écrit
%   phonétiquement
%   \href{https://en.oxforddictionaries.com/definition/now}{\exPH{naʊ}}. Exemple
%   d'utilisation du mot :

%   \begin{itemize}
%   \item\exEN{I am \href{https://youtu.be/xcpxjx2fy\_E}{now} completely free and unencumbered.}
%   \item\exFR{Je suis désormais complètement libre et sans contrainte.}
%   \end{itemize}
  
% \item \exEN{\href{http://www.wordreference.com/enfr/round}{round}} qui
%   s'écrit phonétiquement
%   \href{https://en.oxforddictionaries.com/definition/round}{\exPH{raʊnd}}. Exemple
%   d'utilisation du mot :

%   \begin{itemize}
%   \item\exEN{The boxer won the fight in the second \href{https://youtu.be/oGTBax-Cu4Q}{round}.}
%   \item\exFR{Le boxeur a gagné le combat au deuxième round.}
%   \end{itemize}
  
% \item \exEN{\href{http://www.wordreference.com/enfr/mouth}{mouth}} qui
%   s'écrit phonétiquement
%   \href{https://en.oxforddictionaries.com/definition/mouth}{\exPH{maʊθ}}. Exemple
%   d'utilisation du mot :

%   \begin{itemize}
%   \item\exEN{In order to produce a vowel you need to open your
%       \href{https://youtu.be/kkDHKSNrJ5g}{mouth}.}
%   \item\exFR{Afin de produire une voyelle vous devez ouvrir votre
%       bouche.}
%   \end{itemize}
  
% \item \exEN{\href{http://www.wordreference.com/enfr/brown}{brown}} qui
%   s'écrit phonétiquement
%   \href{https://en.oxforddictionaries.com/definition/brown}{\exPH{braʊn}}. Exemple
%   d'utilisation du mot :

%   \begin{itemize}
%   \item\exEN{\href{https://youtu.be/OwTXBBU0JLo}{Brown} is just a colour.}
%   \item\exFR{Le marron est juste une couleur.}
%   \end{itemize}
  
% \end{enumerate}
% \subsection{ [ɪə] }
% \label{sec:ieenv}

% \begin{enumerate}
% \item \exEN{\href{http://www.wordreference.com/enfr/weird}{weird}} qui s'écrit phonétiquement \href{https://en.oxforddictionaries.com/definition/weird}{\exPH{wɪəd}}. Exemple d'utilisation du mot :

%   \begin{itemize}
%   \item\exEN{He always has \href{https://youtu.be/fcdUXnt87ng}{weird} dreams that \href{https://youtu.be/FikYhD7bXYE}{nobody} understands.}
%   \item\exFR{Il fait toujours des rêves bizarres que personne ne
%       comprend.}
%   \end{itemize}
  
% \item \exEN{\href{http://www.wordreference.com/enfr/beer}{beer}} qui s'écrit
%   phonétiquement
%   \href{https://en.oxforddictionaries.com/definition/beer}{\exPH{bɪə}}. Exemple
%   d'utilisation du mot :

%   \begin{itemize}
%   \item\exEN{Football supporters usually drink \href{https://youtu.be/I1fsk4k-bOs}{beer}.}
%   \item\exFR{Les supporters de foot boivent habituellement de la
%       bière (attention à consommer avec modération).}
%   \end{itemize}
  
% \item \exEN{\href{http://www.wordreference.com/enfr/near}{near}} qui s'écrit
%   phonétiquement
%   \href{https://en.oxforddictionaries.com/definition/near}{\exPH{nɪə}}. Exemple
%   d'utilisation du mot :

%   \begin{itemize}
%   \item\exEN{UK is \href{https://youtu.be/xIS9K-bNt3M}{near} from France.}
%   \item\exFR{Le Royaume-Uni est proche de la France.}
%   \end{itemize}
  
% \item \exEN{\href{http://www.wordreference.com/enfr/steer}{steer}} qui
%   s'écrit phonétiquement
%   \href{https://en.oxforddictionaries.com/definition/steer}{\exPH{stɪə}}. Exemple
%   d'utilisation du mot :

%   \begin{itemize}
%   \item\exEN{The politician \href{https://youtu.be/z\_vSRFODAxU}{steered} the conversation to a different
%       topic.}
%   \item\exFR{L'homme politique a orienté la conversation vers un autre sujet.}
%   \end{itemize}
  
% \end{enumerate}

% \subsection{ \textcolor{red}{[eə]} qui s'écrit aussi \textcolor{red}{[ɛə]} }
% \label{sec:eeteenv}

% \begin{enumerate}
% \item \exEN{\href{http://www.wordreference.com/enfr/bear}{bear}} qui s'écrit
%   phonétiquement
%   \href{https://dictionary.cambridge.org/dictionary/english/bear}{\exPH{bɛə}}. Exemple
%   d'utilisation du mot :

%   \begin{itemize}
%   \item\exEN{This noise is difficult to \href{https://youtu.be/NJ6jv\_lPBN8}{bear}.}
%   \item\exFR{Ce bruit est difficile à supporter.}
%   \end{itemize}
  
% \item \exEN{\href{http://www.wordreference.com/enfr/rare}{rare}} qui s'écrit
%   phonétiquement
%   \href{https://dictionary.cambridge.org/dictionary/english/rare}{\exPH{rɛə}}. Exemple
%   d'utilisation du mot :

%   \begin{itemize}
%   \item\exEN{The consultant is an expert in \href{https://youtu.be/hPncU3924fU}{rare} illnesses.}
%   \item\exFR{Le médecin spécialiste est expert en maladies rares.}
%   \end{itemize}
  
% \item \exEN{\href{http://www.wordreference.com/enfr/there}{there}} qui
%   s'écrit phonétiquement
%   \href{https://dictionary.cambridge.org/dictionary/english/there}{\exPH{ðeər}}. Exemple
%   d'utilisation du mot :

%   \begin{itemize}
%   \item\exEN{My friend is always \href{https://youtu.be/fg9pkAYvrSM}{there} for me when I need her.}
%   \item\exFR{Mon amie est toujours là pour moi quand j'ai besoin
%       d'elle.}
%   \end{itemize}
  
% \item \exEN{\href{http://www.wordreference.com/enfr/care}{care}} qui s'écrit
%   phonétiquement
%   \href{https://dictionary.cambridge.org/dictionary/english/care}{\exPH{keər}}. Exemple
%   d'utilisation du mot :

%   \begin{itemize}
%   \item\exEN{Babies need constant \href{https://youtu.be/ClrSEz\_tBZw}{care}.}
%   \item\exFR{Les bébés ont besoin d'une attention constante.}
%   \end{itemize}
  
% \end{enumerate}