\chapter{Front Vowels (langue vers l'avant)}\label{chap:frontvow}

\speech{4}{voyelles antérieures\CW{https://fr.wikipedia.org/wiki/Voyelle_ant\%C3\%A9rieure}}

\begin{center}
  \begin{figure}[h]
    \centering
    \includegraphics[scale=.75]{../img/Esling_vowel_chart.png}
    \caption[\exEN{Front
     Vowels}]{\exEN{Front
          Vowels}, Source : \href{https://en.wikipedia.org/w/index.php?curid=46937434}{Wikipedia}}
    \label{fig:front-vowels}
  \end{figure}
\end{center}

\newpage
\minitoc
\newpage

\section{Le \son \phon{iː}}\label{sec:ilong}

\hypertarget{ilong}{Ce \son} a pour nom technique\dyse{clear-front-unrounded-vowel} :

\begin{itemize}
\item \exEN{Close Front Unrounded Vowel\CW{https://en.wikipedia.org/wiki/Close_front_unrounded_vowel}.}
\item \exFR{Voyelle fermée antérieure non arrondie\CW{https://fr.wikipedia.org/wiki/Voyelle_ferm\%C3\%A9e_ant\%C3\%A9rieure_non_arrondie}.}
\end{itemize}

\begin{center}
  \begin{figure}[h]
    \centering
    \includegraphics{../img/non-arrondies-wikipedia.png}
    \caption{Voyelles non-arrondies, source : Wikipédia}
    \label{fig:voy-ferm}
  \end{figure}
\end{center}

\indicsound

\properukus{https://youtu.be/EuZa9-QbhG8}{https://youtu.be/PIu5WDIco0I}

\begin{enumerate}
\item \exEN{\href{http://www.wordreference.com/enfr/need}{need}} qui s'écrit en phonétique \href{https://en.oxforddictionaries.com/definition/need}{\phonm{niːd}}

  \begin{itemize}
  \item\exEN{I \href{https://youtu.be/p0quLJutRC8}{need} to work every day if I want to \href{https://www.youtube.com/watch?v=xqMozc4K4pg&list=PLE_vQWWxgaiHUFB8zsbcEYqJL0GHGdMLi}{improve} my level.}
  \item\exFR{Je dois travailler tous les jours si je veux
      améliorer mon niveau.}
  \end{itemize}

  \youglish{need}


\item \exEN{\href{http://www.wordreference.com/enfr/tea}{tea}} qui s'écrit en phonétique \href{https://en.oxforddictionaries.com/definition/tea}{\phonm{tiː}}

  \begin{itemize}
  \item\exEN{\href{https://youtu.be/S8SdWEQg6cE}{Every morning} we are
      used to drinking \href{https://youtu.be/Euh8dY4EU9o}{tea}.}
  \item\exFR{Tous les matins on a l'habitude de boire du thé.}
  \end{itemize}

  \youglish{tea}

\item \exEN{\href{http://www.wordreference.com/enfr/believe}{believe}}
  qui s'écrit phonétiquement
  \href{https://en.oxforddictionaries.com/definition/believe}{\phonm{bɪˈliːv}}
  
  \begin{itemize}
  \item\exEN{\href{https://youtu.be/GIQn8pab8Vc}{I believe I can fly.}}
  \item\exFR{Je crois que je peux voler.}
  \end{itemize}

  \youglish{believe}
  
\item \exEN{\href{http://www.wordreference.com/enfr/see}{see}} qui s'écrit
  phonétiquement
  \href{https://en.oxforddictionaries.com/definition/see}{\phonm{siː}}
  
  \begin{itemize}
  \item\exEN{What You \href{https://youtu.be/Dpf2yHjBVYM}{See} Is What You Get (\href{https://fr.wikipedia.org/wiki/What\_you\_see\_is\_what\_you\_get}{WYSIWYG})}
  \item\exFR{Ce que vous voyez est ce que vous obtenez.}
  \end{itemize}

  \youglish{see}
  
\end{enumerate}
\newpage

\section{Le \son \phon{ɪ}}\label{sec:soni}

\hypertarget{soni}{Ce \son} a pour nom technique\dyse{near-close-near-front-unrounded-vowel} :

\begin{itemize}
\item \exEN{Near-Close Near-Front Unrounded Vowel\CW{https://en.wikipedia.org/wiki/Near-close_near-front_unrounded_vowel}.}
\item \exFR{Voyelle pré-fermée antérieure non arrondie\CW{https://fr.wikipedia.org/wiki/Voyelle_pr\%C3\%A9-ferm\%C3\%A9e_ant\%C3\%A9rieure_non_arrondie}.}
\end{itemize}

\begin{center}
  \begin{figure}[h]
    \centering
    \includegraphics{../img/non-arrondies-wikipedia.png}
    \caption{Voyelles non-arrondies, source : Wikipédia}
    \label{fig:voy-ferm}
  \end{figure}
\end{center}

\indicsound

\properukus{https://youtu.be/7PpuPMrISVc}{https://youtu.be/Ok_HG-0lNCA}

\begin{enumerate}
\item \exEN{\href{http://www.wordreference.com/enfr/england}{England}} qui s'écrit en phonétique \href{https://en.oxforddictionaries.com/definition/england}{\phonm{ˈɪŋɡlənd}}

  \begin{itemize}
  \item\exEN{Last \href{https://youtu.be/j9xSxJRsmx0}{summer} I went to \href{https://youtu.be/QUPBesOdax8}{England}.}
  \item\exFR{L'été dernier je suis allé en Angleterre.}
  \end{itemize}

  \youglish{England}
  
\item \exEN{\href{http://www.wordreference.com/enfr/thin}{thin}} qui s'écrit en phonétique \href{https://en.oxforddictionaries.com/definition/thin}{\phonm{θɪn}}

  \begin{itemize}
  \item\exEN{Usually \href{https://youtu.be/IpRSyVcHu-k}{top models} are \href{https://youtu.be/LekA62H17bo}{thin}.}
  \item\exFR{Habituellement les mannequins sont minces.}
  \end{itemize}

  \youglish{thin}
  
\item \exEN{\href{http://www.wordreference.com/enfr/big}{big}} qui s'écrit phonétiquement \href{https://en.oxforddictionaries.com/definition/big}{\phonm{bɪɡ}}

  \begin{itemize}
  \item\exEN{\href{https://youtu.be/do7w0j4ybeY}{New York} has a nickname: the \href{https://youtu.be/Jha4OkG-ixw}{Big} Apple.}
  \item\exFR{New York a un surnom : la grosse pomme.}
  \end{itemize}

  \youglish{big}
  
\item \exEN{\href{http://www.wordreference.com/enfr/which}{which}} qui s'écrit phonétiquement \href{https://en.oxforddictionaries.com/definition/which}{\phonm{wɪtʃ}}

  \begin{itemize}
  \item\exEN{\href{https://youtu.be/DN74ZuSrtxY}{Pick up} a glass from the table. \href{https://youtu.be/5fR\_\_LXDkRg}{Which} one?}
  \item\exFR{Choisis un verre sur la table. Lequel ?}
  \end{itemize}

  \youglish{which}
  
\end{enumerate}
\newpage

\section{Le \son \phon{ɛ}  parfois noté \phon{\href{https://dictionary.cambridge.org/dictionary/english/bed}{e}}
  }\label{sec:sone}

Il est important de préciser que le \son \phon{e} n'existe pas en anglais
pour la bonne et simple raison que ce symbole phonétique représente le
<<~\exFR{é}~>> comme dans le mot français \exFR{beauté}. Je n'ai pas réussi à
trouver d'explication suffisamment détaillée si ce n'est une <<~raison
historique~>> évoquée sur une page de
\href{http://teflpedia.com/IPA_phonetic_symbol_\%E3\%80\%9A\%C9\%9B\%E3\%80\%9B}{teflpedia}\footnote{Teflpedia
est un site inspiré de Wikipédia consacré au TEFL: Teaching English as
a Foreign Language (Enseignement de l'Anglais comme Langue \'Etrangère
équivalent du FLE : Français Langue \'Etrangère).}.

\begin{center}
  \begin{figure}[h]
    \centering
    \includegraphics[scale=.75]{../img/teflopedia-tab-1}
    \caption[Quel symbole phonétique pour le son <<~\exFR{è}~>>]{Démocratiquement
      la notation \textcolor{teal}{[ɛ]} est préférable à l'ancienne \textcolor{teal}{[e]}}
    \label{fig:teflpedia-1}
  \end{figure}
\end{center}

Paradoxalement, alors que le tableau de la
figure~\ref{fig:teflpedia-1} montre clairement que la majorité des
dictionnaires utilisent à raison le symbole \textcolor{teal}{[ɛ]},
ils\footnote{teflpedia} font partie de la minorité qui persite à
utiliser le symbole \textcolor{teal}{[e]}. Ceci est un problème pour
nous autres francophones parce que ce symbole \textcolor{teal}{[e]}
nous sert à représenter le son <<~\exFR{é}~>> qui n'existe pas en
anglais. Ce détail est \underline{très important} parce que le but de
l'API est précisément de fournir un alphabet international afin que
tous les linguistes (et les utilisateurs de la phonétique) puissent se 
comprendre. Cet ensemble de symboles ne sert pas uniquement pour
apprendre à prononcer l'anglais. Il est valable pour toutes les
langues humaines (y compris celles qui n'ont pas d'alphabet). Alors je
trouve ça très dérangeant qu'un groupe qui a vocation à enseigner
l'anglais aux étrangers persiste à maintenir la confusion. Hélas, ils
ne sont pas les seuls et certains manuels faits par des francophones
pour des francophones perpétuent l'erreur\footnote{Ce qui est d'autant
  plus blâmable puisqu'il s'agit d'une négation de la phonétique
  française !}.

\notation

\hypertarget{sone}{Ce \son} a pour nom technique\dyse{open-mid-front-unrounded-vowel} :

\begin{itemize}
\item \exEN{Open-Mid Front Unrounded Vowel\CW{https://en.wikipedia.org/wiki/Open-mid_front_unrounded_vowel}.}
\item \exFR{Voyelle mi-ouverte antérieure non arrondie\CW{https://fr.wikipedia.org/wiki/Voyelle_mi-ouverte_ant\%C3\%A9rieure_non_arrondie}.}
\end{itemize}

\begin{center}
  \begin{figure}[h]
    \centering
    \includegraphics{../img/non-arrondies-wikipedia.png}
    \caption{Voyelles non-arrondies, source : Wikipédia}
    \label{fig:voy-ferm}
  \end{figure}
\end{center}

\indicsound

\properukus{https://youtu.be/JhBH_rtOXGA}{https://youtu.be/xKxV8XfigaE}

\begin{enumerate}
\item \exEN{\href{http://www.wordreference.com/enfr/bed}{bed}} qui s'écrit
  phonétiquement
  \href{https://en.oxforddictionaries.com/definition/bed}{\phonm{bɛd}}

  \begin{itemize}
  \item\exEN{\href{https://youtu.be/LnshfLOhr2Q}{It's time to go} to \href{https://youtu.be/urARKkLo6MY}{bed}.}
  \item\exFR{C'est l'heure d'aller se coucher.}
  \end{itemize}

  \youglish{bed}
  
\item \exEN{\href{http://www.wordreference.com/enfr/bread}{bread}} qui s'écrit phonétiquement \href{https://en.oxforddictionaries.com/definition/bread}{\phonm{brɛd}}
  
  \begin{itemize}
  \item\exEN{\href{https://youtu.be/yxgE3ifjZ94}{French people} are famous for their \href{https://youtu.be/Ynm9Wrznz4I}{bread}.}
  \item\exFR{Les Français sont célèbres pour leur pain.}
  \end{itemize}

  \youglish{bread}
  
\item \exEN{\href{http://www.wordreference.com/enfr/said}{said}} qui s'écrit
  phonétiquement
  \href{https://en.oxforddictionaries.com/definition/said}{\phonm{sɛd}}

  \begin{itemize}
  \item\exEN{\href{https://www.azlyrics.com/lyrics/beatles/yesterday.html}{Yesterday} you said that \href{https://youtu.be/9IDogHTQgM4}{same thing}.}
  \item\exFR{Hier tu as dit cette même chose.}
  \end{itemize}

  \youglish{said}
  
\item \exEN{\href{http://www.wordreference.com/enfr/friend}{friend}} qui
  s'écrit phonétiquement
  \href{https://en.oxforddictionaries.com/definition/friend}{\phonm{frɛnd}}

  \begin{itemize}
  \item\exEN{\href{https://youtu.be/q-9kPks0IfE}{I'll be there for you} my \href{https://youtu.be/CY8E6N5Nzec}{friend}.}
  \item\exFR{Je serais là pour toi mon ami(e).}
  \end{itemize}

  \youglish{friend}
  
\end{enumerate}
\newpage

\section{Le \son \phon{æ} noté aussi parfois
\phon{\href{https://en.oxforddictionaries.com/definition/cat}{a}}
  }\label{sec:sonae}

  \notation
  
\hypertarget{sonae}{Ce \son} a pour nom technique\dyse{near-open-front-unrounded-vowel} :

\begin{itemize}
\item \exEN{Near-Open Front Unrounded Vowel\CW{https://en.wikipedia.org/wiki/Near-open_front_unrounded_vowel}.}
\item \exFR{Voyelle ouverte antérieure non arrondie\CW{https://fr.wikipedia.org/wiki/Voyelle_pr\%C3\%A9-ouverte_ant\%C3\%A9rieure_non_arrondie}.}
\end{itemize}

\begin{center}
  \begin{figure}[h]
    \centering
    \includegraphics{../img/non-arrondies-wikipedia.png}
    \caption{Voyelles non-arrondies, source : Wikipédia}
    \label{fig:voy-ferm}
  \end{figure}
\end{center}

\indicsound

\properukus{https://youtu.be/NavmTDkd8Z8}{https://youtu.be/mynucZiy-Ug}

\begin{enumerate}
\item \exEN{\href{http://www.wordreference.com/enfr/bat}{bat}} qui s'écrit
  phonétiquement
  \href{https://dictionary.cambridge.org/dictionary/english/bat}{\phonm{bæt}}
  
  \begin{itemize}
  \item\exEN{Have you ever noticed that \href{https://youtu.be/ulOLYnOthIw}{Batman} means the \href{https://youtu.be/WfGMYdalClU}{man} who
      is a \href{https://youtu.be/eozL5n2Plmc}{bat}?}
  \item\exFR{As-tu déjà remarqué que Batman signifie l'homme qui
      est une chauve-souris ?}
  \end{itemize}

  \youglish{bat}
  
\item \exEN{\href{http://www.wordreference.com/enfr/cat}{cat}} qui s'écrit
  phonétiquement
  \href{https://dictionary.cambridge.org/dictionary/english/cat}{\phonm{kæt}}

  \begin{itemize}
  \item\exEN{What \href{https://youtu.be/7FjChUY0zgQ}{about} Catwoman? Is she a \href{https://youtu.be/eNQazP-wdj4}{cat}?}
  \item\exFR{Qu'en est-il de Catwoman ? Est-elle une chatte ?}
  \end{itemize}

  \youglish{cat}
  
\item \exEN{\href{http://www.wordreference.com/enfr/that}{that}} qui s'écrit phonétiquement \href{https://dictionary.cambridge.org/dictionary/english/that}{\phonm{ðæt}} 
\begin{itemize}
\item\exEN{\href{https://youtu.be/HAlz5TiKOCM}{That} house is \href{https://youtu.be/qYvXk_bqlBk}{really} big.}
\item\exFR{Cette maison est vraiment grande.}
\end{itemize}

\youglish{that}

\item \exEN{\href{http://www.wordreference.com/enfr/hand}{hand}} qui s'écrit
  phonétiquement
  \href{https://dictionary.cambridge.org/dictionary/english/hand}{\phonm{hænd}}

  \begin{itemize}
  \item\exEN{\href{https://youtu.be/-ccNkksrfls}{Maradona} scored with his \href{https://youtu.be/KDKBY9FqwQg}{hand} during a famous
      match between Argentina and England.}
  \item\exFR{Maradona avait marqué avec sa main durant un célèbre
      match entre l'Argentine et l'Angleterre.}
  \end{itemize}

  \youglish{hand}
  
\end{enumerate}

\begin{center}
  \begin{figure}[h]
    \centering
    \includegraphics[scale=.75]{../img/Cardinal_vowel_tongue_position-front.svg.png}
    \caption[\exEN{Idealistic Tongue Positions}]{\exEN{Idealistic Tongue Positions}, Source : \href{https://en.wikipedia.org/wiki/Vowel\#Backness}{Wikipedia}}
    \label{fig:front-vowels-in-the-mouth}
  \end{figure}
\end{center}


\newpage
\minitoc
\newpage

