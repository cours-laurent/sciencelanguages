\chapter{Fricative (une compression constante de
  l'air)}\label{chap:fricative}

\speech{9}{consonnes \exFR{fricatives\CW{https://fr.wikipedia.org/wiki/Consonne_fricative}} (\exEN{fricative\CW{https://en.wikipedia.org/wiki/Fricative_consonant}})}

\newpage
\minitoc
\newpage

\section{Le \son~\phon{f}}\label{sec:f}

  Ce \son a pour nom technique\dyse{voiceless-labiodental-fricative-f} :% #1: lien
                                % vers le blog
  %
  \begin{itemize}%
  \item \exEN{Voiceless labiodental fricative\CW{https://en.wikipedia.org/wiki/Voiceless_labiodental_fricative}.}% #2: sound name, #3: wiki EN
  \item \exFR{Consonne fricative labio-dentale sourde \CW{https://fr.wikipedia.org/wiki/Consonne_fricative_labio-dentale_sourde}.}% #4: nom du son, #5: wiki FR sinon blog voir
                         % package ifthen pour gérer ça
  \end{itemize}

\begin{center}
  \begin{figure}[h]
    \centering
    \includegraphics[scale=.5]{../img/cpp/f-v-cpp}
    \caption{Fricatives f et v, source :~\cite{collins}}
    \label{fig:f-v}
  \end{figure}
\end{center}

  \indicsound%
  %
  \properukus{https://youtu.be/4VcU0zNJUiU}{https://youtu.be/YejZ8gAQAfU}% #6: UK YT, #7: US YT

\begin{enumerate}
\item \exEN{\href{http://www.wordreference.com/enfr/fish}{fish}} qui s'écrit
  phonétiquement
  \href{https://en.oxforddictionaries.com/definition/fish}{\phonm{fɪʃ}}
  
  \begin{itemize}
  \item\exEN{He prefers \href{https://youtu.be/rEm4ynLtGx4}{fish} to \href{https://youtu.be/LnA7Au-DLUM}{meat}.}
  \item\exFR{Il préfère le poisson à la viande.}
  \end{itemize}

  \youglish{fish}

\item \exEN{\href{http://www.wordreference.com/enfr/friday}{friday}} qui
  s'écrit phonétiquement
  \href{https://en.oxforddictionaries.com/definition/friday}{\phonm{ˈfrʌɪdi}}
  
  \begin{itemize}
  \item\exEN{The \href{https://youtu.be/DFeDMHIYtlo}{ship} sailed from the port on \href{https://youtu.be/CLi80PYON-c}{Friday}.}
  \item\exFR{Le bateau a quitté le port vendredi.}
  \end{itemize}

  \youglish{friday}
  
\item \exEN{\href{http://www.wordreference.com/enfr/full}{full}} qui s'écrit
  phonétiquement
  \href{https://en.oxforddictionaries.com/definition/full}{\phonm{fʊl}}
  
  \begin{itemize}
  \item\exEN{The \href{https://youtu.be/LR73DrKX\_bs}{full} report is hundreds of pages \href{https://youtu.be/CwfoyVa980U}{long}.}
  \item\exFR{Le rapport complet fait des centaines de pages.}
  \end{itemize}

  \youglish{full}

\item \exEN{\href{http://www.wordreference.com/enfr/knife}{knife}} qui s'écrit phonétiquement \href{https://en.oxforddictionaries.com/definition/knife}{\phonm{nʌɪf}}

  \begin{itemize}
  \item\exEN{The blunt \href{https://youtu.be/JUyzH9HpkqE}{knife} could not cut the \href{https://youtu.be/LYoAScygp1w}{rope}.}
  \item\exFR{Le couteau émoussé ne pouvait pas couper la corde.}
  \end{itemize}

  \youglish{knife}

\end{enumerate}

\newpage

\section{Le \son~\phon{v}}\label{sec:v}

  Ce \son a pour nom technique\dyse{voiced-labiodental-fricative-v} :% #1: lien
                                % vers le blog
  %
  \begin{itemize}%
  \item \exEN{Voiced labiodental fricative\CW{https://en.wikipedia.org/wiki/Voiced_labiodental_fricative}.}% #2: sound name, #3: wiki EN
  \item \exFR{Consonne fricative labio-dentale voisée \CW{https://fr.wikipedia.org/wiki/Consonne_fricative_labio-dentale_vois\%C3\%A9e}.}% #4: nom du son, #5: wiki FR sinon blog voir
                         % package ifthen pour gérer ça
  \end{itemize}

\begin{center}
  \begin{figure}[h]
    \centering
    \includegraphics[scale=.5]{../img/cpp/f-v-cpp}
    \caption{Fricatives f et v, source :~\cite{collins}}
    \label{fig:f-v}
  \end{figure}
\end{center}

  \indicsound
  %
  \properukus{https://youtu.be/0W5PcWptfYY}{https://youtu.be/nR-K3mrHFv0}% #6: UK YT, #7: US YT
  
\begin{enumerate}
\item \exEN{\href{http://www.wordreference.com/enfr/cave}{cave}} qui s'écrit
  phonétiquement
  \href{https://en.oxforddictionaries.com/definition/cave}{\phonm{keɪv}}
  
  \begin{itemize}
  \item\exEN{\href{https://youtu.be/CqGsg01ycpk}{Plato} is famous for his myth of the \href{https://youtu.be/kZQbkzwwinI}{cave}.}
  \item\exFR{Platon est célèbre pour \son mythe de la caverne.}
  \end{itemize}

  \youglish{cave}
  
\item \exEN{\href{http://www.wordreference.com/enfr/vest}{vest}} qui s'écrit
  phonétiquement
  \href{https://en.oxforddictionaries.com/definition/vest}{\phonm{vɛst}}
  
  \begin{itemize}
  \item\exEN{The \href{https://youtu.be/QNMD29BQ9J4}{committee} was \href{https://youtu.be/E4cjvxydHuU}{vested} with the government's full
      authority.}
  \item\exFR{Le comité était investi de toute l'autorité du gouvernement.}
  \end{itemize}

  \youglish{vest}
  
\item \exEN{\href{http://www.wordreference.com/enfr/view}{view}} qui s'écrit
  phonétiquement
  \href{https://en.oxforddictionaries.com/definition/view}{\phonm{vjuː}}
  
  \begin{itemize}
  \item\exEN{There is a splendid \href{https://youtu.be/gODvA\_SdXCY}{view} from the \href{https://youtu.be/QEO-2jYvkto}{balcony}.}
  \item\exFR{Il y a une vue splendide depuis le balcon.}
  \end{itemize}

  \youglish{view}
  
\item \exEN{\href{http://www.wordreference.com/enfr/village}{village}} qui
  s'écrit phonétiquement
  \href{https://en.oxforddictionaries.com/definition/village}{\phonm{ˈvɪlɪdʒ}}
  
  \begin{itemize}
  \item\exEN{The \href{https://youtu.be/Xq8mt6WuD-E}{village} is peaceful at \href{https://youtu.be/nNa6F_fEtMw}{night}.}
  \item\exFR{Le village est tranquille la nuit.}
  \end{itemize}

  \youglish{village}
  
\end{enumerate}

\newpage

\section{Le \son~\phon{θ}}\label{sec:theta}

\hypertarget{ss}{Ce \son} a pour nom technique\dyse{voiceless-dental-fricative} :% #1: lien
                                % vers le blog
  %
  \begin{itemize}%
  \item \exEN{Voiceless dental fricative\CW{https://en.wikipedia.org/wiki/Voiceless_dental_fricative}.}% #2: sound name, #3: wiki EN
  \item \exFR{Consonne fricative dentale sourde \CW{https://fr.wikipedia.org/wiki/Consonne_fricative_dentale_sourde}.}% #4: nom du son, #5: wiki FR sinon blog voir
                         % package ifthen pour gérer ça
  \end{itemize}%
  %
  \indicsound%
  %
  \properukus{https://youtu.be/Fq1atdudgh8}{https://youtu.be/qC0l6GQZtM4}% #6: UK YT, #7: US YT
  
\begin{enumerate}
\item \exEN{\href{http://www.wordreference.com/enfr/author}{author}} qui
  s'écrit phonétiquement
  \href{https://en.oxforddictionaries.com/definition/author}{\phonm{ˈɔːθə}}
  
  \begin{itemize}
  \item\exEN{I am the \href{https://youtu.be/lyGivD8aJi4}{author} of this \href{https://youtu.be/iUX18JVWNf8}{document}.}
  \item\exFR{Je suis l'auteur de ce document.}
  \end{itemize}

  \youglish{author}
  
\item \exEN{\href{http://www.wordreference.com/enfr/path}{path}} qui s'écrit
  phonétiquement
  \href{https://en.oxforddictionaries.com/definition/path}{\phonm{pɑːθ}}
  
  \begin{itemize}
  \item\exEN{A \href{https://youtu.be/9seb8hddeK4}{fork} in the road splits it into two \href{https://youtu.be/EZdFE-nnyyQ}{paths}.}
  \item\exFR{Un embranchement sur la route la divise en deux sentiers.}
  \end{itemize}

  \youglish{path}
  
\item \exEN{\href{http://www.wordreference.com/enfr/thing}{thing}} qui
  s'écrit phonétiquement
  \href{https://en.oxforddictionaries.com/definition/thing}{\phonm{θɪŋ}}
  
  \begin{itemize}
  \item\exEN{\href{https://youtu.be/tOd6l2g4XTw}{Windsurfing} is not really my \href{https://youtu.be/h-pmsrw8XNE}{thing}; I prefer surfing.}
  \item\exFR{La planche à voile n'est pas vraiment mon truc ; je
      préfère surfer.}
  \end{itemize}

  \youglish{thing}
  
\item \exEN{\href{http://www.wordreference.com/enfr/think}{think}} qui s'écrit phonétiquement \href{https://en.oxforddictionaries.com/definition/think}{\phonm{θɪŋk}}

  \begin{itemize}
  \item\exEN{I \href{https://youtu.be/PpD8OvMTRiE}{think} my solution is the \href{https://youtu.be/UfMnEEystJ0}{best}.}
  \item\exFR{Je considère que ma solution est la meilleure.}
  \end{itemize}

  \youglish{think}

\end{enumerate}

\newpage

\section{Le \son~\phon{ð}}\label{sec:th}

\hypertarget{th}{Ce \son} a pour nom technique\dyse{voiced-dental-fricative} :% #1: lien
                                % vers le blog
  %
  \begin{itemize}%
  \item \exEN{Voiced dental fricative\CW{https://en.wikipedia.org/wiki/Voiced_dental_fricative}.}% #2: sound name, #3: wiki EN
  \item \exFR{Consonne fricative dentale voisée \CW{https://fr.wikipedia.org/wiki/Consonne_fricative_dentale_vois\%C3\%A9e}.}% #4: nom du son, #5: wiki FR sinon blog voir
                         % package ifthen pour gérer ça
  \end{itemize}%
  %
  \indicsound%
  %
  \properukus{https://youtu.be/GdtdTJkRtkE}{https://youtu.be/EZb_EWVCUoE}% #6: UK YT, #7: US YT

\begin{enumerate}
\item \exEN{\href{http://www.wordreference.com/enfr/this}{this}} qui s'écrit phonétiquement \href{https://en.oxforddictionaries.com/definition/this}{\phonm{ðɪs}}

  \begin{itemize}
  \item\exEN{The \href{https://youtu.be/XGAvSsjVA8U}{implementation} of \href{https://youtu.be/KqzlYTmFBGY}{this} principle will, as a
      consequence, generate more data than currently available.}
  \item\exFR{L'application de ce principe va donc générer plus de
      données que ce qui est actuellement disponible.}
  \end{itemize}

  \youglish{this}
  
\item \exEN{\href{http://www.wordreference.com/enfr/other}{other}} qui
  s'écrit phonétiquement
  \href{https://en.oxforddictionaries.com/definition/other}{\phonm{ˈʌðə}}
  
  \begin{itemize}
  \item\exEN{The \href{https://youtu.be/ZhfWiU8wGCc}{woman} was selling apples and \href{https://youtu.be/9gXP8wcICqQ}{other} fruits.}
  \item\exFR{La femme vendait des pommes et d'autres fruits.}
  \end{itemize}

  \youglish{other}
  
\item \exEN{\href{http://www.wordreference.com/enfr/breathe}{breathe}} qui s'écrit phonétiquement \href{https://en.oxforddictionaries.com/definition/breathe}{\phonm{briːð}}

  \begin{itemize}
  \item\exEN{The \href{https://youtu.be/slC-emKLVBs}{air} we \href{https://youtu.be/V8rtJRlLdI8}{breathe} is invisible.}
  \item\exFR{L'air que nous respirons est invisible.}
  \end{itemize}

  \youglish{breathe}
  
\item \exEN{\href{http://www.wordreference.com/enfr/bathe}{bathe}} qui s'écrit phonétiquement \href{https://dictionary.cambridge.org/dictionary/english/bathe}{\phonm{beɪð}}

  \begin{itemize}
  \item\exEN{Can I \href{https://youtu.be/U9V8cx2buG0}{bathe} my baby from the first hours of their
      \href{https://youtu.be/aVvR8ABCPvU}{life}?}
  \item\exFR{Puis-je baigner mon bébé dès ses premières heures de vie~?}
  \end{itemize}

  \youglish{bathe}
  
\end{enumerate}

\newpage

\section{Le \son~\phon{s}}\label{sec:s}

Ce \son a pour nom technique\dyse{voiceless-alveolar-sibilant-s} :% #1: lien
                                % vers le blog
  %
  \begin{itemize}%
  \item \exEN{Voiceless alveolar sibilants\CW{https://en.wikipedia.org/wiki/Voiceless_alveolar_fricative\#Voiceless_alveola}.}% #2: sound name, #3: wiki EN
  \item \exFR{Consonne fricative alvéolaire sourde \CW{https://fr.wikipedia.org/wiki/Consonne_fricative_alv\%C3\%A9olaire_sourde}.}% #4: nom du son, #5: wiki FR sinon blog voir
                         % package ifthen pour gérer ça
  \end{itemize}

\begin{center}
  \begin{figure}[h]
    \centering
    \includegraphics[scale=.5]{../img/lodge/fricative-s-z-lodge}
    \caption{Fricative s et z, source :~\cite{lodge}}
    \label{fig:s-z}
  \end{figure}
\end{center}

  \indicsound%
  %
  \properukus{https://youtu.be/L3vyZaQF8vk}{https://youtu.be/xl-7mSeybmI}% #6: UK YT, #7: US YT
  
\begin{enumerate}
\item \exEN{\href{http://www.wordreference.com/enfr/kiss}{kiss}} qui s'écrit phonétiquement \href{https://en.oxforddictionaries.com/definition/kiss}{\phonm{kɪs}}

  \begin{itemize}
  \item\exEN{The princess \href{https://youtu.be/vMbVzr7WqIo}{kissed} the \href{https://youtu.be/qh7EY3geI0M}{frog}.}
  \item\exFR{La princesse a embrassé la grenouille.}
  \end{itemize}

  \youglish{kiss}
  
\item \exEN{\href{http://www.wordreference.com/enfr/cease}{cease}} qui s'écrit phonétiquement \href{https://en.oxforddictionaries.com/definition/cease}{\phonm{siːs}}

  \begin{itemize}
  \item\exEN{My \href{https://youtu.be/6d-rkoW4COE}{wife} never \href{https://youtu.be/6m9bEMejTKI}{ceases} to amaze me.}
  \item\exFR{Ma femme ne cesse de m'étonner.}
  \end{itemize}

  \youglish{cease}

\item \exEN{\href{http://www.wordreference.com/enfr/sister}{sister}} qui s'écrit phonétiquement \href{https://en.oxforddictionaries.com/definition/sister}{\phonm{ˈsɪstə}}

  \begin{itemize}
  \item\exEN{Many \href{https://youtu.be/SBNB13EeRx4}{sisters} live in the \href{https://youtu.be/U90ATbs49jc}{convent}.}
  \item\exFR{De nombreuses religieuses vivent dans le couvent.}
  \end{itemize}

  \youglish{sister}
  
\item \exEN{\href{http://www.wordreference.com/enfr/sight}{sight}} qui s'écrit phonétiquement \href{https://en.oxforddictionaries.com/definition/sight}{\phonm{sʌɪt}} 

  \begin{itemize}
  \item\exEN{I \href{https://youtu.be/Mo_mgcxGYYE}{witnessed} a strange \href{https://youtu.be/JeiVf30VDDU}{sight} in the street.}
  \item\exFR{J'ai été témoin d'une scène étrange dans la rue.}
  \end{itemize}

  \youglish{sight}
  
\end{enumerate}

\newpage

\section{Le \son~\phon{z}}\label{sec:z}

Ce \son a pour nom technique\dyse{voiced-alveolar-sibilant-z} :% #1: lien
                                % vers le blog
  %
  \begin{itemize}%
  \item \exEN{Voiced alveolar sibilant \CW{https://en.wikipedia.org/wiki/Voiced_alveolar_fricative\#Voiced_alveolar_sibilant}.}% #2: sound name, #3: wiki EN
  \item \exFR{Consonne fricative alvéolaire voisée \CW{https://fr.wikipedia.org/wiki/Consonne_fricative_alv\%C3\%A9olaire_vois\%C3\%A9e}.}% #4: nom du son, #5: wiki FR sinon blog voir
                         % package ifthen pour gérer ça
  \end{itemize}

\begin{center}
  \begin{figure}[h]
    \centering
    \includegraphics[scale=.5]{../img/lodge/fricative-s-z-lodge}
    \caption{Fricative s et z, source :~\cite{lodge}}
    \label{fig:s-z}
  \end{figure}
\end{center}

  \indicsound%
  %
  \properukus{https://youtu.be/7jhEYQI1954}{https://youtu.be/xl-7mSeybmI}% #6: UK YT, #7: US YT
  
\begin{enumerate}
\item \exEN{\href{http://www.wordreference.com/enfr/buzz}{buzz}} qui s'écrit phonétiquement \href{https://en.oxforddictionaries.com/definition/buzz}{\phonm{bʌz}}

  \begin{itemize}
  \item\exEN{The \href{https://youtu.be/hTPXDr9pbak}{news} caused a \href{https://youtu.be/OoQJUNv-Jlg}{buzz} in the audience.}
  \item\exFR{La nouvelle a provoqué l'effervescence du public.}
  \end{itemize}

  \youglish{buzz}
    
\item \exEN{\href{http://www.wordreference.com/enfr/crazy}{crazy}} qui s'écrit phonétiquement \href{https://en.oxforddictionaries.com/definition/crazy}{\phonm{ˈkreɪzi}}

  \begin{itemize}
  \item\exEN{My \href{https://youtu.be/SmXNnizCLyw}{aunt} is \href{https://youtu.be/U0EW0s1fN-8}{crazy} about her cats.}
  \item\exFR{Ma tante est dingue de ses chats.}
  \end{itemize}

  \youglish{crazy}

\item \exEN{\href{http://www.wordreference.com/enfr/lazy}{lazy}} qui
  s'écrit phonétiquement
  \href{https://en.oxforddictionaries.com/definition/lazy}{\phonm{ˈleɪzi}}
  
  \begin{itemize}
  \item\exEN{My \href{https://youtu.be/5ZIR0PJ0eXI}{son} is smart but incredibly \href{https://youtu.be/3ev7GXzFTPg}{lazy}.}
  \item\exFR{Mon fils est intelligent mais extrêmement paresseux.}
  \end{itemize}

  \youglish{lazy}
  
\item \exEN{\href{http://www.wordreference.com/enfr/nose}{nose}} qui s'écrit phonétiquement \href{https://en.oxforddictionaries.com/definition/nose}{\phonm{nəʊz}}

  \begin{itemize}
  \item\exEN{The \href{https://youtu.be/nNA9ru2Ox5o}{tip} of my \href{https://youtu.be/1G-nn-b4TJA}{nose} is cold.}
  \item\exFR{Le bout de mon nez est froid.}
  \end{itemize}

  \youglish{nose}
  
\end{enumerate}

\newpage

\section{Le \son~\phon{ʃ}}\label{chap:ts}

\hypertarget{ch}{Ce \son} a pour nom technique\dyse{voiceless-postalveolar-fricative} :% #1: lien
                                % vers le blog
  %
  \begin{itemize}%
  \item \exEN{Voiceless postalveolar fricative \CW{https://en.wikipedia.org/wiki/Voiceless_postalveolar_fricative}.}% #2: sound name, #3: wiki EN
  \item \exFR{Consonne fricative palato-alvéolaire sourde \CW{https://fr.wikipedia.org/wiki/Consonne_fricative_palato-alv\%C3\%A9olaire_sourde}.}% #4: nom du son, #5: wiki FR sinon blog voir
                         % package ifthen pour gérer ça
  \end{itemize}

\begin{center}
  \begin{figure}[h]
    \centering
    \includegraphics[scale=.5]{../img/lodge/postalveolar-lodge}
    \caption{\exEN{Postalveolar Fricative}, source :~\cite{lodge}}
    \label{fig:postalveolar}
  \end{figure}
\end{center}

  \indicsound
  %
  \properukus{https://youtu.be/iDjAf2AhjRc}{https://youtu.be/RxaLKZPPEvY}% #6: UK YT, #7: US YT

\begin{enumerate}
\item \exEN{\href{http://www.wordreference.com/enfr/cash}{cash}} qui s'écrit phonétiquement \href{https://en.oxforddictionaries.com/definition/cash}{\phonm{kaʃ}}

  \begin{itemize}
  \item\exEN{The \href{https://youtu.be/4ahHWROn8M0}{cash} he received for his \href{https://youtu.be/VyXKzKe0QXk}{invention} is a windfall.}
  \item\exFR{L'argent qu'il a reçu pour son invention est une aubaine.}
  \end{itemize}

  \youglish{cash}
    
\item
  \exEN{\href{http://www.wordreference.com/enfr/national}{national}}
  qui s'écrit phonétiquement
  \href{https://en.oxforddictionaries.com/definition/national}{\phonm{ˈnaʃ(ə)n(ə)l}}
  
  \begin{itemize}
  \item\exEN{The country's beautiful \href{https://youtu.be/kWLqbk5vRqo}{landscapes} are a subject of
      \href{https://youtu.be/xZvzCOQ-TPA}{national} pride.}
  \item\exFR{Les beaux paysages du pays sont un objet de fierté nationale.}
  \end{itemize}

  \youglish{national}
    
\item \exEN{\href{http://www.wordreference.com/enfr/crash}{crash}} qui s'écrit phonétiquement \href{https://en.oxforddictionaries.com/definition/crash}{\phonm{kraʃ}}

  \begin{itemize}
  \item\exEN{All \href{https://youtu.be/7BWWWQzTpNU}{passengers} on the plane survived the \href{https://youtu.be/Jw81bRYUzVM}{crash}.}
  \item\exFR{Tous les passagers de l'avion ont survécu à l'accident.}
  \end{itemize}

  \youglish{crash}
    
\item \exEN{\href{http://www.wordreference.com/enfr/ship}{ship}} qui s'écrit phonétiquement \href{https://en.oxforddictionaries.com/definition/ship}{\phonm{ʃɪp}}

  \begin{itemize}
  \item\exEN{The \href{https://youtu.be/o6-s1mlQB5U}{company} mainly \href{https://youtu.be/LLkGsfOfgUw}{ships} parcels to Europe.}
  \item\exFR{L'entreprise expédie principalement des colis vers l'Europe.}
  \end{itemize}

  \youglish{ship}
    
\end{enumerate}

\newpage

\section{Le \son~\phon{ʒ}}\label{chap:dj}

\hypertarget{ez}{Ce \son} a pour nom technique\dyse{voiced-palato-alveolar-fricative} :% #1: lien
                                % vers le blog
  %
  \begin{itemize}%
  \item \exEN{Voiced palato-alveolar fricative \CW{https://en.wikipedia.org/wiki/Voiced_postalveolar_fricative}.}% #2: sound name, #3: wiki EN
  \item \exFR{Consonne fricative palato-alvéolaire voisée \CW{https://fr.wikipedia.org/wiki/Consonne_fricative_palato-alv\%C3\%A9olaire_vois\%C3\%A9e}.}% #4: nom du son, #5: wiki FR sinon blog voir
                         % package ifthen pour gérer ça
  \end{itemize}%
  %
  \indicsound%
  %
  \properukus{https://youtu.be/truPu_ReQ8Y}{https://youtu.be/RxaLKZPPEvY}% #6: UK YT, #7: US YT

\begin{enumerate}
\item \exEN{\href{http://www.wordreference.com/enfr/leisure}{leisure}} qui s'écrit phonétiquement \href{https://en.oxforddictionaries.com/definition/leisure}{\phonm{ˈlɛʒə}}

  \begin{itemize}
  \item\exEN{\href{https://youtu.be/IHkXl7L-lFA}{Everyone} needs moments of \href{https://youtu.be/VSRFE7E4qWI}{leisure} to relax.}
  \item\exFR{Tout le monde a besoin de moments de loisir pour se détendre.}
  \end{itemize}

  \youglish{leisure}
    
\item \exEN{\href{http://www.wordreference.com/enfr/measure}{measure}} qui s'écrit phonétiquement \href{https://en.oxforddictionaries.com/definition/measure}{\phonm{ˈmɛʒə}}

  \begin{itemize}
  \item\exEN{This application \href{https://youtu.be/bN60fb9fzKg}{measures} the speed of the \href{https://youtu.be/9hIQjrMHTv4}{Internet}
      connection.}
  \item\exFR{Cette application calcule la vitesse de la connexion Internet.}
  \end{itemize}

  \youglish{measures}
  
\item \exEN{\href{http://www.wordreference.com/enfr/pleasure}{pleasure}} qui s'écrit phonétiquement \href{https://en.oxforddictionaries.com/definition/pleasure}{\phonm{ˈplɛʒə}}

  \begin{itemize}
  \item\exEN{I \href{https://youtu.be/wv-mD-QHRMQ}{read} your book with great \href{https://youtu.be/Q4-VK5uqY34}{pleasure}.}
  \item\exFR{J'ai lu votre livre avec grand plaisir.}
  \end{itemize}

  \youglish{pleasure}
  
\item \exEN{\href{http://www.wordreference.com/enfr/vision}{vision}} qui s'écrit phonétiquement \href{https://en.oxforddictionaries.com/definition/vision}{\phonm{ˈvɪʒ(ə)n}}

  \begin{itemize}
  \item\exEN{The teacher's \href{https://youtu.be/lk7lIhAmwHI}{vision} was getting \href{https://youtu.be/rln_kZbYaWc}{fuzzy} so he put his
      glasses on.}
  \item\exFR{Comme sa vision devenait floue, le professeur a mis ses lunettes.}
  \end{itemize}

  \youglish{vision}

\end{enumerate}

\newpage

\section{Le \son~\phon{h}}\label{sec:h}

\hypertarget{h}{Ce \son} a pour nom technique\dyse{voiceless-glottal-fricative} :% #1: lien
                                % vers le blog
  %
  \begin{itemize}%
  \item \exEN{Voiceless glottal fricative \CW{https://en.wikipedia.org/wiki/Voiceless_glottal_fricative}.}% #2: sound name, #3: wiki EN
  \item \exFR{Consonne fricative glottale sourde \CW{https://fr.wikipedia.org/wiki/Consonne_fricative_glottale_sourde}.}% #4: nom du son, #5: wiki FR sinon blog voir
                         % package ifthen pour gérer ça
  \end{itemize}%
  %
  \indicsound%
  %
  \properukus{https://youtu.be/r0GW2q9gibY}{https://youtu.be/uOG-4ZjR7ic}% #6: UK YT, #7: US YT
  
\begin{enumerate}
\item \exEN{\href{http://www.wordreference.com/enfr/ahead}{ahead}} qui s'écrit phonétiquement \href{https://en.oxforddictionaries.com/definition/ahead}{\phonm{əˈhɛd}} 

  \begin{itemize}
  \item\exEN{It has been \href{https://youtu.be/kf9xGFd5dTU}{major}, important and time-consuming work,
      because we in actual fact have demanding and important tasks
      \href{https://youtu.be/1rLpIOzKaBA}{ahead} of us.}
  \item\exFR{C'est un travail énorme, important et très long, dans
      la mesure où les missions qui nous attendent sont importantes
      et exigeantes.}
  \end{itemize}

  \youglish{ahead}
  
\item \exEN{\href{http://www.wordreference.com/enfr/hello}{hello}} qui s'écrit phonétiquement \href{https://en.oxforddictionaries.com/definition/hello}{\phonm{hɛˈləʊ}}

  \begin{itemize}
  \item\exEN{\href{https://youtu.be/62XB9IbMnxQ}{Hello} \href{https://en.wikipedia.org/wiki/\%2522Hello,\_World!\%2522\_program}{world}!}
  \item\exFR{Bonjour le monde !}
  \end{itemize}

  \youglish{hello}
  
\item \exEN{\href{http://www.wordreference.com/enfr/high}{high}} qui s'écrit phonétiquement \href{https://en.oxforddictionaries.com/definition/high}{\phonm{hʌɪ}}

  \begin{itemize}
  \item\exEN{\href{https://youtu.be/F7lj4LknWO8}{High} walls surrounded the \href{https://youtu.be/hclQLklBHNs}{castle}.}
  \item\exFR{De hauts murs entouraient le château.}
  \end{itemize}

  \youglish{high}

\item \exEN{\href{http://www.wordreference.com/enfr/whole}{whole}} qui s'écrit phonétiquement \href{https://en.oxforddictionaries.com/definition/whole}{\phonm{həʊl}}

  \begin{itemize}
  \item\exEN{The \href{https://youtu.be/5eTCZ9L834s}{environment} concerns society as a \href{https://youtu.be/bJnw1ma6Xks}{whole}.}
  \item\exFR{L'environnement concerne l'ensemble de la société.}
  \end{itemize}

  \youglish{whole}

\end{enumerate}

\newpage
\minitoc
\newpage

