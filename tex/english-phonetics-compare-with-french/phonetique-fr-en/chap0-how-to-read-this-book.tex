\chapter{Comment lire ce livre ?}\label{chap:howto}
\newpage
\minitoc
\newpage

\section{Au commencement était le\dots lien}\label{sec:link}

Ceci est un \href{https://youtu.be/K88qlGcd7Ek}{lien}
\hypertarget{url}{URL}\footnote{Voir Wikipédia :
  \url{https://fr.wikipedia.org/wiki/Uniform_Resource_Locator} si vous
  souhaitez en savoir plus tout de suite, sinon voir
  page~\pageref{sec:side} (et ça c'est un lien interne).} qui va vous diriger vers une vidéo qui
explique les bonnes raisons qui justifient l'acquisition de ce
livre. Vous pouvez également la retrouver sur la \href{http://doyouspeakenglish.fr/contact/}{page de contact} de mon
blog\footnote{La vidéo se trouve en bas de la page de contact.}. Normalement vous
devriez avoir compris comment les
\href{https://fr.wikipedia.org/wiki/Uniform_Resource_Locator}{liens
  URL} sont représentés dans ce document. Les \hyperlink{linkin}{liens internes} \hypertarget{retour}{sont}\label{retour}
représentés de la même manière que ça\footnote{Ceci est une note de
  bas de page et ceci est un \hyperlink{linkin}{lien interne}.}. Oui les notes de bas de page\footnote{Très fréquente
  et très utile dans ce livre.} sont des liens internes au
document. Il y aura également d'autres liens cliquables comme par
exemple les tables des matières\footnote{Oui il y en a plusieurs.}.

En effet, afin de favoriser la navigation dans le document j'ai mis à la
fois une table des matières générale en début d'ouvrage puis au début
et à la fin de chaque chapitre vous trouverez un sommaire de chapitre
qui vous permettra de sauter ou de revenir sur une section de
chapitre que vous souhaitez cibler en particulier. Ce document est
très similaire à un site web de par sa structure de réseau (de
nombreux éléments divers sont liés les uns aux autres). Il faut savoir
que le langage \href{https://fr.wikipedia.org/wiki/TeX}{\TeX} est un langage de
marquage de texte ou \underline{langage balisé} au même titre que le
langage \href{https://fr.wikipedia.org/wiki/Langage_de_balisage\#Langage_HTML}{HTML}
qui sert à écrire toutes les pages  \href{https://fr.wikipedia.org/wiki/World_Wide_Web}{Web}
que vous consultez sur Internet. Bien que la connexion ne soit pas
directe, il est notable que le langage \TeX{} est antérieur au langage
HTML et qu'il était initialement utilisé majoritairement par les
milieux scientifiques universitaires\dots les mêmes milieux qui ont
fait naître Internet et le langage HTML. En ce sens Donald
Knuth\footnote{L'\href{https://fr.wikipedia.org/wiki/Donald_Knuth}{inventeur} du langage \TeX{}.} était un précurseur de
Tim Berners Lee\footnote{L'\href{https://fr.wikipedia.org/wiki/Tim_Berners-Lee}{inventeur} du Web.}.

Ne serait-il pas plus simple de décrire ces liens en parlant de leurs
couleurs et de leur apparence générale ?

Probablement, mais il se trouve que la magie de \TeX{} et de ses
descendants comme par exemple
\href{https://fr.wikipedia.org/wiki/LaTeX}{\LaTeX}\footnote{Prononcer
  << la
  \href{https://fr.wikipedia.org/wiki/Th\%C3\%A8que_(sport)}{thèque}
  >> comme le sport.} ou encore 
\href{https://en.wikipedia.org/wiki/XeTeX}{\XeLaTeX} (que j'ai utilisé
pour composer ce document), consiste à
permettre au concepteur du document de séparer sens et forme. 

Cela signifie que j'ai précisé à l'ordinateur une commande spéciale
pour lui indiquer que je veux que tel ou tel mot ou groupe de mots
soit un \hyperlink{url}{lien} URL\footnote{Voir section~\ref{sec:link}
  page~\pageref{sec:link}.} par exemple ou un lien interne. Ce qui me
permet de cibler des catégories de mots récurrents (au sens logique)
comme par exemple tous les liens internes, tous les liens
externes\footnote{Vous en verrez d'autres applications dans la
  section~\ref{sec:side} page~\pageref{sec:side}}\dots 

Ensuite au début du document, dans la rubrique des paramétrages, je
peux choisir à loisir les différentes formes que prendront ces
commandes. Par conséquent, toute l'apparence du document peut être
modifiée en << raffale >> si j'ose dire, en une seule
fois\footnote{Grâce à ce qu'on appelle des \href{http://www.tuteurs.ens.fr/logiciels/latex/macros.html}{macros}.}. Et au
moment où j'écris ces lignes je n'ai pas encore arrêté mon choix quant
à la forme que je souhaite qu'elle prenne\footnote{Je parle de la
  forme de l'apparence du document donc c'est bien un féminin
  singulier.}. Il vient que la preuve par l'exemple m'a semblé être la
plus façon la plus pertinente pour faire ma démonstration et vous
expliquer le fonctionnement des liens dans ce livre.

\newpage

\section{De quel côté de la Manche ? de l'Atlantique ?}\label{sec:side}

Et j'aurais pu poursuivre avec l'océan Pacifique et/ou l'océan Indien
pour rester sur les frontières maritimes qui sont nombreuses entre la
francophonie et le monde anglophone. Oui dans cette section nous
allons parler du codage pour faire la distinction entre les mots en
\exEN{English} et ceux en \exFR{Français}.

En fait, cette distinction de couleur sera faite lorsqu'il s'agira de
traduire\footnote{Néanmoins, l'auteur de ce document s'est parfois 
  laissé aller à abuser un petit peu de cette récréation artistique.}.

Par exemple :
\begin{itemize}
\item \exEN{English: This is an English sentence\footnote{\exEN{So British!}}.}
\item \exFR{Français : Ceci est une phrase française\footnote{\exFR{Oui elle
      a bien ses papiers, par contre j'ai pas vérifié pour ses grands-parents.}}}.  
\end{itemize}

Je pense que vous avez saisi le truc. En guise d'illustration
complémentaire voici une petite définition utile\footnote{Dont j'avais
annoncé la définition prochaine ici page~\pageref{sec:link} (encore un
lien interne, j'espère que vous suivez).}, URL : \exEN{Uniform
  Resource Locator} qui se traduit littéralement par
\exFR{Localisateur Uniforme de Ressource} tout de suite ça a plus
d'allure\footnote{Ben oui, ça fait LUR\dots toute ressemblance avec de
  l'humour serait fortuite.}.

La couleur du cadre qui entoure les éléments de la table des matières
ou encore celle des numéros de notes de bas de page est celle réservée
aux \hypertarget{linkin}{liens internes}\footnote{D'ailleurs si vous
  avez atteri sur cette page après avoir cliqué sur le
  \hyperlink{retour}{lien interne} page~\pageref{retour}, vous devriez
savoir comment faire pour y revenir.}. 

\newpage

\section{Et les sons dans tout ça ?}\label{sec:phonetics}

Le but de ce livre étant de traiter de la phonétique anglaise, les
sons seront omniprésent. D'ailleurs même si je vous ai expliqué dans
la section~\ref{sec:side} page~\pageref{sec:side} que j'avais établi un code
couleur pour faire la distinction entre les mots, expressions ou
phrases en \exEN{English} et leurs traductions en \exFR{Français}.

Et bien il se trouve que j'ai également établi un code couleur pour
les \textcolor{teal}{sons}. Alors il faut bien distinguer, par exemple
le\son[~]{e} comme dans le mot français \exFR{beauté} de la
description phonétique, par exemple, du verbe anglais \exEN{do}, qui
s'écrit phonétiquement \exPH{duː}. Techniquement parlant, la
description phonétique d'un mot est un assemblage de
\textcolor{teal}{sons}\footnote{Un \textcolor{teal}{son} est appelé un
<<~phone~>> alors qu'un assemblage de \textcolor{teal}{sons} est un <<~
phonème~>> (je suis sûr que ce soir vous dormirez plus instruit).}.

Attention également à ne pas apprendre par c{\oe}ur les symboles de
l'alphabet phonétique international. La phonétique telle que je
l'introduis ici n'est qu'un guide, une aide à la <<~bonne~>>
prononciation.

En effet, comme en français, en anglais il y a des accents selon qu'on
est à Londres, Manchester, Glasgow, Dublin, NYC,\dots Par conséquent
la transcription fidèle et rigoureuse des \textcolor{teal}{sons}
impliquerait beaucoup trop de combinaison et serait quasiment
impossible à mémoriser.

Je dois vous faire une confession, je ne suis pas linguiste. Et je
suppose que vous non plu sinon vous ne seriez pas en train de lire ce
livre.

C'est précisément pour cette raison que ce livre est abondamment
fourni en liens externes afin que vous puissiez \underline{entendre}
les \textcolor{teal}{sons} plutôt que de chercher à mémoriser leur
représentation phonétique.

\newpage

\section{À qui s'adresse ce livre ?}\label{sec:for-who}

Habituellement, les bons auteurs\footnote{Ils sont un peu comme les
  \href{https://www.amazon.fr/gp/product/B0103QW256/ref=as_li_tl?ie=UTF8\&camp=1642\&creative=6746\&creativeASIN=B0103QW256\&linkCode=as2\&tag=wwwbecomefree-21\&linkId=0a96fb10b5f781c84d66ef5b92ea65b6}{bons
    chasseurs}.}, définissent clairement leur cible de marché avant
d'écrire leur livre\dots et ben moi pas. Je suis probablement pas
(encore) un bon auteur. Initialement j'ai écrit ce livre parce que je
ne trouvais pas ce que je cherchais. Il est très fastidieux de trouver
des ressources fiables concernant ce sujet et surtout en français. De
plus l'harmonisation des symboles phonétiques n'est pas encore
totalement réalisée comme vous pourrez le voir par exemple dans les
sections :

\tdm{c}{sone}{sonae}{sonenv}{sonenvlong}{sonalong}{enenvomegaenv}{omegaenvenenv}{eeteenv}
  

Alors que vous soyez débutant, étudiant ou éventuellement professeur,
dans tous les cas vous gagnerez un temps
considérable\footnote{Croyez-moi, ça prend \underline{beaucoup} de
temps pour parcourir le web et sélectionner les bonnes informations.}
et en plus vous disposerez d'une source intarissable d'exemples
vidéos. Donc je pense en toute modestie que ce livre sera un outil
efficace pour quiconque souhaite améliorer ses compétences orales en anglais.

\newpage

\section{À qui ne s'adresse pas ce livre ?}\label{sec:for-not-who}

Pour être totalement honnête je n'ai envie d'exclure
personne. Néanmoins, il paraît qu'on ne peut pas plaire à tout le
monde. Donc il y aura sans doute des gens qui n'aimeront pas ce
livre. Sachez que je n'ai rien contre les gens qui n'aiment pas mon
travail. Comme dit dans la section~\ref{sec:for-who}
page~\pageref{sec:for-who}, je pense sincèrement que ce livre sera
utile même aux professeurs parce qu'il est une agrégation de beaucoup
d'outils efficaces réalisés par des professionnels essentiellement
anglophones.

Si pour une quelconque raison ce livre ne vous plaisez pas alors je
vous serais gré de m'en faire savoir la ou les raison(s) en m'écrivant un mail
via le formulaire de contact sur mon \href{http://doyouspeakenglish.fr/contact/}{blog}\footnote{Dont l'adresse est
  la suivante : \url{http://doyouspeakenglish.fr/contact/}}. De cette
façon je pourrais savoir comment améliorer l'ouvrage. Naturellement si
vous reperiez une ou des erreurs merci également de me les signaler
via la même procédure.

Dans tous les cas je vous souhaite un bon courage car l'apprentissage
de l'anglais oral est un véritable défi et j'espère que les outils que
je mets à votre disposition vous aideront à le relever de la plus
belle des manières.

\begin{center}
  \exEN{Good luck!}
\end{center}

\newpage
\minitoc

