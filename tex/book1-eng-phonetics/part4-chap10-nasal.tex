\chapter{Nasal (air libéré par le nez)}\label{chap:nasal}

\speech{3}{consonnes \exFR{nasales\CW{https://fr.wikipedia.org/wiki/Consonne_nasale}} (\exEN{nasal\CW{https://en.wikipedia.org/wiki/Nasal_consonant}})}

\begin{center}
  \begin{figure}[h]
    \centering
    \includegraphics{../img/nasales-wikipedia}
    \caption{Consonnes nasales, source : Wikipédia}
    \label{fig:cons-nasal}
  \end{figure}
\end{center}

\newpage
\minitoc
\newpage

\section{Le \son~\phon{m}}\label{chap:m}

Ce \son a pour nom
technique\dyse{bilabial-nasal} :

\begin{itemize}
\item \exEN{Bilabial Nasal\CW{https://en.wikipedia.org/wiki/Bilabial_nasal}.}
\item \exFR{Consonne nasale bilabiale voisée\CW{https://fr.wikipedia.org/wiki/Consonne_nasale_bilabiale_vois\%C3\%A9e}.}
\end{itemize}

\begin{center}
  \begin{figure}[h]
    \centering
    \includegraphics[scale=.5]{../img/cpp/m-cpp}
    \caption{\exEN{Bilabial Nasal m}, source :~\cite{collins}}
    \label{fig:f-v}
  \end{figure}
\end{center}

\indicsound

\properukus{https://youtu.be/_o7H0eoj2SI}{https://youtu.be/tkiN8BsBEfA}

\begin{enumerate}
  
\item \exEN{\href{http://www.wordreference.com/enfr/calm}{calm}} qui s'écrit phonétiquement \href{https://en.oxforddictionaries.com/definition/calm}{\phonm{kɑːm}}

  \begin{itemize}
  \item\exEN{He kept \href{https://youtu.be/1tXBl3Q5Ibc}{calm} in order not to start a \href{https://youtu.be/e1il5yarxLU}{scrap}.}
  \item\exFR{Il est resté calme afin de ne pas déclencher une bagarre.}
  \end{itemize}

  \youglish{calm}


\item \exEN{\href{http://www.wordreference.com/enfr/hammer}{hammer}} qui s'écrit phonétiquement \href{https://en.oxforddictionaries.com/definition/hammer}{\phonm{ˈhamə}}

  \begin{itemize}
  \item\exEN{I need a \href{https://youtu.be/ReRiyfwW6-g}{screwdriver} and a \href{https://youtu.be/t5l2AUlD8Sk}{hammer} to fix the shelf.}
  \item\exFR{J'ai besoin d'un tournevis et d'un marteau pour
      réparer l'étagère.}
  \end{itemize}

  \youglish{hammer}

\item \exEN{\href{http://www.wordreference.com/enfr/mad}{mad}} qui s'écrit phonétiquement \href{https://en.oxforddictionaries.com/definition/mad}{\phonm{mad}}

  \begin{itemize}
  \item\exEN{The \href{https://youtu.be/0T3tg6Xwt7A}{scientist} must be \href{https://youtu.be/Oa-ae6\_okmg}{mad} to try such experiments.}
  \item\exFR{Le scientifique doit être dingue pour tenter de telles expériences.}
  \end{itemize}

  \youglish{mad}

\item \exEN{\href{http://www.wordreference.com/enfr/sum}{sum}} qui s'écrit phonétiquement \href{https://en.oxforddictionaries.com/definition/sum}{\phonm{sʌm}}
  
  \begin{itemize}
  \item\exEN{The \href{https://youtu.be/ymUTWzsoiIg}{sum} is indicated on the \href{https://youtu.be/MAwTnbtC3w4}{invoice}.}
  \item\exFR{Le total est indiqué sur la facture.}
  \end{itemize}

  \youglish{sum}

\end{enumerate}

\newpage

\section{Le \son~\phon{n}}\label{chap:n}

Ce \son a pour nom
technique\dyse{alveolar-nasal} :

\begin{itemize}
\item \exEN{Alveolar Nasal\CW{https://en.wikipedia.org/wiki/Dental,_alveolar_and_postalveolar_nasals}.}
\item \exFR{Consonne nasale alvéolaire voisée\CW{https://fr.wikipedia.org/wiki/Consonne_nasale_alv\%C3\%A9olaire_vois\%C3\%A9e}.}
\end{itemize}

\begin{center}
  \begin{figure}[h]
    \centering
    \includegraphics[scale=.5]{../img/cpp/n-cpp}
    \caption{\exEN{Alveolar Nasal n}, source :~\cite{collins}}
    \label{fig:f-v}
  \end{figure}
\end{center}

\indicsound

\properukus{https://youtu.be/vDrdgvk-G30}{https://youtu.be/1eyr7O4TFmI}

\begin{enumerate}
\item \exEN{\href{http://www.wordreference.com/enfr/know}{know}} qui s'écrit phonétiquement \href{https://en.oxforddictionaries.com/definition/know}{\phonm{nəʊ}}

  \begin{itemize}
  \item\exEN{I \href{https://youtu.be/j-CwwdwQV54}{know} a good restaurant \href{https://youtu.be/M2MofcPjonU}{nearby}.}
  \item\exFR{Je connais un bon restaurant à proximité.}
  \end{itemize}

  \youglish{know}

\item \exEN{\href{http://www.wordreference.com/enfr/nobody}{nobody}} qui s'écrit phonétiquement \href{https://en.oxforddictionaries.com/definition/nobody}{\phonm{ˈnəʊbədi}}

  \begin{itemize}
  \item\exEN{At the time \href{https://youtu.be/icE0AqVSnzo}{nobody} could have known that it would \href{https://youtu.be/bHp3ctfCaAU}{take}
      six months.}
  \item\exFR{Personne ne pouvait à ce moment savoir que ceci prendrait six mois.}
  \end{itemize}

  \youglish{nobody}

\item \exEN{\href{http://www.wordreference.com/enfr/funny}{funny}} qui s'écrit phonétiquement \href{https://en.oxforddictionaries.com/definition/funny}{\phonm{ˈfʌni}}
  
  \begin{itemize}
  \item\exEN{All \href{https://youtu.be/78cIuEMWaI4}{visitors}, especially the children, found the clown
      \href{https://youtu.be/CNXOu7gPEXM}{funny}.}
  \item\exFR{Tous les visiteurs, surtout les enfants, ont trouvé
      le clown amusant.}
  \end{itemize}

  \youglish{funny}

\item \exEN{\href{http://www.wordreference.com/enfr/turn}{turn}} qui s'écrit phonétiquement \href{https://en.oxforddictionaries.com/definition/turn}{\phonm{təːn}}

  \begin{itemize}
  \item\exEN{The \href{https://youtu.be/pjlPgJ1wBdM}{negotiations} have taken a decisive \href{https://youtu.be/z4g45vTgczE}{turn} today.}
  \item\exFR{Les négociations ont pris un tournant décisif
      aujourd'hui.}
  \end{itemize}

  \youglish{turn}

\end{enumerate}

\newpage

\section{Le \son~\phon{ŋ}}\label{chap:ing}

\hypertarget{ing}{Ce \son} a pour nom technique\dyse{velar-nasal} :

\begin{itemize}
\item \exEN{Velar Nasal\CW{https://en.wikipedia.org/wiki/Velar_nasal}.}
\item \exFR{Consonne nasale vélaire voisée\CW{https://fr.wikipedia.org/wiki/Consonne_nasale_v\%C3\%A9laire_vois\%C3\%A9e}.}
\end{itemize}

\begin{center}
  \begin{figure}[h]
    \centering
    \includegraphics[scale=.5]{../img/cpp/ing-cpp}
    \caption{\exEN{Velar Nasal ŋ}, source :~\cite{collins}}
    \label{fig:f-v}
  \end{figure}
\end{center}

\indicsound

\properukus{https://youtu.be/-RfiBn9qPlM}{https://youtu.be/5xVq8T88oJw}


\begin{enumerate}
\item \exEN{\href{http://www.wordreference.com/enfr/anger}{anger}} qui s'écrit phonétiquement \href{https://en.oxforddictionaries.com/definition/anger}{\phonm{ˈaŋɡə}}

  \begin{itemize}
  \item\exEN{He suddenly \href{https://youtu.be/8jsC9KVI0A4}{unleashed} his \href{https://youtu.be/lw64e7JVRj0}{anger}.}
  \item\exFR{Il a soudainement déchaîné sa colère.}
  \end{itemize}

  \youglish{anger}

\item \exEN{\href{http://www.wordreference.com/enfr/bang}{bang}} qui s'écrit phonétiquement \href{https://en.oxforddictionaries.com/definition/bang}{\phonm{baŋ}}. Exemple d'utilisation de ce
  mot :
  
  \begin{itemize}
  \item\exEN{A loud \href{https://youtu.be/N-AgYXz2n9Y}{bang} woke me up in the \href{https://youtu.be/XAa11jc-LU0}{middle} of the night.}
  \item\exFR{Un grand fracas m'a réveillé en pleine nuit.}
  \end{itemize}

  \youglish{bang}

\item \exEN{\href{http://www.wordreference.com/enfr/king}{king}} qui s'écrit phonétiquement \href{https://en.oxforddictionaries.com/definition/king}{\phonm{kɪŋ}}

  \begin{itemize}
  \item\exEN{The \href{https://youtu.be/MRgFeZa\_I48}{king} and \href{https://youtu.be/wiDCwqpupj8}{queen} live in a magnificent palace.}
  \item\exFR{Le roi et la reine habitent dans un palais magnifique.}
  \end{itemize}

  \youglish{king}

\item \exEN{\href{http://www.wordreference.com/enfr/thanks}{thanks}} qui s'écrit phonétiquement \href{https://en.oxforddictionaries.com/definition/thanks}{\phonm{θaŋks}}
  
  \begin{itemize}
  \item\exEN{We solved the \href{https://youtu.be/nQpCuYS41Oo}{problem} \href{https://youtu.be/hQiipuDbbxw}{thanks} to a concerted effort.}
  \item\exFR{Nous avons résolu le problème grâce à un effort
      concerté.}
  \end{itemize}

  \youglish{thanks}

\end{enumerate}

\newpage
\minitoc
\newpage

