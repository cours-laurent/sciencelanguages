\chapter{Ce que vous avez réussi à parcourir}\label{chap:success}

Bravo à vous, vous avez réussi à parcourir mon livre de bout en
bout. Voilà, nous sommes arrivés à la fin de ce voyage. Si je prends
la métaphore du voyage c'est à dessein parce que de la même manière
que l'on découvre une nouvelle destination pour la première fois il en
va de même pour un nouveau domaine tel que la \gls{phon}. Lors du
premier voyage on est souvent impressionné, surpris mais à moins d'y
passer beaucoup de temps en général on ne vit qu'une expérience
superficielle. Selon le degré d'implication on peut déjà pressentir le
désir d'approfondir. Et c'est précisément mon but avec ce premier
ouvrage sur le sujet.

J'espère que cette première lecture vous aura donné envie de le relire
encore et encore. En effet, mon but est de vous avoir fourni un guide
qui vous servira autant de fois que nécessaire pour parfaire votre
utilisation et votre compréhension des sons de la langue anglaise.

Après vous avoir présenté très brièvement les outils qui m'ont permis
d'écrire cet ouvrage, et pour lesquels je propose d'ailleurs des
formations (avis aux \href{http://doyouspeakenglish.fr/contact/}{amateurs}\footnote{Si vous êtes intéressés vous
  pouvez me contacter via le formulaire de contact de mon blog sur
  l'anglais \url{http://doyouspeakenglish.fr/contact/}.}), j'ai essayé de partager avec vous mon
goût prononcé pour découvrir la \gls{linguistic}.

En effet, j'ai voulu partager avec vous ma joie de découvrir ce
merveilleux domaine qui embrasse toutes les disciplines possibles
puisqu'il s'agit de l'étude de l'outil que nous utilisons chaque jour
pour communiquer, et ce, quelle que soit notre activité.

Les plus pragmatiques et/ou impatients d'entre vous ont certainement
sauté directement sur la partie pratique qui consiste à étudier
concrètement comment écouter et essayer de formuler les sons
correctement. Avec les nombreux exemples que je vous ai fourni ainsi
que les liens innombrables je pense que vous pourrez utiliser ce guide
aussi longtemps qu'il sera possible d'héberger des vidéos et audios en
ligne\dots{} autant dire pendant très longtemps !

