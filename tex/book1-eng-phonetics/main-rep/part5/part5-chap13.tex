\chapter{Et pour quelques conseils de plus}\label{chap:conseils}

\section{En quoi ce livre est différent ?}\label{sec:diff}

Ce livre n'est pas un livre traditionnel ! Loin s'en faut, ce livre
tire toute sa force de notre capacité à être hyperconnecté à
l'Internet ou devrais-je dire aux internets. En effet, la force
principale de ce livre est de rassembler en un seul endroit de
nombreux outils, de nombreux liens qui vous seront très utiles tout au
long de votre apprentissage de l'anglais.

\section{Mise en garde contre une lecture linéaire}\label{sec:lin}

Ne vous jetez pas à corps perdu dans l'ouvrage mais essayez plutôt
d'organiser vos apprentissage. En effet, je pense qu'une étude
quotidienne par session de 10 à 20-25 minutes peut véritablement,
considérablement, augmenter votre niveau. Le mot \underline{quotidien}
revêt toute son importance ici. Il est vraiment \textbf{fondamental}
d'être régulier, c'est la clé de la réussite. Je vous rappelle
également que vous pouvez vous inspirez des 30 astuces que j'ai
partagé dans cette \href{https://www.youtube.com/playlist?list=PLfKvL-VUSKAnf4oZzkI3q24X4FJrGzcGr}{playlist}\footnote{\url{https://www.youtube.com/playlist?list=PLfKvL-VUSKAnf4oZzkI3q24X4FJrGzcGr}}.

N'ouvrez pas trop d'onglets à la fois, concentrez-vous sur quelques
sons à la fois. Je pense qu'au delà de 2 ou 3 par jour ça commence à
faire vraiment beaucoup. Bien sûr, vous êtes libre d'adapter la dose à votre
ressenti. Mais sachant que chaque son est présenté via 4 exemples avec
en plus de nombreux liens externes je pense qu'une étude d'un son par
jour me paraît déjà beaucoup. À vrai dire je pense qu'un son par
semaine est le meilleur équilibre.

\section{Plan d'actions}\label{sec:plan}

Voici un plan d'actions que je proposerais pour celles et ceux qui
veulent un guide précis pour tirer le meilleur parti de cet ouvrage
dans votre apprentissage. Voilà comment organiser votre semaine idéale :

\begin{enumerate}
\item Les jours 1 à 4 (de lundi à jeudi inclus) vous allez analyser
  chaque mot-exemple qui illustre le \son. Pour chaque mot construisez
  vos propres phrases. \'Ecrivez-les d'abord, lisez-les puis essayez
  d'improviser en les utilisant. Si les liens fournis vous dirigent
  vers des chansons essayez de fredonner l'air. Et si les musiques ne
  vous plaisent pas cherchez-en avec les mots proposés. Pour chaque
  jour étudier tous les liens qui se trouve dans la phrase exemple en
  question. Appropriez-vous chaque mot en essayant de construire vos
  propres exemples. 
\item  Ensuite les jours 5 à 7 vous allez vous concentrer sur les variations
 possibles entre l'accent britannique, américain et australien grâce
 aux liens fournis par
 \href{https://youtu.be/IBsEkslVxoQ}{YouGlish}. Pour les accents
 britanniques et américains notez et utilisez les mots donnés en
 exemples par les professeurs dans les vidéos. Ils enrichiront votre
 vocabulaire et votre base de groupes de mots associés au \son en question.
\end{enumerate}

\section{Conseils généraux}\label{sec:gen}

Il est important de ne pas aller plus vite que la musique mais il est
aussi important de ne pas se focaliser sur un seul accent. Le monde
se rétrécit chaque jour et à moins que vous soyiez amené à vous fixer
dans une région en particulier, il vaut mieux prendre l'habitude
d'éduquer son oreille aux variations. En revanche pour votre
prononciation vous pouvez effectivement vous concentrer sur un accent
en particulier. A priori je vous conseillerais celui qui vous semble
le plus simple.  

Prenez également le temps de consulter les nombreuses annexes que j'ai
richement fourni. Je pense en particulier aux outils concrets pour
vous permettre de transcrire instantannément les mots, groupes de
mots, phrases et même petits paragraphes en symboles \glspl{phon}. Mais
n'oubliez pas que le but du jeu n'est pas d'apprendre les symboles par
c{\oe}ur, le but est d'être capable de percevoir les différences à
l'oreille et de produire les sons les plus compréhensibles possibles
pour vos interlocuteurs. Savoir écrire tous les symboles sans savoir
entendre ou produire les sons est inutile. Bien sûr on peut prendre du
plaisir à jouer avec ce nouvel ensemble de symboles mais cela doit
venir après la maîtrise de l'écoute et de la parole. Pratiquez,
pratiquez, et encore pratiquez\footnote{Les anglo-saxons disent
  \exEN{Practice makes perfect} à juste titre !}. 

