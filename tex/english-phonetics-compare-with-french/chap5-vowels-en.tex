\chapter{\exEN{Vowel sounds}}
\label{chap:vow}
Un son de voyelle est fait en façonnant l'air comme il quitte la
bouche. Nous utilisons les articulateurs pour façonner l'air - les
lèvres, la mâchoire, la langue. L'anglais britannique utilise 12 positions de la bouche.
\section{\textenglish{Front Vowels} (langue vers l'avant)}
\label{sec:orge433061}
Il y en a 4 et pour chaque son, je vous proposerai au moins 4
exemples. La structure sera toujours la même, le son écrit selon la
norme de l'API ou IPA en anglais (à partir de maintenant on utilisera
la terminologie anglaise), puis les exemples pour illustrer.
\subsection{Le son \href{https://youtu.be/EuZa9-QbhG8}{[iː]} comme dans les mots anglais}
\label{sec:org62768e9}
\begin{enumerate}
\item \href{http://www.wordreference.com/enfr/need}{need} qui s'écrit en phonétique \href{https://en.oxforddictionaries.com/definition/need}{\emph{niːd}}. Considérez la
phrase suivante :
\begin{description}
\item[{english}] \textenglish{I \href{https://youtu.be/p0quLJutRC8}{need} to work every day if I want to improve my level.}
\item[{français}] Je dois travailler tous les jours si je veux
améliorer mon niveau.
\end{description}
\item \href{http://www.wordreference.com/enfr/tea}{tea} qui s'écrit en phonétique \href{https://en.oxforddictionaries.com/definition/tea}{\emph{tiː}}. Voici un exemple simple dans
lequel ce mot apparaît :
\begin{description}
\item[{english}] \textenglish{Every morning we use to drink \href{https://youtu.be/Euh8dY4EU9o}{tea}.}
\item[{français}] Tous les matins on a l'habitude de boire du thé.
\end{description}
\item \href{http://www.wordreference.com/enfr/believe}{believe} qui s'écrit phonétiquement \href{https://en.oxforddictionaries.com/definition/believe}{\emph{bɪˈliːv}}. En voici un exemple
célèbre :
\begin{description}
\item[{english}] \href{https://youtu.be/GIQn8pab8Vc}{\textenglish{I believe I can fly.}}
\item[{français}] Je crois que je peux voler.
\end{description}
\item \href{http://www.wordreference.com/enfr/see}{see} qui s'écrit phonétiquement \href{https://en.oxforddictionaries.com/definition/see}{\emph{siː}}. Exemple :
\begin{description}
\item[{english}] \textenglish{What You \href{https://youtu.be/Dpf2yHjBVYM}{See} Is What You Get (\href{https://fr.wikipedia.org/wiki/What\_you\_see\_is\_what\_you\_get}{WYSIWYG})}
\item[{français}] Ce que vous voyez est ce que vous obtenez
\end{description}
\end{enumerate}

\subsection{Le son [ɪ] comme dans les mots anglais}
\label{sec:orgf03eb34}
\begin{enumerate}
\item \href{http://www.wordreference.com/enfr/england}{England} qui s'écrit en phonétique \href{https://en.oxforddictionaries.com/definition/england}{\emph{ˈɪŋɡlənd}}. Exemple :
\begin{description}
\item[{english}] \textenglish{Last summer I went to \href{https://youtu.be/QUPBesOdax8}{England}.}
\item[{française}] L'été dernier je suis allé en Angleterre.
\end{description}
\item \href{http://www.wordreference.com/enfr/thin}{thin} qui s'écrit en phonétique \href{https://en.oxforddictionaries.com/definition/thin}{\emph{θɪn}}. Exemple :
\begin{description}
\item[{english}] \textenglish{Usually female top model are \href{https://youtu.be/LekA62H17bo}{thin}.}
\item[{française}] Habituellement les mannequins féminins sont minces.
\end{description}
\item \href{http://www.wordreference.com/enfr/big}{big} qui s'écrit phonétiquement \href{https://en.oxforddictionaries.com/definition/big}{\emph{bɪɡ}}. Exemple :
\begin{description}
\item[{english}] \textenglish{New York has got a nickname: the \href{https://youtu.be/Jha4OkG-ixw}{big} apple.}
\item[{français}] New York a un surnom : la grosse pomme.
\end{description}
\item \href{http://www.wordreference.com/enfr/which}{which} qui s'écrit phonétiquement \href{https://en.oxforddictionaries.com/definition/which}{\emph{wɪtʃ}}. Exemple :
\begin{description}
\item[{english}] \textenglish{Pick up a word in the list. \href{https://youtu.be/5fR\_\_LXDkRg}{Which} one?}
\item[{français}] Choisis un mot dans la liste. Lequel ?
\end{description}
\end{enumerate}

\subsection{Le son [e] noté aussi [ɛ] comme dans les mots anglais}
\label{sec:orgd503395}
\begin{enumerate}
\item \href{http://www.wordreference.com/enfr/bed}{bed} qui s'écrit phonétiquement \href{https://en.oxforddictionaries.com/definition/bed}{\emph{bɛd}}. Exemple d'utilisation du mot bed :
\begin{description}
\item[{english}] \textenglish{It's time to go to \href{https://youtu.be/urARKkLo6MY}{bed}.}
\item[{français}] C'est l'heure d'aller se coucher.
\end{description}
\item \href{http://www.wordreference.com/enfr/bread}{bread} qui s'écrit phonétiquement \href{https://en.oxforddictionaries.com/definition/bread}{\emph{brɛd}}. Exemple d'utilisation du
mot :
\begin{description}
\item[{english}] \textenglish{French people are famous for their \href{https://youtu.be/Ynm9Wrznz4I}{bread}.}
\item[{français}] Les Français sont célèbres pour leur pain.
\end{description}
\item \href{http://www.wordreference.com/enfr/said}{said} qui s'écrit phonétiquement \href{https://en.oxforddictionaries.com/definition/said}{\emph{sɛd}}. Exemple d'utilisation du mot :
\begin{description}
\item[{english}] \textenglish{\href{https://www.azlyrics.com/lyrics/beatles/yesterday.html}{Yesterday} you said that same thing.}
\item[{français}] Hier tu as dit cette même chose.
\end{description}
\item \href{http://www.wordreference.com/enfr/friend}{friend} qui s'écrit phonétiquement \href{https://en.oxforddictionaries.com/definition/friend}{\emph{frɛnd}}. Exemple d'utilisation du mot :
\begin{description}
\item[{english}] \textenglish{\href{https://youtu.be/q-9kPks0IfE}{I'll be there for you} my friend.}
\item[{français}] Je serais là pour toi mon ami(e).
\end{description}
\end{enumerate}
\begin{enumerate}
\item Mot français qui utilise le même son
\label{sec:org74483b7}
\href{http://www.wordreference.com/fren/m\%25C3\%25A8re}{mère} qui s'écrit phonétiquement \href{http://www.larousse.fr/dictionnaires/francais-anglais/m\%25c3\%25a8re/50499}{\emph{mεr}}
\end{enumerate}
\subsection{Le son [æ] noté aussi [a] comme dans les mots anglais}
\label{sec:orgb23ea7e}
\begin{enumerate}
\item \href{http://www.wordreference.com/enfr/bat}{bat} qui s'écrit phonétiquement \href{https://en.oxforddictionaries.com/definition/bat}{\emph{bat}}. Exemple d'utilisation du mot :
\begin{description}
\item[{english}] \textenglish{Have you ever noticed that \href{https://www.youtube.com/watch?v=O24Ui015YXM}{Batman} means the \href{https://youtu.be/24howVwYgHY}{man} who
is a \href{https://youtu.be/eozL5n2Plmc}{bat}?}
\item[{français}] As-tu déjà remarqué que Batman signifie l'homme qui
est une chauve-souris ?
\end{description}
\item \href{http://www.wordreference.com/enfr/cat}{cat} qui s'écrit phonétiquement \href{https://en.oxforddictionaries.com/definition/cat}{\emph{kat}}. Exemple d'utilisation du mot :
\begin{description}
\item[{english}] \textenglish{What \href{https://youtu.be/7FjChUY0zgQ}{about} Catwoman? Is she a \href{https://youtu.be/eNQazP-wdj4}{cat}?}
\item[{français}] Qu'en est-il de Catwoman ? Est-elle une chatte ?
\end{description}
\item \href{http://www.wordreference.com/enfr/that}{that} qui s'écrit phonétiquement \href{https://en.oxforddictionaries.com/definition/that}{\emph{ðat}} lorsqu'il est considéré comme
un pronom, un déterminant, un adverbe. En revanche, en tant que
conjonction il se prononce parfois différemment \href{https://en.oxforddictionaries.com/definition/that}{\emph{ðət}}. Exemple d'utilisation du mot :
\begin{description}
\item[{english}] \textenglish{\href{https://youtu.be/HAlz5TiKOCM}{That} house is really big.}
\item[{français}] Cette maison est vraiment grande.
\end{description}
\item \href{http://www.wordreference.com/enfr/hand}{hand} qui s'écrit phonétiquement \href{https://en.oxforddictionaries.com/definition/hand}{\emph{hand}}. Exemple d'utilisation du mot : 
\begin{description}
\item[{english}] \textenglish{Maradona had scored with his \href{https://youtu.be/KDKBY9FqwQg}{hand} during a famous
match Argentina versus England.}
\item[{français}] Maradona avait marqué avec sa main durant un célèbre
match Argentine contre Angleterre.
\end{description}
\end{enumerate}

\section{Centre Vowels (langue relativement plate)}
\label{sec:org05de138}
Comme pour les Front Vowels il y en a 4 donc je vous proposerais 4
exemples à chaque fois.
\subsection{Le son [ə] qui se note aussi [ɜ] comme dans les mots anglais}
\label{sec:org909cc21}
\begin{enumerate}
\item \href{http://www.wordreference.com/enfr/ago}{ago} qui s'écrit phonétiquement \href{https://en.oxforddictionaries.com/definition/ago}{\emph{əˈɡəʊ}}. Exemple d'utilisation du mot :
\begin{description}
\item[{english}] \textenglish{I started to learn English when I was in Middle School
25 years \href{https://youtu.be/RO4fWbM3WA8}{ago}!}
\item[{français}] J'ai commencé à apprendre l'Anglais quand j'étais au
Collège il y a 25 ans.
\end{description}
\item \href{http://www.wordreference.com/enfr/today}{today} qui s'écrit phonétiquement \href{https://en.oxforddictionaries.com/definition/today}{\emph{təˈdeɪ}}. Exemple d'utilisation du mot :
\begin{description}
\item[{english}] \textenglish{\href{https://youtu.be/yCSLK0WCUd8}{Today} is Wednesday.}
\item[{français}] Aujourd'hui c'est mercredi.
\end{description}
\item \href{http://www.wordreference.com/enfr/rhythm}{rhythm} (attention il y a bien 2 h) qui s'écrit phonétiquement
\href{https://en.oxforddictionaries.com/definition/rhythm}{\emph{ˈrɪð(ə)m}}. Exemple d'utilisation du mot :
\begin{description}
\item[{english}] \textenglish{Did you know that any language has got its own \href{https://youtu.be/XQJVoS3SlX0}{rhythm}?}
\item[{français}] Saviez-vous que chaque langue a son propre rythme ?
\end{description}
\item \href{http://www.wordreference.com/enfr/supply}{supply} qui s'écrit phonétiquement \href{https://en.oxforddictionaries.com/definition/supply}{\emph{səˈplʌɪ}}. Exemple d'utilisation du mot : 
\begin{description}
\item[{english}] \textenglish{Do not worry I will always \href{https://youtu.be/qEd6QUbK2Mw}{supply} you with multimedia
documents, audio links, videos, text.}
\item[{français}] Ne vous inquiétez pas, je vous fournirai toujours des
documents multimédia, des liens audio, des vidéos, du texte.
\end{description}
\end{enumerate}

\subsection{Le son [ɜː] qui se note aussi [əː] comme dans les mots anglais}
\label{sec:org53c6f2e}
\begin{enumerate}
\item \href{http://www.wordreference.com/enfr/bird}{bird} qui s'écrit phonétiquement \href{https://en.oxforddictionaries.com/definition/bird}{\emph{bəːd}}. Exemple d'utilisation du mot :
\begin{description}
\item[{english}] \textenglish{\href{https://genius.com/The-beatles-free-as-a-bird-lyrics}{Free as a bird.}}
\item[{français}] Libre comme l'air (littéralement : libre tel un
oiseau)
\end{description}
\item \href{http://www.wordreference.com/enfr/turn}{turn} qui s'écrit phonétiquement \href{https://en.oxforddictionaries.com/definition/turn}{\emph{təːn}}. Exemple d'utilisation du mot : 
\begin{description}
\item[{english}] \textenglish{\href{https://youtu.be/WLTI2rWAlV4}{Turn} off your TV, actually, you should sell it.}
\item[{français}] Éteins ta télé, en fait, tu devrais la vendre.
\end{description}
\item \href{http://www.wordreference.com/enfr/worse}{worse} qui s'écrit phonétiquement \href{https://en.oxforddictionaries.com/definition/worse}{\emph{wəːs}}. Exemple d'utilisation du mot : 
\begin{description}
\item[{english}] \textenglish{I don't know if watching silly cat videos on YouTube
is \href{https://youtu.be/JHWhzS0zdOc}{worse} than watching TV, but you won't improve your
intellectual level by doing so.}
\item[{français}] Je ne sais pas si regarder des vidéos débiles de chat
sur YouTube est pire que de regarder la télé, mais tu
n'augmenteras pas ton niveau intellectuel en le faisant.
\end{description}
\item \href{http://www.wordreference.com/enfr/learn}{learn} qui s'écrit phonétiquement \href{https://en.oxforddictionaries.com/definition/learn}{\emph{ləːn}}. Exemple d'utilisation du mot :
\begin{description}
\item[{english}] \textenglish{If you want to \href{https://youtu.be/1xXs7MAsB0w}{learn} English, you need to \href{https://youtu.be/wmCAKUFKZ7Y}{practice}
the sounds.}
\item[{français}] Si tu veux apprendre l'anglais, il faut que tu
pratiques les sons.
\end{description}
\end{enumerate}
\subsection{Le son [ʌ] comme dans les mots anglais}
\label{sec:org2b5aca8}
\begin{enumerate}
\item \href{http://www.wordreference.com/enfr/cup}{cup} qui s'écrit phonétiquement \href{https://en.oxforddictionaries.com/definition/cup}{\emph{kʌp}}. Exemple d'utilisation du mot :
\begin{description}
\item[{english}] \textenglish{Do you want a \href{https://youtu.be/pjcOzqxu4JQ}{cup} of tea?}
\item[{français}] Voulez-vous une tasse de thé ?
\end{description}
\item \href{http://www.wordreference.com/enfr/something}{something} qui s'écrit phonétiquement \href{https://en.oxforddictionaries.com/definition/something}{\emph{ˈsʌmθɪŋ}}. Exemple d'utilisation du mot : 
\begin{description}
\item[{english}] \textenglish{She does \href{https://youtu.be/UelDrZ1aFeY}{something} special with her voice that I can't
\href{https://genius.com/The-beatles-something-lyrics}{describe}, but I like it.}
\item[{français}] Elle fait quelque chose de spécial avec sa voix que
je ne peux pas décrire, mais j'aime ça.
\end{description}
\item \href{http://www.wordreference.com/enfr/fun}{fun} qui s'écrit phonétiquement \href{https://en.oxforddictionaries.com/definition/fun}{\emph{fʌn}}. Exemple d'utilisation du mot : 
\begin{description}
\item[{english}] \textenglish{Some studies have shown that having \href{https://youtu.be/KXJNoC6CuYE}{fun} is the best
way to learn.}
\item[{français}] Des études ont montré que s'amuser est le meilleur
moyen pour apprendre.
\end{description}
\item \href{http://www.wordreference.com/enfr/luck}{luck} qui s'écrit phonétiquement \href{https://en.oxforddictionaries.com/definition/luck}{\emph{lʌk}}. Exemple d'utilisation du mot :
\begin{description}
\item[{english}] \textenglish{They wish you good \href{https://youtu.be/LQCY2zL0Jr8}{luck} for your \href{https://youtu.be/o61dD6hwrdM}{learning}.}
\item[{français}] Ils vous souhaietent bonne chance pour votre
apprentissage.
\end{description}
\end{enumerate}
\subsection{Le son [ɑː] comme dans les mots anglais}
\label{sec:org73a1177}
\begin{enumerate}
\item \href{http://www.wordreference.com/enfr/father}{father} qui s'écrit phonétiquement \href{https://en.oxforddictionaries.com/definition/father}{\emph{ˈfɑːðə}}. Exemple d'utilisation du mot :
\begin{description}
\item[{english}] \textenglish{My \href{https://youtu.be/MZDAUbeSwNY}{father} used to tell me that you never waste your
time when you think.}
\item[{français}] Mon père avait l'habitude de me dire qu'on ne perd
jamais son temps à réfléchir.
\end{description}
\item \href{http://www.wordreference.com/enfr/arm}{arm} qui s'écrit phonétiquement \href{https://en.oxforddictionaries.com/definition/arm}{\emph{ɑːm}}. Exemple d'utilisation du mot :
\begin{description}
\item[{english}] \textenglish{We are lucky because we have two \href{https://youtu.be/tlhQghmuMf8}{arms} and two legs;
sorry if one of them is harmed.}
\item[{français}] Nous avons la chance d'avoir deux bras et deux
jambes; désolé si l'un d'eux est blessé.
\end{description}
\item \href{http://www.wordreference.com/enfr/dance}{dance} qui s'écrit phonétiquement \href{https://en.oxforddictionaries.com/definition/dance}{\emph{dɑːns}}. Exemple d'utilisation du mot :
\begin{description}
\item[{english}] \textenglish{Would you like to \href{https://youtu.be/aagbeWUDe7w}{dance} with me pretty lady?}
\item[{français}] Veux-tu danser avec moi jolie demoiselle ?
\end{description}
\item \href{http://www.wordreference.com/enfr/half}{half} qui s'écrit phonétiquement \href{https://en.oxforddictionaries.com/definition/half}{\emph{hɑːf}}. Exemple d'utilisation du mot :
\begin{description}
\item[{english}] \textenglish{\href{https://youtu.be/XWamnSNgiCM}{Half} time! That's the right moment to get some drinks!}
\item[{français}] Mi-temps ! C'est le bon moment pour prendre à boire !
\end{description}
\end{enumerate}
\section{Back Vowels (langue vers l'arrière)}
\label{sec:org02857bf}
Comme pour les Centre Vowels il y en a 4 donc je vous proposerais 4
exemples à chaque fois.
\subsection{Le son [uː] comme dans les mots anglais}
\label{sec:org635c7e0}
\begin{enumerate}
\item \href{http://www.wordreference.com/enfr/too}{too} qui s'écrit phonétiquement \href{https://en.oxforddictionaries.com/definition/too}{\emph{tuː}}. Exemple d'utilisation du mot :
\begin{description}
\item[{english}] \textenglish{I like to speak English, and you? Me \href{https://youtu.be/RaveinO4\_vs}{too}.}
\item[{français}] J'aime parler Anglais, et toi ? Moi aussi.
\end{description}
\item \href{http://www.wordreference.com/enfr/few}{few} qui s'écrit phonétiquement \href{https://en.oxforddictionaries.com/definition/few}{\emph{fjuː}}. Exemple d'utilisation du mot :
\begin{description}
\item[{english}] \textenglish{\href{https://youtu.be/r3TaGhdqEiA}{Few} people understand the key role of phonetics.}
\item[{français}] Peu de gens comprennent le rôle clé de la phonétique.
\end{description}
\item \href{http://www.wordreference.com/enfr/rule}{rule} qui s'écrit phonétiquement \href{https://en.oxforddictionaries.com/definition/rule}{\emph{ruːl}}. Exemple d'utilisation du mot : 
\begin{description}
\item[{english}] \textenglish{\href{https://youtu.be/rStL7niR7gs}{Do you want to rule?}}
\item[{français}] Voulez-vous diriger ?
\end{description}
\item \href{http://www.wordreference.com/enfr/lose}{lose} qui s'écrit phonétiquement \href{https://en.oxforddictionaries.com/definition/lose}{\emph{luːz}}. Exemple d'utilisation du mot :
\begin{description}
\item[{english}] \textenglish{You \href{https://youtu.be/UNcCTgA5lzo}{lose} the game this time, do you want to try again?}
\item[{français}] Vous avez perdu la partie cette fois, voulez-vous
essayer à nouveau ?
\end{description}
\end{enumerate}
\subsection{Le son [ʊ] comme dans les mots anglais}
\label{sec:org2ccc235}
\begin{enumerate}
\item \href{http://www.wordreference.com/enfr/good}{good} qui s'écrit phonétiquement \href{https://en.oxforddictionaries.com/definition/good}{\emph{ɡʊd}}. Exemple d'utilisation du mot :
\begin{description}
\item[{english}] \textenglish{Your book is \href{https://youtu.be/o3TQSaqHBtM}{good}.}
\item[{français}] Votre le livre est bon.
\end{description}
\item \href{http://www.wordreference.com/enfr/put}{put} qui s'écrit phonétiquement \href{https://en.oxforddictionaries.com/definition/put}{\emph{pʊt}}. Exemple d'utilisation du mot :
\begin{description}
\item[{english}] \textenglish{\href{https://youtu.be/BSpoa7TsiD0}{Put} your energy in something you like.}
\item[{français}] Mettez votre énergie dans quelque chose que vous
aimez.
\end{description}
\item \href{http://www.wordreference.com/enfr/would}{would} qui s'écrit phonétiquement \href{https://en.oxforddictionaries.com/definition/would}{\emph{wʊd}}. Exemple d'utilisation du mot :
\begin{description}
\item[{english}] \textenglish{\href{https://youtu.be/wRSNm3pr100}{Would} you like to drink something?}
\item[{français}] Voulez-vous boire quelque chose ?
\end{description}
\item \href{http://www.wordreference.com/enfr/look}{look} qui s'écrit phonétiquement \href{https://en.oxforddictionaries.com/definition/look}{\emph{lʊk}}. Exemple d'utilisation du mot :
\begin{description}
\item[{english}] \textenglish{\href{https://youtu.be/b4xcpMCPhfE}{Look} at this!}
\item[{français}] Regarde ça !
\end{description}
\end{enumerate}
\subsection{Le son [ɔː] comme dans les mots anglais}
\label{sec:orgab83d97}
\begin{enumerate}
\item \href{http://www.wordreference.com/enfr/pork}{pork} qui s'écrit phonétiquement \href{https://en.oxforddictionaries.com/definition/pork}{\emph{pɔːk}}. Exemple d'utilisation du mot :
\begin{description}
\item[{english}] \textenglish{Do you eat \href{https://youtu.be/WqTJbyfewzw}{pork}?}
\item[{français}] Mangez-vous du porc ?
\end{description}
\item \href{http://www.wordreference.com/enfr/law}{law} qui s'écrit phonétiquement \href{https://en.oxforddictionaries.com/definition/law}{\emph{lɔː}}. Exemple d'utilisation du mot : 
\begin{description}
\item[{english}] \textenglish{\href{https://youtu.be/us5CUAsH0U0}{Hackers like to say: code is law.}}
\item[{français}] Les hackers aiment dire que le code est la loi.
\end{description}
\item \href{http://www.wordreference.com/enfr/taught}{taught} qui s'écrit phonétiquement \href{https://en.oxforddictionaries.com/definition/taught}{\emph{tɔːt}}. Exemple d'utilisation du mot :
\begin{description}
\item[{english}] \textenglish{I \href{https://youtu.be/U2BG2\_K2fGk}{taught} you how to write English phonetics yesterday.}
\item[{français}] Hier je t'ai enseigné comment écrire la phonétique
Anglaise.
\end{description}
\item \href{http://www.wordreference.com/enfr/thought}{thought} qui s'écrit phonétiquement \href{https://en.oxforddictionaries.com/definition/thought}{\emph{θɔːt}}. Exemple d'utilisation du mot : 
\begin{description}
\item[{english}] \textenglish{Tell me your \href{https://youtu.be/8kR-GDbYHhc}{thoughts}.}
\item[{français}] Raconte-moi tes pensées.
\end{description}
\end{enumerate}
\subsection{Le son [ɒ] comme dans les mots anglais}
\label{sec:orgb3f3634}
\begin{enumerate}
\item \href{http://www.wordreference.com/enfr/got}{got} qui s'écrit phonétiquement \href{https://en.oxforddictionaries.com/definition/got}{\emph{ɡɒt}}. Exemple d'utilisation du mot :
\begin{description}
\item[{english}] \textenglish{I \href{https://youtu.be/Bo09BiPb24Y}{got} you. (slang: \href{https://youtu.be/EWRaAbVUkjA}{Gotcha})}
\item[{français}] Je t'ai eu. (argot: Gotcha)
\end{description}
\item \href{http://www.wordreference.com/enfr/watch}{watch} qui s'écrit phonétiquement \href{https://en.oxforddictionaries.com/definition/watch}{\emph{wɒtʃ}}. Exemple d'utilisation du mot :
\begin{description}
\item[{english}] \textenglish{\href{https://youtu.be/qOs8MagOfwg}{Watch} this video carefully.}
\item[{français}] Regardez attentivement cette vidéo.
\end{description}
\item \href{http://www.wordreference.com/enfr/rob}{rob} qui s'écrit phonétiquement \href{https://en.oxforddictionaries.com/definition/rob}{\emph{rɒb}}. Exemple d'utilisation du mot :
\begin{description}
\item[{english}] \textenglish{Are you planning to \href{https://youtu.be/X3uZ0Gf104A}{rob} a bank? I discourage you to do
that.}
\item[{français}] Êtes-vous en train d'envisager de cambrioler une
banque ? Je vous déconseille de faire ça.
\end{description}
\item \href{http://www.wordreference.com/enfr/top}{top} qui s'écrit phonétiquement \href{https://en.oxforddictionaries.com/definition/top}{\emph{tɒp}}. Exemple d'utilisation du mot : 
\begin{description}
\item[{english}] \textenglish{\href{https://youtu.be/gPaD513xWOY}{Top} videos are sometime very boring.}
\item[{français}] Les vidéos de top sont parfois très ennuyeuses.
\end{description}
\end{enumerate}
\section{Diphthong Vowels}
\label{sec:orgbc226d9}
Il y a 7 diphtongues en anglais.
\subsection{Le son [eɪ] comme dans les mots anglais}
\label{sec:orgccc0406}
\begin{enumerate}
\item \href{http://www.wordreference.com/enfr/snake}{snake} qui s'écrit phonétiquement \href{https://en.oxforddictionaries.com/definition/snake}{\emph{sneɪk}}. Exemple d'utilisation du mot :
\begin{description}
\item[{english}] \textenglish{\href{https://youtu.be/MOltIVdyAHQ}{Snakes} regularly shed their skin.}
\item[{français}] Les serpents perdent régulièrement leur peau.
\end{description}
\item \href{http://www.wordreference.com/enfr/pay}{pay} qui s'écrit phonétiquement \href{https://en.oxforddictionaries.com/definition/pay}{\emph{peɪ}}. Exemple d'utilisation du mot : 
\begin{description}
\item[{english}] \textenglish{How much would you be able to \href{https://youtu.be/mBuLm5XeF44}{pay} for additional
content?}
\item[{français}] Combien seriez-vous capable de payer pour du contenu
supplémentaire ?
\end{description}
\item \href{http://www.wordreference.com/enfr/mail}{mail} qui s'écrit phonétiquement \href{https://en.oxforddictionaries.com/definition/mail}{\emph{meɪl}}. Exemple d'utilisation du mot :
\begin{description}
\item[{english}] \textenglish{The post office redirected the \href{https://youtu.be/KX1CSSZa1v0}{mail} to my new address.}
\item[{français}] Le bureau de poste a fait suivre le courrier à ma
nouvelle adresse.
\end{description}
\item \href{http://www.wordreference.com/enfr/great}{great} qui s'écrit phonétiquement \href{https://en.oxforddictionaries.com/definition/great}{\emph{ɡreɪt}}. Exemple d'utilisation du mot :
\begin{description}
\item[{english}] \textenglish{Your content is \href{https://youtu.be/e0qM84DWXzA}{great}!}
\item[{français}] Ton contenu est génial !
\end{description}
\end{enumerate}
\subsection{Le son [ɔɪ] comme dans les mots anglais}
\label{sec:org4d2fca6}
\begin{enumerate}
\item \href{http://www.wordreference.com/enfr/toy}{toy} qui s'écrit phonétiquement \href{https://en.oxforddictionaries.com/definition/toy}{\emph{tɔɪ}}. Exemple d'utilisation du mot :
\begin{description}
\item[{english}] \textenglish{The little boy was delighted with all his \href{https://youtu.be/1qbuZhVUj\_g}{toys}.}
\item[{français}] Le petit garçon était enchanté par tous ses jouets.
\end{description}
\item \href{http://www.wordreference.com/enfr/choice}{choice} qui s'écrit phonétiquement \href{https://en.oxforddictionaries.com/definition/choice}{\emph{tʃɔɪs}}. Exemple d'utilisation du mot :
\begin{description}
\item[{english}] \textenglish{Looking at my additional content is your \href{https://youtu.be/qBfeK\_IIHag}{choice}.}
\item[{française}] Regarder mon contenu supplémentaire est votre choix.
\end{description}
\item \href{http://www.wordreference.com/enfr/joy}{joy} qui s'écrit phonétiquement \href{https://en.oxforddictionaries.com/definition/joy}{\emph{dʒɔɪ}}. Exemple d'utilisation du mot :
\begin{description}
\item[{english}] \textenglish{The music creates a sensation of \href{https://youtu.be/-GjW1pSYgUk}{joy} and playfulness.}
\item[{français}] La musique crée une sensation de joie et de gaieté.
\end{description}
\item \href{http://www.wordreference.com/enfr/oyster}{oyster} qui s'écrit phonétiquement \href{https://en.oxforddictionaries.com/definition/oyster}{\emph{ˈɔɪstə}}. Exemple d'utilisation du mot :
\begin{description}
\item[{english}] \textenglish{Inside the \href{https://youtu.be/PVn6b9QQZeM}{oyster}, I found a pearl.}
\item[{français}] À l'intérieur de l'huître, j'ai trouvé une perle.
\end{description}
\end{enumerate}
\subsection{Le son [aɪ] comme dans les mots anglais}
\label{sec:ai}
\begin{enumerate}
\item \href{http://www.wordreference.com/enfr/my}{my} qui s'écrit phonétiquement \href{https://dictionary.cambridge.org/dictionary/english/my}{\emph{maɪ}}. Exemple d'utilisation du mot :
\begin{description}
\item[{english}] \textenglish{\href{https://youtu.be/SMwEkjcEACM}{My} content is made to help you \href{https://www.youtube.com/watch?v=m\_uWS6K-VF8\&list=PL0J5xb8JH3VukoRHgk86Yr9BSVeBewCuZ}{progress in English}.}
\item[{français}] Mon contenu est fait pour vous aider à progresser en
anglais.
\end{description}
\item \href{http://www.wordreference.com/enfr/while}{while} qui s'écrit phonétiquement \href{https://dictionary.cambridge.org/dictionary/english/while}{\emph{waɪl}}. Exemple d'utilisation du mot :
\begin{description}
\item[{english}] \textenglish{She partied \href{https://youtu.be/8q182kWAhiM}{while} I worked.}
\item[{français}] Elle faisait la fête alors que je travaillais.
\end{description}
\item \href{http://www.wordreference.com/enfr/might}{might} qui s'écrit phonétiquement \href{https://dictionary.cambridge.org/dictionary/english/might}{\emph{maɪt}}. Exemple d'utilisation du mot :
\begin{description}
\item[{english}] \textenglish{Hurricanes show us the \href{https://youtu.be/Nqlr35WnqTk}{might} of nature.}
\item[{français}] Les ouragans nous démontrent la puissance de la
nature.
\end{description}
\item \href{http://www.wordreference.com/enfr/life}{life} qui s'écrit phonétiquement \href{https://dictionary.cambridge.org/dictionary/english/life}{\emph{laɪf}}. Exemple d'utilisation du mot :
\begin{description}
\item[{english}] \textenglish{The author withdrew from public \href{https://youtu.be/zyKGKoGACVk}{life}.}
\item[{français}] L'auteur s'est retiré de la vie publique.
\end{description}
\end{enumerate}
\subsection{Le son [əʊ] comme dans les mots anglais}
\label{sec:org7325c5a}
\begin{enumerate}
\item \href{http://www.wordreference.com/enfr/alone}{alone} qui s'écrit phonétiquement \href{https://en.oxforddictionaries.com/definition/alone}{\emph{əˈləʊn}}. Exemple d'utilisation du mot :
\begin{description}
\item[{english}] \textenglish{I experience real \href{https://youtu.be/cnsk7iXFCtY}{joy} when I am alone in nature.}
\item[{français}] Je ressens une joie réelle quand je suis seul dans la
nature.
\end{description}
\item \href{http://www.wordreference.com/enfr/goat}{goat} qui s'écrit phonétiquement \href{https://en.oxforddictionaries.com/definition/goat}{\emph{ɡəʊt}}. Exemple d'utilisation du mot :
\begin{description}
\item[{english}] \textenglish{Behind a door there is a sports car and behind each of
the other two there is a \href{https://youtu.be/pEHWbpy-EpI}{goat}.}
\item[{français}] Derrière une porte il y a une voiture de sport et
derrière chacune des deux autres il y a une chèvre.
\end{description}
\item \href{http://www.wordreference.com/enfr/hope}{hope} qui s'écrit phonétiquement \href{https://en.oxforddictionaries.com/definition/hope}{\emph{həʊp}}. Exemple d'utilisation du mot :
\begin{description}
\item[{english}] \textenglish{I \href{https://youtu.be/\_pKcv0Fml-A}{hope} you will enjoy your stay.
\item[{français}] J'espère que vous apprécierez votre séjour.}
\end{description}
\item \href{http://www.wordreference.com/enfr/road}{road} qui s'écrit phonétiquement \href{https://en.oxforddictionaries.com/definition/road}{\emph{rəʊd}}. Exemple d'utilisation du mot :
\begin{description}
\item[{english}] \textenglish{\href{https://youtu.be/jzmy6iUGDo8}{Body like a back road.}}
\item[{français}] Un corps comme une route de retour.
\end{description}
\end{enumerate}
\subsection{Le son [aʊ] comme dans les mots anglais}
\label{sec:org729933e}
\begin{enumerate}
\item \href{http://www.wordreference.com/enfr/now}{now} qui s'écrit phonétiquement \href{https://en.oxforddictionaries.com/definition/now}{\emph{naʊ}}. Exemple d'utilisation du mot :
\begin{description}
\item[{english}] \textenglish{I am \href{https://youtu.be/xcpxjx2fy\_E}{now} completely free and unencumbered.}
\item[{français}] Je suis désormais complètement libre et sans contrainte.
\end{description}
\item \href{http://www.wordreference.com/enfr/round}{round} qui s'écrit phonétiquement \href{https://en.oxforddictionaries.com/definition/round}{\emph{raʊnd}}. Exemple d'utilisation du mot :
\begin{description}
\item[{english}] \textenglish{The boxer won the fight in the second \href{https://youtu.be/oGTBax-Cu4Q}{round}.}
\item[{français}] Le boxeur a gagné le combat au deuxième round.
\end{description}
\item \href{http://www.wordreference.com/enfr/mouth}{mouth} qui s'écrit phonétiquement \href{https://en.oxforddictionaries.com/definition/mouth}{\emph{maʊθ}}. Exemple d'utilisation du mot :
\begin{description}
\item[{english}] \textenglish{In order to produce a vowel you need to open your
\href{https://youtu.be/kkDHKSNrJ5g}{mouth}.}
\item[{français}] Afin de produire une voyelle vous devez ouvrir votre
bouche.
\end{description}
\item \href{http://www.wordreference.com/enfr/brown}{brown} qui s'écrit phonétiquement \href{https://en.oxforddictionaries.com/definition/brown}{\emph{braʊn}}. Exemple d'utilisation du mot :
\begin{description}
\item[{english}] \textenglish{\href{https://youtu.be/OwTXBBU0JLo}{Brown} is just a colour.}
\item[{français}] Le marron est juste une couleur.
\end{description}
\end{enumerate}
\subsection{Le son [ɪə] comme dans les mots anglais}
\label{sec:org2e2326e}
\begin{enumerate}
\item \href{http://www.wordreference.com/enfr/weird}{weird} qui s'écrit phonétiquement \href{https://en.oxforddictionaries.com/definition/weird}{\emph{wɪəd}}. Exemple d'utilisation du mot :
\begin{description}
\item[{english}] \textenglish{He always has \href{https://youtu.be/fcdUXnt87ng}{weird} dreams that \href{https://youtu.be/FikYhD7bXYE}{nobody} understands.}
\item[{français}] Il fait toujours des rêves bizarres que personne ne
comprend.
\end{description}
\item \href{http://www.wordreference.com/enfr/beer}{beer} qui s'écrit phonétiquement \href{https://en.oxforddictionaries.com/definition/beer}{\emph{bɪə}}. Exemple d'utilisation du mot :
\begin{description}
\item[{english}] \textenglish{Football supporters usually drink \href{https://youtu.be/I1fsk4k-bOs}{beer}.}
\item[{français}] Les supporters de foot boivent habituellement de la
bière (attention à consommer avec modération).
\end{description}
\item \href{http://www.wordreference.com/enfr/near}{near} qui s'écrit phonétiquement \href{https://en.oxforddictionaries.com/definition/near}{\emph{nɪə}}. Exemple d'utilisation du mot : 
\begin{description}
\item[{english}] \textenglish{UK is \href{https://youtu.be/xIS9K-bNt3M}{near} from France.}
\item[{français}] Le Royaume-Uni est proche de la France.
\end{description}
\item \href{http://www.wordreference.com/enfr/steer}{steer} qui s'écrit phonétiquement \href{https://en.oxforddictionaries.com/definition/steer}{\emph{stɪə}}. Exemple d'utilisation du mot :
\begin{description}
\item[{english}] \textenglish{The politician \href{https://youtu.be/z\_vSRFODAxU}{steered} the conversation to a different
topic.}
\item[{français}] L'homme politique a orienté la conversation vers un autre sujet.
\end{description}
\end{enumerate}
\subsection{Le son [eə] qui s'écrit aussi [ɛə] comme dans les mots anglais}
\label{sec:org0206cfe}
\begin{enumerate}
\item \href{http://www.wordreference.com/enfr/bear}{bear} qui s'écrit phonétiquement \href{https://dictionary.cambridge.org/dictionary/english/bear}{\emph{bɛə}}. Exemple d'utilisation du mot :
\begin{description}
\item[{english}] \textenglish{This noise is difficult to \href{https://youtu.be/NJ6jv\_lPBN8}{bear}.}
\item[{français}] Ce bruit est difficile à supporter.
\end{description}
\item \href{http://www.wordreference.com/enfr/rare}{rare} qui s'écrit phonétiquement \href{https://dictionary.cambridge.org/dictionary/english/rare}{\emph{rɛə}}. Exemple d'utilisation du mot :
\begin{description}
\item[{english}] \textenglish{The consultant is an expert in \href{https://youtu.be/hPncU3924fU}{rare} illnesses.}
\item[{français}] Le médecin spécialiste est expert en maladies rares.
\end{description}
\item \href{http://www.wordreference.com/enfr/there}{there} qui s'écrit phonétiquement \href{https://dictionary.cambridge.org/dictionary/english/there}{\emph{ðeər}}. Exemple d'utilisation du mot :
\begin{description}
\item[{english}] \textenglish{My friend is always \href{https://youtu.be/fg9pkAYvrSM}{there} for me when I need her.}
\item[{français}] Mon amie est toujours là pour moi quand j'ai besoin
d'elle.
\end{description}
\item \href{http://www.wordreference.com/enfr/care}{care} qui s'écrit phonétiquement \href{https://dictionary.cambridge.org/dictionary/english/care}{\emph{keər}}. Exemple d'utilisation du mot :
\begin{description}
\item[{english}] \textenglish{Babies need constant \href{https://youtu.be/ClrSEz\_tBZw}{care}.}
\item[{français}] Les bébés ont besoin d'une attention constante.
\end{description}
\end{enumerate}