\part{Conclusion}\label{chap:conc}

Voilà, nous sommes arrivés à la fin de ce voyage. Si je prends la
métaphore du voyage c'est à dessein parce que de la même manière que
l'on découvre une nouvelle destination pour la première fois il en va
de même pour un nouveau domaine tel que la phonétique. Lors du premier
voyage on est souvent impressionné, surpris mais à moins d'y passer
beaucoup de temps en général on ne vit qu'une expérience
superficiel. Selon le degré d'implication on peut déjà pressentir le
désir d'approfondir. Et c'est précisément mon but avec ce premier
ouvrage sur le sujet.

D'ailleurs je suis d'ores et déjà motivé pour poursuivre mon
exploration de l'univers merveilleux de la phonétique. Comme je l'ai
déjà dit plus haut dans l'ouvrage l'API ne concerne pas que l'anglais,
il concerne toutes les langues humaines. Bien qu'il me serait
difficile de traiter toutes les langues du monde, je suis déjà en
cours de rédaction d'une introduction à la phonétique française. Et
une bonne partie des sons et symboles utilisés pour l'anglais sont
valables également pour le français. Parmi les
\href{https://www.youtube.com/playlist?list=PLfKvL-VUSKAnkBk88BAb3oq1MlGVnhwcY}{autres
  langues} que je compte explorer comme je l'ai déjà fait à plusieurs
reprises il y a :
\begin{itemize}
  \item  l'\href{https://www.youtube.com/playlist?list=PLfKvL-VUSKAnM9MWJT9F1z1QZTdb73i7r}{allemand}
  \item
    l'\href{https://www.youtube.com/playlist?list=PLfKvL-VUSKAkXu2x3Fp74QxxYUVP43haA}{arabe}
  \item le
    \href{https://www.youtube.com/playlist?list=PLfKvL-VUSKAl4R0Mh7sKvQjqCsiEEa6D9}{chinois}
  \item
    l'\href{https://www.youtube.com/playlist?list=PLfKvL-VUSKAm_p6ikI_pTbxNuHco73REt}{espagnol}
  \item
    l'\href{https://www.youtube.com/playlist?list=PLfKvL-VUSKAkbDhpbtXc7RdroMBBeTJx0}{hébreu}
  \item le
    \href{https://www.youtube.com/playlist?list=PLfKvL-VUSKAn0zUUPYsMDd8_1J_UtfRxh}{portugais}
  \item le \href{https://www.youtube.com/playlist?list=PLfKvL-VUSKAk0YrJ3rV6cBj-w6rNCeOJB}{russe}
  \end{itemize}

Pour chaque langue je proposerai des connexions avec la phonétique
anglaise et la phonétique française. C'est pour cette raison que j'ai
commencé par l'anglais et que j'enchaîne avec le français.

Vous trouverez en annexes différentes classifications des sons. En
effet, il existe de nombreuses façons de regrouper les sons et les
linguistes ne sont pas toujours d'accord entre eux pour savoir
laquelle serait <<~la meilleure~>>. Pour ma part, n'étant pas
linguiste professionnel et ayant une approche pragmatique je vous
recommanderais de ne pas accorder trop d'importance à ces
considérations. Comme je l'ai dit et répété ce qui compte c'est votre
capacité d'écoute et votre production sonore. C'est pour ça que le
document est particulièrement centré sur la connexion avec de
nombreuses sources variées\footnote{Cette richesse de documentation
  vidéo sera beaucoup moins abondante concernant les autres
  langues.}.
Si vous souhaitez aller plus loin en anglais vous pouvez tout à fait
vous inscrire dans l'une de mes formations en passant par le
formulaire de contact sur mon blog
\url{http://doyouspeakenglish.fr/contact/}.

J'espère que ce livre vous aura aporté de la valeur et tous vos
commentaires seront les bienvenus que ça soit sur le blog en dessous
des articles ou sous les vidéos de la chaîne YouTube ou via le
formulaire de contact.

Merci pour votre attention et bon voyage au pays des langues.