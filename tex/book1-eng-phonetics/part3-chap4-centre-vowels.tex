\chapter{Centre Vowels (langue relativement plate)}\label{chap:centvow}

\begin{center}
  \begin{figure}[h]
    \centering
    \includegraphics[scale=.85]{../img/cpp/cent-vow-cpp}
    \caption[\exEN{Centre Vowels}]{\exEN{Centre Vowels}, source :~\cite{collins}}
    \label{fig:centr-vow}
  \end{figure}
\end{center}

\speech{4}{voyelles centrales\CW{https://fr.wikipedia.org/wiki/Voyelle_centrale}}

\newpage
\minitoc
\newpage

\section{Le \son \phon{ə} à ne pas confondre avec \href{https://en.wikipedia.org/wiki/Open-mid_central_unrounded_vowel}{\phon{ɜ}} }\label{sec:sonenv}

\notation

\hypertarget{sonenv}{Ce \son} a pour nom technique\dyse{mid-central-vowel} :

\begin{itemize}
\item \exEN{Mid-Central Vowel\CW{https://en.wikipedia.org/wiki/Mid_central_vowel}.}
\item \exFR{Voyelle moyenne centrale\CW{https://fr.wikipedia.org/wiki/Voyelle_moyenne_centrale}.}
\end{itemize}

\indicsound

\properukus{https://youtu.be/RVvn6204I_Y}{https://youtu.be/m1mDSUSwNls}

\begin{enumerate}
\item \exEN{\href{http://www.wordreference.com/enfr/ago}{ago}} qui s'écrit
  phonétiquement
  \href{https://en.oxforddictionaries.com/definition/ago}{\phonm{əˈɡəʊ}}

  \begin{itemize}
  \item\exEN{I started to \href{https://youtu.be/G5dViczwTXo}{learn English} when I was in Middle School
      twenty-five years \href{https://youtu.be/RO4fWbM3WA8}{ago}! Was
    it \href{https://youtu.be/Z9SfV4BDdHg}{really} a \href{https://youtu.be/0OCyFTT0kCY}{good way} to learn it?}
  \item\exFR{J'ai commencé à apprendre l'anglais quand j'étais au
      Collège il y a vingt-cinq ans ! \'Etait-ce réellement une bonne
      façon de l'apprendre ?}
  \end{itemize}
  
\item \exEN{\href{http://www.wordreference.com/enfr/today}{today}} qui
  s'écrit phonétiquement
  \href{https://en.oxforddictionaries.com/definition/today}{\phonm{təˈdeɪ}}

  \begin{itemize}
  \item\exEN{\href{https://youtu.be/yCSLK0WCUd8}{Today} is \href{https://youtu.be/Sox7KmmAEZI}{Wednesday}.}
  \item\exFR{Aujourd'hui c'est mercredi.}
  \end{itemize}
  
\item \exEN{\href{http://www.wordreference.com/enfr/rhythm}{rhythm}}
  (attention il y a bien deux fois la lettre 'h') qui s'écrit phonétiquement
\href{https://en.oxforddictionaries.com/definition/rhythm}{\phonm{ˈrɪð(ə)m}}

\begin{itemize}
\item\exEN{Did you know that all \href{https://www.youtube.com/watch?v=W8B2E11TFe0&list=PLwwOk5fvpuuLgntaf5Z9QKh5uV2Vl2Xme}{languages} have their own \href{https://youtu.be/XQJVoS3SlX0}{rhythm}?}
\item\exFR{Saviez-vous que chaque langue a son propre rythme ?}
\end{itemize}

\item \exEN{\href{http://www.wordreference.com/enfr/supply}{supply}} qui
  s'écrit phonétiquement
  \href{https://en.oxforddictionaries.com/definition/supply}{\phonm{səˈplʌɪ}}

  \begin{itemize}
  \item\exEN{Do not \href{https://youtu.be/Xh4ugYiXF-Q}{worry} I will always \href{https://youtu.be/qEd6QUbK2Mw}{supply} you with multimedia
      documents, audio links, videos, texts, and so on.}
  \item\exFR{Ne vous inquiétez pas, je vous fournirai toujours des
      documents multimédias, des liens audios, des vidéos, des textes\dots}
  \end{itemize}
\end{enumerate}
\newpage

\section{Le \son \phon{ɜː} qui se  note aussi parfois \href{https://en.oxforddictionaries.com/definition/bird}{\phon{əː}} }\label{sec:sonenvlong}

\notation

\hypertarget{sonenvlong}{Ce \son} a pour nom technique\dyse{open-mid-central-unrounded-vowel} :

\begin{itemize}
\item \exEN{Open-Mid Central Unrounded Vowel\CW{https://en.wikipedia.org/wiki/Open-mid_central_unrounded_vowel}.}
\item \exFR{Voyelle mi-ouverte centrale non arrondie\CW{https://fr.wikipedia.org/wiki/Voyelle_mi-ouverte_centrale_non_arrondie}.}
\end{itemize}

\begin{center}
  \begin{figure}[h]
    \centering
    \includegraphics{../img/non-arrondies-wikipedia.png}
    \caption{Voyelles non-arrondies, source : Wikipédia}
    \label{fig:voy-ferm}
  \end{figure}
\end{center}

\indicsound

\properukus{https://youtu.be/dweBtpz3gco}{https://youtu.be/Ehn6XixUBKs}

\begin{enumerate}
\item \exEN{\href{http://www.wordreference.com/enfr/bird}{bird}} qui s'écrit
  phonétiquement
  \href{https://dictionary.cambridge.org/fr/dictionnaire/anglais/bird}{\phonm{bɜːd}}
  
  \begin{itemize}
  \item\exEN{\href{https://genius.com/The-beatles-free-as-a-bird-lyrics}{Free as a bird.}}
  \item\exFR{Libre comme l'air (littéralement : libre tel un
      oiseau)}
  \end{itemize}
  
\item \exEN{\href{http://www.wordreference.com/enfr/turn}{turn}} qui s'écrit
  phonétiquement
  \href{https://dictionary.cambridge.org/fr/dictionnaire/anglais/turn}{\phonm{tɜːn}}
  
  \begin{itemize}
  \item\exEN{\href{https://youtu.be/WLTI2rWAlV4}{Turn} off your TV; in
      fact, you should \href{https://youtu.be/pixcJiRq6Tk}{sell it}.}
  \item\exFR{Éteins ta télé, en fait, tu devrais la vendre.}
  \end{itemize}
  
\item \exEN{\href{http://www.wordreference.com/enfr/worse}{worse}} qui
  s'écrit phonétiquement
  \href{https://dictionary.cambridge.org/fr/dictionnaire/anglais/worse}{\phonm{wɜːs}}

  \begin{itemize}
  \item\exEN{I don't know if watching silly cat videos on YouTube
      is \href{https://youtu.be/JHWhzS0zdOc}{worse} than watching TV, but you won't \href{https://youtu.be/-wcn2EbOIbQ}{improve} your
      intellectual level by doing so.}
  \item\exFR{Je ne sais pas si regarder des vidéos débiles avec des chats
      sur YouTube est pire que de regarder la télé, mais tu
      n'augmenteras pas ton niveau intellectuel en le faisant.}
  \end{itemize}
  
\item \exEN{\href{http://www.wordreference.com/enfr/learn}{learn}} qui
  s'écrit phonétiquement
  \href{https://dictionary.cambridge.org/fr/dictionnaire/anglais/learn}{\phonm{lɜːn}}
  
  \begin{itemize}
  \item\exEN{If you want to \href{https://youtu.be/1xXs7MAsB0w}{learn} \href{https://youtu.be/YEaSxhcns7Y}{English}, you need to \href{https://youtu.be/wmCAKUFKZ7Y}{practice}
      the sounds.}
  \item\exFR{Si tu veux apprendre l'anglais, il faut que tu
      pratiques les sons.}
  \end{itemize}
  
\end{enumerate}
\newpage

\section{Le \son \phon{ʌ}}\label{sec:sonup}

\hypertarget{sonup}{Ce \son} a pour nom technique\dyse{open-mid-back-unrounded-vowel} :

\begin{itemize}
\item \exEN{Open-mid Back Unrounded Vowel\CW{https://en.wikipedia.org/wiki/Open-mid_back_unrounded_vowel}.}
\item \exFR{Voyelle mi-ouverte postérieure non arrondie\CW{https://fr.wikipedia.org/wiki/Voyelle_mi-ouverte_post\%C3\%A9rieure_non_arrondie}.}
\end{itemize}

\begin{center}
  \begin{figure}[h]
    \centering
    \includegraphics{../img/non-arrondies-wikipedia.png}
    \caption{Voyelles non-arrondies, source : Wikipédia}
    \label{fig:voy-ferm}
  \end{figure}
\end{center}

\indicsound

\properukus{https://youtu.be/zUpF0pYoTZ8}{https://youtu.be/_63fTgbG-yQ}


\begin{enumerate}
\item \exEN{\href{http://www.wordreference.com/enfr/cup}{cup}} qui s'écrit
  phonétiquement
  \href{https://en.oxforddictionaries.com/definition/cup}{\phonm{kʌp}}

  \begin{itemize}
  \item\exEN{Do \href{https://youtu.be/R7iN71uJcG0}{you want} a \href{https://youtu.be/pjcOzqxu4JQ}{cup} of tea?}
  \item\exFR{Voulez-vous une tasse de thé ?}
  \end{itemize}
  
\item \exEN{\href{http://www.wordreference.com/enfr/something}{something}}
  qui s'écrit phonétiquement
  \href{https://en.oxforddictionaries.com/definition/something}{\phonm{ˈsʌmθɪŋ}}
  
  \begin{itemize}
  \item\exEN{She does \href{https://youtu.be/UelDrZ1aFeY}{something} special with \href{https://youtu.be/b4xcpMCPhfE}{her voice} that I can't
      \href{https://genius.com/The-beatles-something-lyrics}{describe}, but I like it.}
  \item\exFR{Elle fait quelque chose de spécial avec sa voix que
      je ne peux pas décrire, mais j'aime ça.}
  \end{itemize}
  
\item \exEN{\href{http://www.wordreference.com/enfr/fun}{fun}} qui s'écrit
  phonétiquement
  \href{https://en.oxforddictionaries.com/definition/fun}{\phonm{fʌn}}

  \begin{itemize}
  \item\exEN{Some studies have shown that having \href{https://youtu.be/KXJNoC6CuYE}{fun} is the \href{https://youtu.be/_f-qkGJBPts}{best}
      way to \href{https://youtu.be/p60rN9JEapg}{learn}.}
  \item\exFR{Des études ont montré que s'amuser est le meilleur
      moyen pour apprendre.}
  \end{itemize}
  
\item \exEN{\href{http://www.wordreference.com/enfr/luck}{luck}} qui s'écrit
  phonétiquement
  \href{https://en.oxforddictionaries.com/definition/luck}{\phonm{lʌk}}
  
  \begin{itemize}
  \item\exEN{They wish you good \href{https://youtu.be/LQCY2zL0Jr8}{luck} in your \href{https://youtu.be/o61dD6hwrdM}{studies}.}
  \item\exFR{Ils vous souhaietent bonne chance pour votre
      apprentissage.}
  \end{itemize}
  
\end{enumerate}
\newpage

\section{Le \son \phon{ɑː} qui devrait plutôt être noté \phon{aː}}\label{sec:sonalong}

\notation

\hypertarget{sonalong}{Ce \son} a pour nom technique\dyse{open-back-unrounded-vowel} :

\begin{itemize}
\item \exEN{Open Back Unrounded Vowel\CW{https://en.wikipedia.org/wiki/Open_back_unrounded_vowel}.}
\item \exFR{Voyelle arrière non arrondie\CW{https://fr.wikipedia.org/wiki/Voyelle_ouverte_post\%C3\%A9rieure_non_arrondie}.}
\end{itemize}

\begin{center}
  \begin{figure}[h]
    \centering
    \includegraphics{../img/non-arrondies-wikipedia.png}
    \caption{Voyelles non-arrondies, source : Wikipédia}
    \label{fig:voy-ferm}
  \end{figure}
\end{center}

\indicsound

\properukus{https://youtu.be/1F47WdIjn5U}{https://youtu.be/R5CY1UniS68}


\begin{enumerate}
\item \exEN{\href{http://www.wordreference.com/enfr/father}{father}} qui
  s'écrit phonétiquement
  \href{https://en.oxforddictionaries.com/definition/father}{\phonm{ˈfɑːðə}}
  
  \begin{itemize}
  \item\exEN{My \href{https://youtu.be/MZDAUbeSwNY}{father} used to
      tell me that you never \href{https://youtu.be/gYSZjGeK5VE}{waste
        your time} when you are thinking.}
  \item\exFR{Mon père avait l'habitude de me dire qu'on ne perd
      jamais son temps à réfléchir.}
  \end{itemize}
  
\item \exEN{\href{http://www.wordreference.com/enfr/arm}{arm}} qui s'écrit
  phonétiquement
  \href{https://en.oxforddictionaries.com/definition/arm}{\phonm{ɑːm}}

  \begin{itemize}
  \item\exEN{We are \href{https://youtu.be/H75AFiLGd8Y}{lucky} because we have two \href{https://youtu.be/tlhQghmuMf8}{arms} and two legs;
      less lucky if one of them is harmed.}
  \item\exFR{Nous avons la chance d'avoir deux bras et deux
      jambes; désolé si l'un d'eux est blessé.}
  \end{itemize}
  
\item \exEN{\href{http://www.wordreference.com/enfr/dance}{dance}} qui
  s'écrit phonétiquement
  \href{https://en.oxforddictionaries.com/definition/dance}{\phonm{dɑːns}}
  
  \begin{itemize}
  \item\exEN{Would you like to \href{https://youtu.be/VNgoPAq5UjU}{dance} with me \href{https://youtu.be/0zQHNygI_ko}{pretty lady}?}
  \item\exFR{Veux-tu danser avec moi jolie demoiselle ?}
  \end{itemize}
  
\item \exEN{\href{http://www.wordreference.com/enfr/half}{half}} qui s'écrit
  phonétiquement
  \href{https://en.oxforddictionaries.com/definition/half}{\phonm{hɑːf}}
  
  \begin{itemize}
  \item\exEN{\href{https://youtu.be/XWamnSNgiCM}{Half} time! This is the \href{https://youtu.be/2I2kKXWjwhM}{right time} to get some drinks!}
  \item\exFR{Mi-temps ! C'est le bon moment pour prendre à boire !}
  \end{itemize}
  
\end{enumerate}


\begin{center}
  \begin{figure}[h]
    \centering
    \includegraphics{../img/cpp/prim-card-vow-cpp}
    \caption{8 Voyelles primaires cardinales, source :~\cite{collins}}
    \label{fig:voy-prim-card}
  \end{figure}
\end{center}

\newpage
\minitoc
\newpage

