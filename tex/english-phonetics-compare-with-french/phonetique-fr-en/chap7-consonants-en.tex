\part{Consonant sounds}\label{chap:conson}


Un \textcolor{red}{son} de \textcolor{red}{consonne}\CW{https://en.wikipedia.org/wiki/English_phonology\#Consonants} est fait en bloquant l'air quand il quitte la
bouche. Nous utilisons la langue, les lèvres, les dents, la bouche et
la gorge pour bloquer l'air. Certains sons utilisent la voix
(exprimé), d'autres seulement l'air (sans voix).


\chapter{Plosive (un bloc d'air complet)}\label{chap:plosive}

\speech{7}{consonnes \exFR{occlusives\CW{https://fr.wikipedia.org/wiki/Consonne_occlusive}} (\exEN{plosive\CW{https://en.wikipedia.org/wiki/Stop_consonant}})}

\newpage
\minitoc
\newpage

\section{\son{p} }\label{sec:p}

Ce \textcolor{red}{son} a pour nom
technique\dyse{voiceless-bilabial-stop-p} :

\begin{itemize}
\item \exEN{Voiceless Bilabial Stop\CW{https://en.wikipedia.org/wiki/Voiceless_bilabial_stop}.}
\item \exFR{Consonne occlusive bilabiale sourde\CW{https://fr.wikipedia.org/wiki/Consonne_occlusive_bilabiale_sourde}.}
\end{itemize}

\indicsound

\properukus{https://youtu.be/W3bI_PE_kNc}{https://youtu.be/V_n_rUKQSew}


\begin{enumerate}
\item \exEN{\href{http://www.wordreference.com/enfr/pause}{pause}} qui
  s'écrit phonétiquement
  \href{https://en.oxforddictionaries.com/definition/pause}{\exPH{pɔːz}}
  
  \begin{itemize}
  \item\exEN{\href{https://youtu.be/LNHBMFCzznE}{After} a brief \href{https://youtu.be/v\_UlZ0Y9Vho}{pause}, I continued.}
  \item\exFR{Après une courte pause, j'ai recommencé.}
  \end{itemize}

  \youglish{pause}
  
\item \exEN{\href{http://www.wordreference.com/enfr/pin}{pin}} qui s'écrit phonétiquement \href{https://en.oxforddictionaries.com/definition/pin}{\exPH{pɪn}}

  \begin{itemize}
  \item\exEN{She wore a diamond \href{https://youtu.be/DMoeYWQmRuQ}{pin} on her evening \href{https://youtu.be/qKNj5zG3NjA}{dress}.}
  \item\exFR{Elle portait une broche en diamants sur sa robe du soir.}
  \end{itemize}

  \youglish{pin}
  
\item \exEN{\href{http://www.wordreference.com/enfr/purpose}{purpose}} qui s'écrit phonétiquement \href{https://en.oxforddictionaries.com/definition/purpose}{\exPH{ˈpəːpəs}}

  \begin{itemize}
  \item\exEN{The \href{https://youtu.be/J8yhsbMULsQ}{purpose} of the game is to score \href{https://youtu.be/_vroLokIrCo}{points}.}
  \item\exFR{Le but du jeu consiste à marquer des points.}
  \end{itemize}

  \youglish{purpose}

\item \exEN{\href{http://www.wordreference.com/enfr/cap}{cap}} qui s'écrit
  phonétiquement
  \href{https://en.oxforddictionaries.com/definition/cap}{\exPH{kap}}.
  
  \begin{itemize}
  \item\exEN{The \href{https://youtu.be/ROXqddQYBH4}{baseball} player is wearing a \href{https://youtu.be/Dkzh8b5Mj3s}{cap} on his head.}
  \item\exFR{Le joueur de base-ball porte une casquette sur la tête.}
  \end{itemize}

  \youglish{cap}

\end{enumerate}

\newpage

\section{\son{b} }\label{sec:b}

Ce \textcolor{red}{son} a pour nom
technique\dyse{voiced-bilabial-stop-b} :

\begin{itemize}
\item \exEN{Voiced Bilabial Stop\CW{https://en.wikipedia.org/wiki/Voiced_bilabial_stop}.}
\item \exFR{Consonne occlusive bilabiale voisée\CW{https://fr.wikipedia.org/wiki/Consonne_occlusive_bilabiale_vois\%C3\%A9e}.}
\end{itemize}

\indicsound

\properukus{https://youtu.be/MPdBni6svgQ}{https://youtu.be/LbCOXRz7Uf8}


\begin{enumerate}
\item \exEN{\href{http://www.wordreference.com/enfr/bag}{bag}} qui s'écrit
  phonétiquement
  \href{https://en.oxforddictionaries.com/definition/bag}{\exPH{baɡ}}
  
  \begin{itemize}
  \item\exEN{I \href{https://youtu.be/TY4uxdAt4-M}{put} the fruit in a \href{https://youtu.be/DLQhGy7BIjQ}{bag}.}
  \item\exFR{J'ai mis les fruits dans un sac.}
  \end{itemize}

  \youglish{bag}
  
\item \exEN{\href{http://www.wordreference.com/enfr/bubble}{bubble}}
  qui s'écrit phonétiquement
  \href{https://en.oxforddictionaries.com/definition/bubble}{\exPH{ˈbʌb(ə)l}}
  
  \begin{itemize}
  \item\exEN{The \href{https://youtu.be/FE14vq6CDJk}{hot} soup was \href{https://youtu.be/h6OhzwkBAEc}{bubbling} in the saucepan.}
  \item\exFR{La soupe chaude bouillonnait dans la casserole.}
  \end{itemize}

  \youglish{bubble}
  
\item \exEN{\href{http://www.wordreference.com/enfr/build}{build}} qui
  s'écrit phonétiquement
  \href{https://en.oxforddictionaries.com/definition/build}{\exPH{bɪld}}
  
  \begin{itemize}
  \item\exEN{I \href{https://youtu.be/OVcbTvoh0L4}{bezeled} some wooden rods to \href{https://youtu.be/UC4eCvuxjIU}{build} a picture frame.}
  \item\exFR{J'ai taillé en biseau des baguettes de bois pour fabriquer un cadre.}
  \end{itemize}

  \youglish{build}
  
\item \exEN{\href{http://www.wordreference.com/enfr/robe}{robe}} qui s'écrit
  phonétiquement
  \href{https://en.oxforddictionaries.com/definition/robe}{\exPH{rəʊb}}
  
  \begin{itemize}
  \item\exEN{The \href{https://youtu.be/fFaWJLc18xU}{judge} entered the court room wearing her \href{https://youtu.be/eiObDiqVcPk}{robe}.}
  \item\exFR{La juge a fait \textcolor{red}{son} entrée dans le tribunal portant sa robe.}
  \end{itemize}

  \youglish{robe}
  
\end{enumerate}

\newpage

\section{\son{t} }\label{sec:t}

  Ce \textcolor{red}{son} a pour nom technique\dyse{voiceless-alveolar-stop-t} :% #1: lien
                                % vers le blog
  %
  \begin{itemize}%
  \item \exEN{Voicless Alveolar Stop\CW{https://en.wikipedia.org/wiki/Voiceless_dental_and_alveolar_stops}.}% #2: sound name, #3: wiki EN
  \item \exFR{Consonne alvéolaire sourde\CW{voiceless-alveolar-stop-t}.}% #4: nom du son, #5: wiki FR sinon blog voir
                         % package ifthen pour gérer ça
  \end{itemize}%
  %
  \indicsound%
  %
  \properukus{https://youtu.be/8gP9ygo2988}{https://youtu.be/mLlotV_0dRI}% #6: UK YT, #7: US YT


\begin{enumerate}
\item \exEN{\href{http://www.wordreference.com/enfr/time}{time}} qui s'écrit phonétiquement \href{https://en.oxforddictionaries.com/definition/time}{tʌɪm/}

  \begin{itemize}
  \item\exEN{It takes \href{https://youtu.be/A7pI96Osp9c}{time} to get a good level in \href{https://youtu.be/H3r9bOkYW9s}{English}.}
  \item\exFR{Il faut du temps pour obtenir un bon niveau en anglais.}
  \end{itemize}

  \youglish{time}

\item \exEN{\href{http://www.wordreference.com/enfr/tow}{tow}} qui s'écrit
  phonétiquement
  \href{https://en.oxforddictionaries.com/definition/tow}{\exPH{təʊ}}
  
  \begin{itemize}
  \item\exEN{\href{https://youtu.be/h66Lvzlj1Uk}{Pleasure} craft are not permitted to \href{https://youtu.be/tGIx9uoJh9M}{tow} small personal
      boats or dinghies while transiting Canadian locks.}
  \item\exFR{Les embarcations de plaisance ne sont pas autorisées
      à remorquer de petits bateaux ou dériveurs personnels pendant
      le transit des écluses canadiennes.}
  \end{itemize}

  \youglish{tow}
    
\item \exEN{\href{http://www.wordreference.com/enfr/train}{train}} qui
  s'écrit phonétiquement
  \href{https://en.oxforddictionaries.com/definition/train}{\exPH{treɪn}}
  
  \begin{itemize}
  \item\exEN{I'm sorry, I \href{https://youtu.be/5EXC_rjs7tg}{missed} my \href{https://youtu.be/jgxKrH-O2Kk}{train} this morning.}
  \item\exFR{Je suis désolé, j'ai loupé mon train ce matin.}
  \end{itemize}

  \youglish{train}

\item \exEN{\href{http://www.wordreference.com/enfr/late}{late}} qui s'écrit
  phonétiquement
  \href{https://en.oxforddictionaries.com/definition/late}{\exPH{leɪt}}
  
  \begin{itemize}
  \item\exEN{It is very \href{https://youtu.be/YCGty3CBexc}{likely} that I will be \href{https://youtu.be/v\_HNcDj7-Kw}{late}.}
  \item\exFR{Il est très probable que j'arrive en retard.}
  \end{itemize}

  \youglish{late}
    
  \end{enumerate}

  \newpage
  
\section{\son{d}}\label{sec:d}

  Ce \textcolor{red}{son} a pour nom technique\dyse{voiced-alveolar-stop-d} :% #1: lien
                                % vers le blog
  %
  \begin{itemize}%
  \item \exEN{Voiced Alveolar Stop\CW{https://en.wikipedia.org/wiki/Voiced_dental_and_alveolar_stops}.}% #2: sound name, #3: wiki EN
  \item \exFR{Consonne alvéolaire voisée\CW{voiced-alveolar-stop-d}.}% #4: nom du son, #5: wiki FR sinon blog voir
                         % package ifthen pour gérer ça
  \end{itemize}%
  %
  \indicsound%
  %
  \properukus{https://youtu.be/6R6hh1aKDiE}{https://youtu.be/N73xPe0x79g}% #6: UK YT, #7: US YT

\begin{enumerate}
\item \exEN{\href{http://www.wordreference.com/enfr/day}{day}} qui s'écrit
  phonétiquement
  \href{https://en.oxforddictionaries.com/definition/day}{\exPH{deɪ}}
  
  \begin{itemize}
  \item\exEN{One \href{https://youtu.be/sl9voSKJmEU}{day} you'll understand that \href{https://youtu.be/S2DnMGnAGNs}{practice makes perfect}.}
  \item\exFR{Un jour tu comprendras que la perfection n'est approchable que par la répétition.}
  \end{itemize}

  \youglish{day}
  
\item \exEN{\href{http://www.wordreference.com/enfr/door}{door}} qui s'écrit
  phonétiquement
  \href{https://en.oxforddictionaries.com/definition/door}{\exPH{dɔː}}
  
  \begin{itemize}
  \item\exEN{The \href{https://youtu.be/eDW\_yAwaHnc}{doors} are opened so you can \href{https://youtu.be/KDXOzr0GoA4}{come} in.}
  \item\exFR{Les portes sont ouvertes donc tu peux entrer.}
  \end{itemize}

  \youglish{door}
  
\item \exEN{\href{http://www.wordreference.com/enfr/down}{down}} qui s'écrit
  phonétiquement
  \href{https://en.oxforddictionaries.com/definition/down}{\exPH{daʊn}}
  
  \begin{itemize}
  \item\exEN{Following the \href{https://youtu.be/K--kIdOpbJM}{storm}, many trees are \href{https://youtu.be/pn4oaQNiNQc}{down}.}
  \item\exFR{Suite à la tempête, de nombreux arbres sont à terre.}
  \end{itemize}

  \youglish{down}
  
\item \exEN{\href{http://www.wordreference.com/enfr/drive}{drive}} qui s'écrit phonétiquement \href{https://en.oxforddictionaries.com/definition/drive}{\exPH{drʌɪv}}

  \begin{itemize}
  \item\exEN{The \href{https://youtu.be/mPBCO17bFms}{drive} to work is \href{https://youtu.be/vDjcWlCT8rg}{short}.}
  \item\exFR{Le trajet jusqu'au travail est court.}
  \end{itemize}

  \youglish{drive}

\end{enumerate}

\newpage

\section{\son{k}}\label{sec:k}

  Ce \textcolor{red}{son} a pour nom technique\dyse{voiceless-velar-stop-k} :% #1: lien
                                % vers le blog
  %
  \begin{itemize}%
  \item \exEN{Voiceless Velar Stop\CW{https://en.wikipedia.org/wiki/Voiceless_velar_stop}.}% #2: sound name, #3: wiki EN
  \item \exFR{Consonne occlusive vélaire sourde\CW{https://fr.wikipedia.org/wiki/Consonne_occlusive_v\%C3\%A9laire_sourde}.}% #4: nom du son, #5: wiki FR sinon blog voir
                         % package ifthen pour gérer ça
  \end{itemize}%
  %
  \indicsound%
  %
  \properukus{https://youtu.be/6YJdNE8na9M}{https://youtu.be/zxrveu6yu6E}% #6: UK YT, #7: US YT

\begin{enumerate}
\item \exEN{\href{http://www.wordreference.com/enfr/cash}{cash}} qui s'écrit
  phonétiquement
  \href{https://en.oxforddictionaries.com/definition/cash}{\exPH{kaʃ}}
  
  \begin{itemize}
  \item\exEN{The \href{https://youtu.be/VJ1OnnkwjrM}{supermarket} only accepts \href{https://youtu.be/ALGi0tcFCcw}{cash}.}
  \item\exFR{Le supermarché n'accepte que les espèces.}
  \end{itemize}

  \youglish{cash}
  
\item \exEN{\href{http://www.wordreference.com/enfr/cricket}{cricket}} qui
  s'écrit phonétiquement
  \href{https://en.oxforddictionaries.com/definition/cricket}{\exPH{ˈkrɪkɪt}}
  
  \begin{itemize}
  \item\exEN{In March, India's \href{https://youtu.be/c5oZPB-grGI}{cricket} team will be visiting
      Pakistan for the first time in a \href{https://youtu.be/tBkxOQodLnE}{decade}.}
  \item\exFR{Au mois de mars, l'équipe de cricket indienne se rendra au Pakistan pour la première fois depuis dix ans.}
  \end{itemize}

  \youglish{cricket}
  
\item \exEN{\href{http://www.wordreference.com/enfr/quick}{quick}} qui
  s'écrit phonétiquement
  \href{https://en.oxforddictionaries.com/definition/quick}{\exPH{kwɪk}}
  
  \begin{itemize}
  \item\exEN{We would \href{https://youtu.be/4bGTjagyJkQ}{appreciate} a \href{https://youtu.be/OB-YD47ddWI}{quick} reply.}
  \item\exFR{Nous apprécierions une réponse rapide.}
  \end{itemize}

  \youglish{quick}
  
\item \exEN{\href{http://www.wordreference.com/enfr/sock}{sock}} qui s'écrit
  phonétiquement
  \href{https://en.oxforddictionaries.com/definition/sock}{\exPH{sɒk}}
  
  \begin{itemize}
  \item\exEN{I put on \href{https://youtu.be/Eu1fW2BafnM}{socks} before putting on my \href{https://youtu.be/4srOE3pCCo8}{shoes}.}
  \item\exFR{J'ai enfilé des chaussettes avant de mettre mes chaussures.}
  \end{itemize}

  \youglish{sock}
  
\end{enumerate}

\newpage

\section{\son{g}}\label{sec:g}

  Ce \textcolor{red}{son} a pour nom technique\dyse{voiced-velar-stop-g} :% #1: lien
                                % vers le blog
  %
  \begin{itemize}%
  \item \exEN{Voiced Velar Stop\CW{https://en.wikipedia.org/wiki/Voiced_velar_stop}.}% #2: sound name, #3: wiki EN
  \item \exFR{Consonne occlusive vélaire voisée\CW{https://fr.wikipedia.org/wiki/Consonne_occlusive_v\%C3\%A9laire_vois\%C3\%A9e}.}% #4: nom du son, #5: wiki FR sinon blog voir
                         % package ifthen pour gérer ça
  \end{itemize}%
  %
  \indicsound%
  %
  \properukus{https://youtu.be/8H_Xis2wigA}{https://youtu.be/vP5XKYvxe0Q}% #6: UK YT, #7: US YT
  
\begin{enumerate}
\item \exEN{\href{http://www.wordreference.com/enfr/girl}{girl}} qui s'écrit phonétiquement \href{https://en.oxforddictionaries.com/definition/girl}{\exPH{ɡəːl}}

  \begin{itemize}
  \item\exEN{\href{https://en.wikipedia.org/wiki/In\_the\_Pines}{My}
      \href{https://youtu.be/bpFuH8vcXbw}{girl},
      \href{https://genius.com/Nirvana-where-did-you-sleep-last-night-lyrics}{my}
      \href{https://youtu.be/PsfcUZBMSSg}{girl},
      \href{https://fr.wikipedia.org/wiki/Where\_Did\_You\_Sleep\_Last\_Night}{don't
        lie} to me.}
  \end{itemize}

  \youglish{girl}

\item \exEN{\href{http://www.wordreference.com/enfr/green}{green}} qui s'écrit phonétiquement \href{https://en.oxforddictionaries.com/definition/green}{ɡriːn/}. Exemple d'utilisation du mot :

  \begin{itemize}
  \item\exEN{The \href{https://youtu.be/Ej-qVhL1a-Q}{mayor} launched a \href{https://youtu.be/a1BS7XnEZqc}{green} initiative to plant more
      trees.}
  \item\exFR{Le maire a lancé une initiative écologique pour planter davantage d'arbres.}
  \end{itemize}

  \youglish{green}

\item \exEN{\href{http://www.wordreference.com/enfr/grass}{grass}} qui s'écrit phonétiquement \href{https://en.oxforddictionaries.com/definition/grass}{\exPH{ɡrɑːs}}

  \begin{itemize}
  \item\exEN{\href{https://youtu.be/6NzSxKpyS2I}{Cows} feed on fresh \href{https://youtu.be/QsfJscoMx5M}{grass}.}
  \item\exFR{Les vaches se nourrissent d'herbe fraîche.}
  \end{itemize}

  \youglish{grass}
  
\item \exEN{\href{http://www.wordreference.com/enfr/flag}{flag}} qui s'écrit
  phonétiquement
  \href{https://en.oxforddictionaries.com/definition/flag}{\exPH{flaɡ}}
  
  \begin{itemize}
  \item\exEN{The \href{https://youtu.be/fiyYoe678yI}{vessel} flies the British \href{https://youtu.be/EBl2PVjVNqA}{flag}.}
  \item\exFR{Le navire bat pavillon britannique.}
  \end{itemize}

  \youglish{flag}
  
\end{enumerate}

\newpage

\section{\son{?}  \href{https://en.wikipedia.org/wiki/Glottal\_stop}{glottal}
  }\label{sec:glottal}

  Ce \textcolor{red}{son} a pour nom technique\dyse{glottal-stop} :% #1: lien
                                % vers le blog
  %
  \begin{itemize}%
  \item \exEN{Glottal Stop\CW{https://en.wikipedia.org/wiki/Glottal_stop}.}% #2: sound name, #3: wiki EN
  \item \exFR{Coup de glotte \CW{https://fr.wikipedia.org/wiki/Coup_de_glotte}.}% #4: nom du son, #5: wiki FR sinon blog voir
                         % package ifthen pour gérer ça
  \end{itemize}%
  %
  \indicsound%
  %
  \properukus{https://youtu.be/A11_co6jJsI}{https://youtu.be/Vabg-EUHOQk}% #6: UK YT, #7: US YT
  
\begin{enumerate}
\item \exEN{\href{http://www.wordreference.com/enfr/football}{football}} qui
  s'écrit phonétiquement
  \href{https://www.phon.ucl.ac.uk/home/wells/phoneticsymbolsforenglish.htm}{\exPH{ˈfʊ?bɔːl}}
  
  \begin{itemize}
  \item\exEN{This summer the \href{https://youtu.be/6v5Ao0tYhBw}{football} \href{https://youtu.be/zVr3dTMY9Ag}{world cup} will be in Russia
      and twenty four years ago it was in \href{https://youtu.be/mAYvjOzh1ag}{America}.}
  \item\exFR{Cet été la coupe du monde de football sera en Russie
      et il y a vingt-quatre ans c'était en Amérique.}
  \end{itemize}

  \youglish{football}
  
\item \href{http://www.wordreference.com/enfr/department}{department}
  qui s'écrit phonétiquement
  \href{https://en.oxforddictionaries.com/definition/department}{\exPH{dɪˈpɑː?m(ə)nt}}
  
  \begin{itemize}
  \item\exEN{\href{https://youtu.be/dDxOyf-LLhw}{Guadeloupe} is an overseas \href{https://youtu.be/0CUWPGLVRoU}{department} of France.}
  \item\exFR{La Guadeloupe est un département d'outre-mer de la France.}
  \end{itemize}

  \youglish{department}

\item \exEN{\href{http://www.wordreference.com/enfr/apartment}{apartment}}
  qui s'écrit phonétiquement
  \href{https://tophonetics.com/}{\exPH{əˈpɑː?mənt}}
  
  \begin{itemize}
  \item\exEN{My \href{https://youtu.be/H0HjU9956Z8}{apartment} is not in your \href{https://youtu.be/XQ4h1sDpKno}{department}.}
  \item\exFR{Mon appartement n'est pas dans votre département.}
  \end{itemize}

  \youglish{apartment}

\item \exEN{\href{http://www.wordreference.com/enfr/button}{button}} qui
  s'écrit phonétiquement
  \href{https://en.wikipedia.org/wiki/Glottal\_stop}{\exPH{ˈbɐʔn̩n}}
  
  \begin{itemize}
  \item\exEN{Click the \href{https://youtu.be/IJcwc5Gz8K0}{button} to subscribe.}
  \item\exFR{Cliquez sur le bouton pour vous abonner.}
  \end{itemize}

  \youglish{button}
  
\end{enumerate}

\newpage
\minitoc
\newpage

\chapter{Fricative (une compression constante de
  l'air)}\label{chap:fricative}

\speech{9}{consonnes \exFR{fricatives\CW{https://fr.wikipedia.org/wiki/Consonne_fricative}} (\exEN{fricative\CW{https://en.wikipedia.org/wiki/Fricative_consonant}})}

\newpage
\minitoc
\newpage

\section{\son{f}}\label{sec:f}

  Ce \textcolor{red}{son} a pour nom technique\dyse{voiceless-labiodental-fricative-f} :% #1: lien
                                % vers le blog
  %
  \begin{itemize}%
  \item \exEN{Voiceless labiodental fricative\CW{https://en.wikipedia.org/wiki/Voiceless_labiodental_fricative}.}% #2: sound name, #3: wiki EN
  \item \exFR{Consonne fricative labio-dentale sourde \CW{https://fr.wikipedia.org/wiki/Consonne_fricative_labio-dentale_sourde}.}% #4: nom du son, #5: wiki FR sinon blog voir
                         % package ifthen pour gérer ça
  \end{itemize}%
  %
  \indicsound%
  %
  \properukus{https://youtu.be/4VcU0zNJUiU}{https://youtu.be/YejZ8gAQAfU}% #6: UK YT, #7: US YT

\begin{enumerate}
\item \exEN{\href{http://www.wordreference.com/enfr/fish}{fish}} qui s'écrit
  phonétiquement
  \href{https://en.oxforddictionaries.com/definition/fish}{\exPH{fɪʃ}}
  
  \begin{itemize}
  \item\exEN{He prefers \href{https://youtu.be/rEm4ynLtGx4}{fish} to \href{https://youtu.be/LnA7Au-DLUM}{meat}.}
  \item\exFR{Il préfère le poisson à la viande.}
  \end{itemize}

  \youglish{fish}

\item \exEN{\href{http://www.wordreference.com/enfr/friday}{friday}} qui
  s'écrit phonétiquement
  \href{https://en.oxforddictionaries.com/definition/friday}{\exPH{ˈfrʌɪdi}}
  
  \begin{itemize}
  \item\exEN{The \href{https://youtu.be/DFeDMHIYtlo}{ship} sailed from the port on \href{https://youtu.be/lQ\_pgrjjHLo}{Friday}.}
  \item\exFR{Le bateau a quitté le port vendredi.}
  \end{itemize}

  \youglish{friday}
  
\item \exEN{\href{http://www.wordreference.com/enfr/full}{full}} qui s'écrit
  phonétiquement
  \href{https://en.oxforddictionaries.com/definition/full}{\exPH{fʊl}}
  
  \begin{itemize}
  \item\exEN{The \href{https://youtu.be/LR73DrKX\_bs}{full} report is hundreds of pages \href{https://youtu.be/CwfoyVa980U}{long}.}
  \item\exFR{Le rapport complet fait des centaines de pages.}
  \end{itemize}

  \youglish{full}

\item \exEN{\href{http://www.wordreference.com/enfr/knife}{knife}} qui s'écrit phonétiquement \href{https://en.oxforddictionaries.com/definition/knife}{\exPH{nʌɪf}}

  \begin{itemize}
  \item\exEN{The blunt \href{https://youtu.be/JUyzH9HpkqE}{knife} could not cut the \href{https://youtu.be/LYoAScygp1w}{rope}.}
  \item\exFR{Le couteau émoussé ne pouvait pas couper la corde.}
  \end{itemize}

  \youglish{knife}

\end{enumerate}

\newpage

\section{\son{v}}\label{sec:v}

  Ce \textcolor{red}{son} a pour nom technique\dyse{voiced-labiodental-fricative-v} :% #1: lien
                                % vers le blog
  %
  \begin{itemize}%
  \item \exEN{Voiced labiodental fricative\CW{https://en.wikipedia.org/wiki/Voiced_labiodental_fricative}.}% #2: sound name, #3: wiki EN
  \item \exFR{Consonne fricative labio-dentale voisée \CW{https://fr.wikipedia.org/wiki/Consonne_fricative_labio-dentale_vois\%C3\%A9e}.}% #4: nom du son, #5: wiki FR sinon blog voir
                         % package ifthen pour gérer ça
  \end{itemize}%
  %
  \indicsound%
  %
  \properukus{https://youtu.be/0W5PcWptfYY}{https://youtu.be/nR-K3mrHFv0}% #6: UK YT, #7: US YT
  
\begin{enumerate}
\item \exEN{\href{http://www.wordreference.com/enfr/cave}{cave}} qui s'écrit
  phonétiquement
  \href{https://en.oxforddictionaries.com/definition/cave}{\exPH{keɪv}}
  
  \begin{itemize}
  \item\exEN{\href{https://youtu.be/CqGsg01ycpk}{Plato} is famous for his myth of the \href{https://youtu.be/kZQbkzwwinI}{cave}.}
  \item\exFR{Platon est célèbre pour \textcolor{red}{son} mythe de la caverne.}
  \end{itemize}

  \youglish{cave}
  
\item \exEN{\href{http://www.wordreference.com/enfr/vest}{vest}} qui s'écrit
  phonétiquement
  \href{https://en.oxforddictionaries.com/definition/vest}{\exPH{vɛst}}
  
  \begin{itemize}
  \item\exEN{The \href{https://youtu.be/QNMD29BQ9J4}{committee} was \href{https://youtu.be/E4cjvxydHuU}{vested} with the government's full
      authority.}
  \item\exFR{Le comité était investi de toute l'autorité du gouvernement.}
  \end{itemize}

  \youglish{vest}
  
\item \exEN{\href{http://www.wordreference.com/enfr/view}{view}} qui s'écrit
  phonétiquement
  \href{https://en.oxforddictionaries.com/definition/view}{\exPH{vjuː}}
  
  \begin{itemize}
  \item\exEN{There is a splendid \href{https://youtu.be/gODvA\_SdXCY}{view} from the \href{https://youtu.be/QEO-2jYvkto}{balcony}.}
  \item\exFR{Il y a une vue splendide depuis le balcon.}
  \end{itemize}

  \youglish{view}
  
\item \exEN{\href{http://www.wordreference.com/enfr/village}{village}} qui
  s'écrit phonétiquement
  \href{https://en.oxforddictionaries.com/definition/village}{\exPH{ˈvɪlɪdʒ}}
  
  \begin{itemize}
  \item\exEN{The \href{https://youtu.be/Xq8mt6WuD-E}{village} is peaceful at \href{https://youtu.be/nNa6F_fEtMw}{night}.}
  \item\exFR{Le village est tranquille la nuit.}
  \end{itemize}

  \youglish{village}
  
\end{enumerate}

\newpage

\section{\son{θ}}\label{sec:theta}

Ce \textcolor{red}{son} a pour nom technique\dyse{voiceless-dental-fricative} :% #1: lien
                                % vers le blog
  %
  \begin{itemize}%
  \item \exEN{Voiceless dental fricative\CW{https://en.wikipedia.org/wiki/Voiceless_dental_fricative}.}% #2: sound name, #3: wiki EN
  \item \exFR{Consonne fricative dentale sourde \CW{https://fr.wikipedia.org/wiki/Consonne_fricative_dentale_sourde}.}% #4: nom du son, #5: wiki FR sinon blog voir
                         % package ifthen pour gérer ça
  \end{itemize}%
  %
  \indicsound%
  %
  \properukus{https://youtu.be/Fq1atdudgh8}{https://youtu.be/qC0l6GQZtM4}% #6: UK YT, #7: US YT
  
\begin{enumerate}
\item \exEN{\href{http://www.wordreference.com/enfr/author}{author}} qui
  s'écrit phonétiquement
  \href{https://en.oxforddictionaries.com/definition/author}{\exPH{ˈɔːθə}}
  
  \begin{itemize}
  \item\exEN{I am the \href{https://youtu.be/lyGivD8aJi4}{author} of this \href{https://youtu.be/iUX18JVWNf8}{document}.}
  \item\exFR{Je suis l'auteur de ce document.}
  \end{itemize}

  \youglish{author}
  
\item \exEN{\href{http://www.wordreference.com/enfr/path}{path}} qui s'écrit
  phonétiquement
  \href{https://en.oxforddictionaries.com/definition/path}{\exPH{pɑːθ}}
  
  \begin{itemize}
  \item\exEN{A \href{https://youtu.be/9seb8hddeK4}{fork} in the road splits it into two \href{https://youtu.be/EZdFE-nnyyQ}{paths}.}
  \item\exFR{Un embranchement sur la route la divise en deux sentiers.}
  \end{itemize}

  \youglish{path}
  
\item \exEN{\href{http://www.wordreference.com/enfr/thing}{thing}} qui
  s'écrit phonétiquement
  \href{https://en.oxforddictionaries.com/definition/thing}{\exPH{θɪŋ}}
  
  \begin{itemize}
  \item\exEN{\href{https://youtu.be/tOd6l2g4XTw}{Windsurfing} is not really my \href{https://youtu.be/h-pmsrw8XNE}{thing}; I prefer surfing.}
  \item\exFR{La planche à voile n'est pas vraiment mon truc ; je
      préfère surfer.}
  \end{itemize}

  \youglish{thing}
  
\item \exEN{\href{http://www.wordreference.com/enfr/think}{think}} qui s'écrit phonétiquement \href{https://en.oxforddictionaries.com/definition/think}{\exPH{θɪŋk}}

  \begin{itemize}
  \item\exEN{I \href{https://youtu.be/PpD8OvMTRiE}{think} my solution is the \href{https://youtu.be/UfMnEEystJ0}{best}.}
  \item\exFR{Je considère que ma solution est la meilleure.}
  \end{itemize}

  \youglish{think}

\end{enumerate}

\newpage

\section{\son{ð}}\label{sec:th}

Ce \textcolor{red}{son} a pour nom technique\dyse{voiced-dental-fricative} :% #1: lien
                                % vers le blog
  %
  \begin{itemize}%
  \item \exEN{Voiced dental fricative\CW{https://en.wikipedia.org/wiki/Voiced_dental_fricative}.}% #2: sound name, #3: wiki EN
  \item \exFR{Consonne fricative dentale voisée \CW{https://fr.wikipedia.org/wiki/Consonne_fricative_dentale_vois\%C3\%A9e}.}% #4: nom du son, #5: wiki FR sinon blog voir
                         % package ifthen pour gérer ça
  \end{itemize}%
  %
  \indicsound%
  %
  \properukus{https://youtu.be/GdtdTJkRtkE}{https://youtu.be/EZb_EWVCUoE}% #6: UK YT, #7: US YT

\begin{enumerate}
\item \exEN{\href{http://www.wordreference.com/enfr/this}{this}} qui s'écrit phonétiquement \href{https://en.oxforddictionaries.com/definition/this}{\exPH{ðɪs}}

  \begin{itemize}
  \item\exEN{The \href{https://youtu.be/XGAvSsjVA8U}{implementation} of \href{https://youtu.be/KqzlYTmFBGY}{this} principle will, as a
      consequence, generate more data than currently available.}
  \item\exFR{L'application de ce principe va donc générer plus de
      données que ce qui est actuellement disponible.}
  \end{itemize}

  \youglish{this}
  
\item \exEN{\href{http://www.wordreference.com/enfr/other}{other}} qui
  s'écrit phonétiquement
  \href{https://en.oxforddictionaries.com/definition/other}{\exPH{ˈʌðə}}
  
  \begin{itemize}
  \item\exEN{The \href{https://youtu.be/ZhfWiU8wGCc}{woman} was selling apples and \href{https://youtu.be/9gXP8wcICqQ}{other} fruits.}
  \item\exFR{La femme vendait des pommes et d'autres fruits.}
  \end{itemize}

  \youglish{other}
  
\item \exEN{\href{http://www.wordreference.com/enfr/breathe}{breathe}} qui s'écrit phonétiquement \href{https://en.oxforddictionaries.com/definition/breathe}{\exPH{briːð}}

  \begin{itemize}
  \item\exEN{The \href{https://youtu.be/slC-emKLVBs}{air} we \href{https://youtu.be/V8rtJRlLdI8}{breathe} is invisible.}
  \item\exFR{L'air que nous respirons est invisible.}
  \end{itemize}

  \youglish{breathe}
  
\item \exEN{\href{http://www.wordreference.com/enfr/bathe}{bathe}} qui s'écrit phonétiquement \href{https://dictionary.cambridge.org/dictionary/english/bathe}{\exPH{beɪð}}

  \begin{itemize}
  \item\exEN{Can I \href{https://youtu.be/U9V8cx2buG0}{bathe} my baby from the first hours of their
      \href{https://youtu.be/aVvR8ABCPvU}{life}?}
  \item\exFR{Puis-je baigner mon bébé dès ses premières heures de vie~?}
  \end{itemize}

  \youglish{bathe}
  
\end{enumerate}

\newpage

\section{\son{s}}\label{sec:s}

Ce \textcolor{red}{son} a pour nom technique\dyse{voiceless-alveolar-sibilant-s} :% #1: lien
                                % vers le blog
  %
  \begin{itemize}%
  \item \exEN{Voiceless alveolar sibilants\CW{https://en.wikipedia.org/wiki/Voiceless_alveolar_fricative\#Voiceless_alveola}.}% #2: sound name, #3: wiki EN
  \item \exFR{Consonne fricative alvéolaire sourde \CW{https://fr.wikipedia.org/wiki/Consonne_fricative_alv\%C3\%A9olaire_sourde}.}% #4: nom du son, #5: wiki FR sinon blog voir
                         % package ifthen pour gérer ça
  \end{itemize}%
  %
  \indicsound%
  %
  \properukus{https://youtu.be/L3vyZaQF8vk}{https://youtu.be/xl-7mSeybmI}% #6: UK YT, #7: US YT
  
\begin{enumerate}
\item \exEN{\href{http://www.wordreference.com/enfr/kiss}{kiss}} qui s'écrit phonétiquement \href{https://en.oxforddictionaries.com/definition/kiss}{\exPH{kɪs}}

  \begin{itemize}
  \item\exEN{The princess \href{https://youtu.be/vMbVzr7WqIo}{kissed} the \href{https://youtu.be/qh7EY3geI0M}{frog}.}
  \item\exFR{La princesse a embrassé la grenouille.}
  \end{itemize}

  \youglish{kiss}
  
\item \exEN{\href{http://www.wordreference.com/enfr/cease}{cease}} qui s'écrit phonétiquement \href{https://en.oxforddictionaries.com/definition/cease}{\exPH{siːs}}

  \begin{itemize}
  \item\exEN{My \href{https://youtu.be/6d-rkoW4COE}{wife} never \href{https://youtu.be/6m9bEMejTKI}{ceases} to amaze me.}
  \item\exFR{Ma femme ne cesse de m'étonner.}
  \end{itemize}

  \youglish{cease}

\item \exEN{\href{http://www.wordreference.com/enfr/sister}{sister}} qui s'écrit phonétiquement \href{https://en.oxforddictionaries.com/definition/sister}{\exPH{ˈsɪstə}}

  \begin{itemize}
  \item\exEN{Many \href{https://youtu.be/SBNB13EeRx4}{sisters} live in the \href{https://youtu.be/U90ATbs49jc}{convent}.}
  \item\exFR{De nombreuses religieuses vivent dans le couvent.}
  \end{itemize}

  \youglish{sister}
  
\item \exEN{\href{http://www.wordreference.com/enfr/sight}{sight}} qui s'écrit phonétiquement \href{https://en.oxforddictionaries.com/definition/sight}{\exPH{sʌɪt}} 

  \begin{itemize}
  \item\exEN{I \href{https://youtu.be/Mo_mgcxGYYE}{witnessed} a strange \href{https://youtu.be/JeiVf30VDDU}{sight} in the street.}
  \item\exFR{J'ai été témoin d'une scène étrange dans la rue.}
  \end{itemize}

  \youglish{sight}
  
\end{enumerate}

\newpage

\section{\son{z}}\label{sec:z}

Ce \textcolor{red}{son} a pour nom technique\dyse{voiced-alveolar-sibilant-z} :% #1: lien
                                % vers le blog
  %
  \begin{itemize}%
  \item \exEN{Voiced alveolar sibilant \CW{https://en.wikipedia.org/wiki/Voiced_alveolar_fricative\#Voiced_alveolar_sibilant}.}% #2: sound name, #3: wiki EN
  \item \exFR{Consonne fricative alvéolaire voisée \CW{https://fr.wikipedia.org/wiki/Consonne_fricative_alv\%C3\%A9olaire_vois\%C3\%A9e}.}% #4: nom du son, #5: wiki FR sinon blog voir
                         % package ifthen pour gérer ça
  \end{itemize}%
  %
  \indicsound%
  %
  \properukus{https://youtu.be/7jhEYQI1954}{https://youtu.be/xl-7mSeybmI}% #6: UK YT, #7: US YT
  
\begin{enumerate}
\item \exEN{\href{http://www.wordreference.com/enfr/buzz}{buzz}} qui s'écrit phonétiquement \href{https://en.oxforddictionaries.com/definition/buzz}{\exPH{bʌz}}

  \begin{itemize}
  \item\exEN{The \href{https://youtu.be/hTPXDr9pbak}{news} caused a \href{https://youtu.be/OoQJUNv-Jlg}{buzz} in the audience.}
  \item\exFR{La nouvelle a provoqué l'effervescence du public.}
  \end{itemize}

  \youglish{buzz}
    
\item \exEN{\href{http://www.wordreference.com/enfr/crazy}{crazy}} qui s'écrit phonétiquement \href{https://en.oxforddictionaries.com/definition/crazy}{\exPH{ˈkreɪzi}}

  \begin{itemize}
  \item\exEN{My \href{https://youtu.be/SmXNnizCLyw}{aunt} is \href{https://youtu.be/U0EW0s1fN-8}{crazy} about her cats.}
  \item\exFR{Ma tante est dingue de ses chats.}
  \end{itemize}

  \youglish{crazy}

\item \exEN{\href{http://www.wordreference.com/enfr/lazy}{lazy}} qui
  s'écrit phonétiquement
  \href{https://en.oxforddictionaries.com/definition/lazy}{\exPH{ˈleɪzi}}
  
  \begin{itemize}
  \item\exEN{My \href{https://youtu.be/5ZIR0PJ0eXI}{son} is smart but incredibly \href{https://youtu.be/3ev7GXzFTPg}{lazy}.}
  \item\exFR{Mon fils est intelligent mais extrêmement paresseux.}
  \end{itemize}

  \youglish{lazy}
  
\item \exEN{\href{http://www.wordreference.com/enfr/nose}{nose}} qui s'écrit phonétiquement \href{https://en.oxforddictionaries.com/definition/nose}{\exPH{nəʊz}}

  \begin{itemize}
  \item\exEN{The \href{https://youtu.be/nNA9ru2Ox5o}{tip} of my \href{https://youtu.be/1G-nn-b4TJA}{nose} is cold.}
  \item\exFR{Le bout de mon nez est froid.}
  \end{itemize}

  \youglish{nose}
  
\end{enumerate}

\newpage

\section{\son{ʃ}}\label{chap:ts}

Ce \textcolor{red}{son} a pour nom technique\dyse{voiceless-postalveolar-fricative} :% #1: lien
                                % vers le blog
  %
  \begin{itemize}%
  \item \exEN{Voiceless postalveolar fricative \CW{https://en.wikipedia.org/wiki/Voiceless_postalveolar_fricative}.}% #2: sound name, #3: wiki EN
  \item \exFR{Consonne fricative palato-alvéolaire sourde \CW{https://fr.wikipedia.org/wiki/Consonne_fricative_palato-alv\%C3\%A9olaire_sourde}.}% #4: nom du son, #5: wiki FR sinon blog voir
                         % package ifthen pour gérer ça
  \end{itemize}%
  %
  \indicsound%
  %
  \properukus{https://youtu.be/iDjAf2AhjRc}{https://youtu.be/RxaLKZPPEvY}% #6: UK YT, #7: US YT

\begin{enumerate}
\item \exEN{\href{http://www.wordreference.com/enfr/cash}{cash}} qui s'écrit phonétiquement \href{https://en.oxforddictionaries.com/definition/cash}{\exPH{kaʃ}}

  \begin{itemize}
  \item\exEN{The \href{https://youtu.be/4ahHWROn8M0}{cash} he received for his \href{https://youtu.be/VyXKzKe0QXk}{invention} is a windfall.}
  \item\exFR{L'argent qu'il a reçu pour son invention est une aubaine.}
  \end{itemize}

  \youglish{cash}
    
\item
  \exEN{\href{http://www.wordreference.com/enfr/national}{national}}
  qui s'écrit phonétiquement
  \href{https://en.oxforddictionaries.com/definition/national}{\exPH{ˈnaʃ(ə)n(ə)l}}
  
  \begin{itemize}
  \item\exEN{The country's beautiful \href{https://youtu.be/kWLqbk5vRqo}{landscapes} are a subject of
      \href{https://youtu.be/xZvzCOQ-TPA}{national} pride.}
  \item\exFR{Les beaux paysages du pays sont un objet de fierté nationale.}
  \end{itemize}

  \youglish{national}
    
\item \exEN{\href{http://www.wordreference.com/enfr/crash}{crash}} qui s'écrit phonétiquement \href{https://en.oxforddictionaries.com/definition/crash}{\exPH{kraʃ}}

  \begin{itemize}
  \item\exEN{All \href{https://youtu.be/7BWWWQzTpNU}{passengers} on the plane survived the \href{https://youtu.be/Jw81bRYUzVM}{crash}.}
  \item\exFR{Tous les passagers de l'avion ont survécu à l'accident.}
  \end{itemize}

  \youglish{crash}
    
\item \exEN{\href{http://www.wordreference.com/enfr/ship}{ship}} qui s'écrit phonétiquement \href{https://en.oxforddictionaries.com/definition/ship}{\exPH{ʃɪp}}

  \begin{itemize}
  \item\exEN{The \href{https://youtu.be/o6-s1mlQB5U}{company} mainly \href{https://youtu.be/LLkGsfOfgUw}{ships} parcels to Europe.}
  \item\exFR{L'entreprise expédie principalement des colis vers l'Europe.}
  \end{itemize}

  \youglish{ship}
    
\end{enumerate}

\newpage

\section{\son{ʒ}}\label{chap:dj}

Ce \textcolor{red}{son} a pour nom technique\dyse{voiced-palato-alveolar-fricative} :% #1: lien
                                % vers le blog
  %
  \begin{itemize}%
  \item \exEN{Voiced palato-alveolar fricative \CW{https://en.wikipedia.org/wiki/Voiced_postalveolar_fricative}.}% #2: sound name, #3: wiki EN
  \item \exFR{Consonne fricative palato-alvéolaire voisée \CW{https://fr.wikipedia.org/wiki/Consonne_fricative_palato-alv\%C3\%A9olaire_vois\%C3\%A9e}.}% #4: nom du son, #5: wiki FR sinon blog voir
                         % package ifthen pour gérer ça
  \end{itemize}%
  %
  \indicsound%
  %
  \properukus{https://youtu.be/truPu_ReQ8Y}{https://youtu.be/RxaLKZPPEvY}% #6: UK YT, #7: US YT

\begin{enumerate}
\item \exEN{\href{http://www.wordreference.com/enfr/leisure}{leisure}} qui s'écrit phonétiquement \href{https://en.oxforddictionaries.com/definition/leisure}{\exPH{ˈlɛʒə}}

  \begin{itemize}
  \item\exEN{\href{https://youtu.be/IHkXl7L-lFA}{Everyone} needs moments of \href{https://youtu.be/VSRFE7E4qWI}{leisure} to relax.}
  \item\exFR{Tout le monde a besoin de moments de loisir pour se détendre.}
  \end{itemize}

  \youglish{leisure}
    
\item \exEN{\href{http://www.wordreference.com/enfr/measure}{measure}} qui s'écrit phonétiquement \href{https://en.oxforddictionaries.com/definition/measure}{\exPH{ˈmɛʒə}}

  \begin{itemize}
  \item\exEN{This application \href{https://youtu.be/bN60fb9fzKg}{measures} the speed of the \href{https://youtu.be/9hIQjrMHTv4}{Internet}
      connection.}
  \item\exFR{Cette application calcule la vitesse de la connexion Internet.}
  \end{itemize}

  \youglish{measures}
  
\item \exEN{\href{http://www.wordreference.com/enfr/pleasure}{pleasure}} qui s'écrit phonétiquement \href{https://en.oxforddictionaries.com/definition/pleasure}{\exPH{ˈplɛʒə}}

  \begin{itemize}
  \item\exEN{I \href{https://youtu.be/wv-mD-QHRMQ}{read} your book with great \href{https://youtu.be/Q4-VK5uqY34}{pleasure}.}
  \item\exFR{J'ai lu votre livre avec grand plaisir.}
  \end{itemize}

  \youglish{pleasure}
  
\item \exEN{\href{http://www.wordreference.com/enfr/vision}{vision}} qui s'écrit phonétiquement \href{https://en.oxforddictionaries.com/definition/vision}{\exPH{ˈvɪʒ(ə)n}}

  \begin{itemize}
  \item\exEN{The teacher's \href{https://youtu.be/lk7lIhAmwHI}{vision} was getting \href{https://youtu.be/rln_kZbYaWc}{fuzzy} so he put his
      glasses on.}
  \item\exFR{Comme sa vision devenait floue, le professeur a mis ses lunettes.}
  \end{itemize}

  \youglish{vision}

\end{enumerate}

\newpage

\section{\son{h}}\label{sec:h}

Ce \textcolor{red}{son} a pour nom technique\dyse{voiceless-glottal-fricative} :% #1: lien
                                % vers le blog
  %
  \begin{itemize}%
  \item \exEN{Voiceless glottal fricative \CW{https://en.wikipedia.org/wiki/Voiceless_glottal_fricative}.}% #2: sound name, #3: wiki EN
  \item \exFR{Consonne fricative glottale sourde \CW{https://fr.wikipedia.org/wiki/Consonne_fricative_glottale_sourde}.}% #4: nom du son, #5: wiki FR sinon blog voir
                         % package ifthen pour gérer ça
  \end{itemize}%
  %
  \indicsound%
  %
  \properukus{https://youtu.be/r0GW2q9gibY}{https://youtu.be/uOG-4ZjR7ic}% #6: UK YT, #7: US YT
  
\begin{enumerate}
\item \exEN{\href{http://www.wordreference.com/enfr/ahead}{ahead}} qui s'écrit phonétiquement \href{https://en.oxforddictionaries.com/definition/ahead}{\exPH{əˈhɛd}} 

  \begin{itemize}
  \item\exEN{It has been \href{https://youtu.be/kf9xGFd5dTU}{major}, important and time-consuming work,
      because we in actual fact have demanding and important tasks
      \href{https://youtu.be/1rLpIOzKaBA}{ahead} of us.}
  \item\exFR{C'est un travail énorme, important et très long, dans
      la mesure où les missions qui nous attendent sont importantes
      et exigeantes.}
  \end{itemize}

  \youglish{ahead}
  
\item \exEN{\href{http://www.wordreference.com/enfr/hello}{hello}} qui s'écrit phonétiquement \href{https://en.oxforddictionaries.com/definition/hello}{\exPH{hɛˈləʊ}}

  \begin{itemize}
  \item\exEN{\href{https://youtu.be/62XB9IbMnxQ}{Hello} \href{https://en.wikipedia.org/wiki/\%2522Hello,\_World!\%2522\_program}{world}!}
  \item\exFR{Bonjour le monde !}
  \end{itemize}

  \youglish{hello}
  
\item \exEN{\href{http://www.wordreference.com/enfr/high}{high}} qui s'écrit phonétiquement \href{https://en.oxforddictionaries.com/definition/high}{\exPH{hʌɪ}}

  \begin{itemize}
  \item\exEN{\href{https://youtu.be/F7lj4LknWO8}{High} walls surrounded the \href{https://youtu.be/hclQLklBHNs}{castle}.}
  \item\exFR{De hauts murs entouraient le château.}
  \end{itemize}

  \youglish{high}

\item \exEN{\href{http://www.wordreference.com/enfr/whole}{whole}} qui s'écrit phonétiquement \href{https://en.oxforddictionaries.com/definition/whole}{\exPH{həʊl}}

  \begin{itemize}
  \item\exEN{The \href{https://youtu.be/5eTCZ9L834s}{environment} concerns society as a \href{https://youtu.be/bJnw1ma6Xks}{whole}.}
  \item\exFR{L'environnement concerne l'ensemble de la société.}
  \end{itemize}

  \youglish{whole}

\end{enumerate}

\newpage
\minitoc
\newpage

\chapter{Affricate (plosive + fricative)}\label{chap:affricative}

\speech{2}{consonnes \exFR{affriquées\CW{https://fr.wikipedia.org/wiki/Consonne_affriqu\%C3\%A9e}} (\exEN{africate\CW{https://en.wikipedia.org/wiki/Affricate_consonant}})}

\newpage
\minitoc
\newpage

\section{\son{tʃ} }\label{sec:tss}

Ce \textcolor{red}{son} a pour nom technique\dyse{voiceless-postalveolar-affricate} :% #1: lien
                                % vers le blog
  %
  \begin{itemize}%
  \item \exEN{Voiceless postalveolar affricate \CW{https://en.wikipedia.org/wiki/Voiceless_postalveolar_affricate}.}% #2: sound name, #3: wiki EN
  \item \exFR{Consonne affriquée palato-alvéolaire sourde \CW{https://fr.wikipedia.org/wiki/Consonne_affriqu\%C3\%A9e_palato-alv\%C3\%A9olaire_sourde}.}% #4: nom du son, #5: wiki FR sinon blog voir
                         % package ifthen pour gérer ça
  \end{itemize}%
  %
  \indicsound%
  %
  \properukus{https://youtu.be/6SreswdXlAk}{https://youtu.be/unfuGPc3iXo}% #6: UK YT, #7: US YT

\begin{enumerate}
\item \exEN{\href{http://www.wordreference.com/enfr/cheese}{cheese}} qui
  s'écrit phonétiquement
  \href{https://en.oxforddictionaries.com/definition/cheese}{\exPH{tʃiːz}}
  
  \begin{itemize}
  \item\exEN{The \href{https://youtu.be/xYyP9o8wXtc}{cheese} had an \href{https://youtu.be/uxBg3KSeZO8}{awful} smell.}
  \item\exFR{Le fromage dégageait une odeur horrible.}
  \end{itemize}

  \youglish{cheese}
  
\item \exEN{\href{http://www.wordreference.com/enfr/match}{match}} qui
  s'écrit phonétiquement
  \href{https://en.oxforddictionaries.com/definition/match}{\exPH{matʃ}}
  
  \begin{itemize}
  \item\exEN{The password \href{https://youtu.be/-o\_IoZdtbWs}{matches} the one in the \href{https://youtu.be/FR4QIeZaPeM}{database}.}
  \item\exFR{Le mot de passe correspond à celui de la base de données.}
  \end{itemize}

  \youglish{match}
  
\item \exEN{\href{http://www.wordreference.com/enfr/nature}{nature}} qui
  s'écrit phonétiquement
  \href{https://en.oxforddictionaries.com/definition/nature}{\exPH{ˈneɪtʃə}}
  
  \begin{itemize}
  \item\exEN{Preserving \href{https://youtu.be/K\_jwPJM0QSc}{nature} is a \href{https://youtu.be/wyRy8kowyM8}{matter} of public concern.}
  \item\exFR{Préserver la nature est une question de responsabilité publique.}
  \end{itemize}

  \youglish{nature}
  
\item \exEN{\href{http://www.wordreference.com/enfr/watch}{watch}} qui
  s'écrit phonétiquement
  \href{https://en.oxforddictionaries.com/definition/watch}{\exPH{wɒtʃ}}
  
  \begin{itemize}
  \item\exEN{When my \href{https://youtu.be/jskG0yVDMLk}{parents} go out I have to \href{https://youtu.be/Eya0daHX-Fw}{watch} my little
      sister.}
  \item\exFR{Quand mes parents sortent je dois surveiller ma petite soeur.}
  \end{itemize}

  \youglish{watch}
  
\end{enumerate}

\newpage

\section{\son{dʒ} }\label{sec:dj}

Ce \textcolor{red}{son} a pour nom technique\dyse{voiced-postalveolar-affricate} :% #1: lien
                                % vers le blog
  %
  \begin{itemize}%
  \item \exEN{Voiced postalveolar affricate \CW{https://en.wikipedia.org/wiki/Voiced_postalveolar_affricate}.}% #2: sound name, #3: wiki EN
  \item \exFR{Consonne affriquée palato-alvéolaire voisée \CW{https://fr.wikipedia.org/wiki/Consonne_affriqu\%C3\%A9e_palato-alv\%C3\%A9olaire_vois\%C3\%A9e}.}% #4: nom du son, #5: wiki FR sinon blog voir
                         % package ifthen pour gérer ça
  \end{itemize}%
  %
  \indicsound%
  %
  \properukus{https://youtu.be/vqL9ivPb09A}{https://youtu.be/unfuGPc3iXo}% #6: UK YT, #7: US YT
  
\begin{enumerate}
\item \exEN{\href{http://www.wordreference.com/enfr/age}{age}} qui s'écrit
  phonétiquement
  \href{https://en.oxforddictionaries.com/definition/age}{\exPH{eɪdʒ}}
  
  \begin{itemize}
  \item\exEN{\href{https://youtu.be/wKU5khnuY\_Y}{Age} and inactivity reduce joint \href{https://youtu.be/sYrIMdOBHkg}{mobility}.}
  \item\exFR{L'âge et l'inactivité réduisent la mobilité articulaire.}
  \end{itemize}

  \youglish{age}
  
\item \exEN{\href{http://www.wordreference.com/enfr/joy}{joy}} qui s'écrit
  phonétiquement
  \href{https://en.oxforddictionaries.com/definition/joy}{\exPH{dʒɔɪ}}
  
  \begin{itemize}
  \item\exEN{The \href{https://youtu.be/TyYIxGL2p6c}{joy} of \href{https://www.amazon.fr/gp/product/B013RQ72R2/ref=as\_li\_tl?ie=UTF8\&camp=1642\&creative=6746\&creativeASIN=B013RQ72R2\&linkCode=as2\&tag=wwwbecomefree-21\&linkId=e8ebecacb076d66dd3e5a435789050d5}{phonetics}.}
  \item\exFR{La joie de la phonétique.}
  \end{itemize}

  \youglish{joy}
  
\item \exEN{\href{http://www.wordreference.com/enfr/juggle}{juggle}} qui
  s'écrit phonétiquement
  \href{https://en.oxforddictionaries.com/definition/juggle}{\exPH{ˈdʒʌɡ(ə)l}}
  
  \begin{itemize}
  \item\exEN{He can \href{https://youtu.be/kCt1bmSASCI}{juggle} with five \href{https://youtu.be/wLxMbzWm5Es}{balls}.}
  \item\exFR{Il peut jongler avec cinq balles.}
  \end{itemize}

  \youglish{juggle}
  
\item \exEN{\href{http://www.wordreference.com/enfr/soldier}{soldier}} qui
  s'écrit phonétiquement
  \href{https://en.oxforddictionaries.com/definition/soldier}{\exPH{ˈsəʊldʒə}}
  
  \begin{itemize}
  \item\exEN{The \href{https://youtu.be/ucoSdNM2Atw}{soldier} defused the \href{https://youtu.be/cYyQcWPywHo}{bomb}.}
  \item\exFR{Le soldat a désamorcé la bombe.}
  \end{itemize}

  \youglish{soldier}
  
\end{enumerate}

\newpage
\minitoc
\newpage

\chapter{Nasal (air libéré par le nez)}\label{chap:nasal}

\speech{3}{consonnes \exFR{nasales\CW{https://fr.wikipedia.org/wiki/Consonne_nasale}} (\exEN{nasal\CW{https://en.wikipedia.org/wiki/Nasal_consonant}})}

\newpage
\minitoc
\newpage

\section{\son{m}}\label{chap:m}

Ce \textcolor{red}{son} a pour nom
technique\dyse{bilabial-nasal} :

\begin{itemize}
\item \exEN{Bilabial Nasal\CW{https://en.wikipedia.org/wiki/Bilabial_nasal}.}
\item \exFR{Consonne nasale bilabiale voisée\CW{https://fr.wikipedia.org/wiki/Consonne_nasale_bilabiale_vois\%C3\%A9e}.}
\end{itemize}

\indicsound

\properukus{https://youtu.be/_o7H0eoj2SI}{https://youtu.be/tkiN8BsBEfA}

\begin{enumerate}
  
\item \exEN{\href{http://www.wordreference.com/enfr/calm}{calm}} qui s'écrit phonétiquement \href{https://en.oxforddictionaries.com/definition/calm}{\exPH{kɑːm}}

  \begin{itemize}
  \item\exEN{He kept \href{https://youtu.be/1tXBl3Q5Ibc}{calm} in order not to start a \href{https://youtu.be/e1il5yarxLU}{scrap}.}
  \item\exFR{Il est resté calme afin de ne pas déclencher une bagarre.}
  \end{itemize}

  \youglish{calm}


\item \exEN{\href{http://www.wordreference.com/enfr/hammer}{hammer}} qui s'écrit phonétiquement \href{https://en.oxforddictionaries.com/definition/hammer}{\exPH{ˈhamə}}

  \begin{itemize}
  \item\exEN{I need a \href{https://youtu.be/ReRiyfwW6-g}{screwdriver} and a \href{https://youtu.be/t5l2AUlD8Sk}{hammer} to fix the shelf.}
  \item\exFR{J'ai besoin d'un tournevis et d'un marteau pour
      réparer l'étagère.}
  \end{itemize}

  \youglish{hammer}

\item \exEN{\href{http://www.wordreference.com/enfr/mad}{mad}} qui s'écrit phonétiquement \href{https://en.oxforddictionaries.com/definition/mad}{\exPH{mad}}

  \begin{itemize}
  \item\exEN{The \href{https://youtu.be/0T3tg6Xwt7A}{scientist} must be \href{https://youtu.be/Oa-ae6\_okmg}{mad} to try such experiments.}
  \item\exFR{Le scientifique doit être dingue pour tenter de telles expériences.}
  \end{itemize}

  \youglish{mad}

\item \exEN{\href{http://www.wordreference.com/enfr/sum}{sum}} qui s'écrit phonétiquement \href{https://en.oxforddictionaries.com/definition/sum}{\exPH{sʌm}}
  
  \begin{itemize}
  \item\exEN{The \href{https://youtu.be/ymUTWzsoiIg}{sum} is indicated on the \href{https://youtu.be/MAwTnbtC3w4}{invoice}.}
  \item\exFR{Le total est indiqué sur la facture.}
  \end{itemize}

  \youglish{sum}

\end{enumerate}

\newpage

\section{\son{n}}\label{chap:n}

Ce \textcolor{red}{son} a pour nom
technique\dyse{alveolar-nasal} :

\begin{itemize}
\item \exEN{Alveolar Nasal\CW{https://en.wikipedia.org/wiki/Dental,_alveolar_and_postalveolar_nasals}.}
\item \exFR{Consonne nasale alvéolaire voisée\CW{https://fr.wikipedia.org/wiki/Consonne_nasale_alv\%C3\%A9olaire_vois\%C3\%A9e}.}
\end{itemize}

\indicsound

\properukus{https://youtu.be/vDrdgvk-G30}{https://youtu.be/1eyr7O4TFmI}

\begin{enumerate}
\item \exEN{\href{http://www.wordreference.com/enfr/know}{know}} qui s'écrit phonétiquement \href{https://en.oxforddictionaries.com/definition/know}{\exPH{nəʊ}}

  \begin{itemize}
  \item\exEN{I \href{https://youtu.be/j-CwwdwQV54}{know} a good restaurant \href{https://youtu.be/M2MofcPjonU}{nearby}.}
  \item\exFR{Je connais un bon restaurant à proximité.}
  \end{itemize}

  \youglish{know}

\item \exEN{\href{http://www.wordreference.com/enfr/nobody}{nobody}} qui s'écrit phonétiquement \href{https://en.oxforddictionaries.com/definition/nobody}{\exPH{ˈnəʊbədi}}

  \begin{itemize}
  \item\exEN{At the time \href{https://youtu.be/icE0AqVSnzo}{nobody} could have known that it would \href{https://youtu.be/bHp3ctfCaAU}{take}
      six months.}
  \item\exFR{Personne ne pouvait à ce moment savoir que ceci prendrait six mois.}
  \end{itemize}

  \youglish{nobody}

\item \exEN{\href{http://www.wordreference.com/enfr/funny}{funny}} qui s'écrit phonétiquement \href{https://en.oxforddictionaries.com/definition/funny}{\exPH{ˈfʌni}}
  
  \begin{itemize}
  \item\exEN{All \href{https://youtu.be/78cIuEMWaI4}{visitors}, especially the children, found the clown
      \href{https://youtu.be/CNXOu7gPEXM}{funny}.}
  \item\exFR{Tous les visiteurs, surtout les enfants, ont trouvé
      le clown amusant.}
  \end{itemize}

  \youglish{funny}

\item \exEN{\href{http://www.wordreference.com/enfr/turn}{turn}} qui s'écrit phonétiquement \href{https://en.oxforddictionaries.com/definition/turn}{\exPH{təːn}}

  \begin{itemize}
  \item\exEN{The \href{https://youtu.be/pjlPgJ1wBdM}{negotiations} have taken a decisive \href{https://youtu.be/z4g45vTgczE}{turn} today.}
  \item\exFR{Les négociations ont pris un tournant décisif
      aujourd'hui.}
  \end{itemize}

  \youglish{turn}

\end{enumerate}

\newpage

\section{\son{ŋ}}\label{chap:ing}

Ce \textcolor{red}{son} a pour nom
technique\dyse{velar-nasal} :

\begin{itemize}
\item \exEN{Velar Nasal\CW{https://en.wikipedia.org/wiki/Velar_nasal}.}
\item \exFR{Consonne nasale vélaire voisée\CW{https://fr.wikipedia.org/wiki/Consonne_nasale_v\%C3\%A9laire_vois\%C3\%A9e}.}
\end{itemize}

\indicsound

\properukus{https://youtu.be/-RfiBn9qPlM}{https://youtu.be/5xVq8T88oJw}


\begin{enumerate}
\item \exEN{\href{http://www.wordreference.com/enfr/anger}{anger}} qui s'écrit phonétiquement \href{https://en.oxforddictionaries.com/definition/anger}{\exPH{ˈaŋɡə}}

  \begin{itemize}
  \item\exEN{He suddenly \href{https://youtu.be/8jsC9KVI0A4}{unleashed} his \href{https://youtu.be/lw64e7JVRj0}{anger}.}
  \item\exFR{Il a soudainement déchaîné sa colère.}
  \end{itemize}

  \youglish{anger}

\item \exEN{\href{http://www.wordreference.com/enfr/bang}{bang}} qui s'écrit phonétiquement \href{https://en.oxforddictionaries.com/definition/bang}{\exPH{baŋ}}. Exemple d'utilisation de ce
  mot :
  
  \begin{itemize}
  \item\exEN{A loud \href{https://youtu.be/N-AgYXz2n9Y}{bang} woke me up in the \href{https://youtu.be/XAa11jc-LU0}{middle} of the night.}
  \item\exFR{Un grand fracas m'a réveillé en pleine nuit.}
  \end{itemize}

  \youglish{bang}

\item \exEN{\href{http://www.wordreference.com/enfr/king}{king}} qui s'écrit phonétiquement \href{https://en.oxforddictionaries.com/definition/king}{\exPH{kɪŋ}}

  \begin{itemize}
  \item\exEN{The \href{https://youtu.be/MRgFeZa\_I48}{king} and \href{https://youtu.be/wiDCwqpupj8}{queen} live in a magnificent palace.}
  \item\exFR{Le roi et la reine habitent dans un palais magnifique.}
  \end{itemize}

  \youglish{king}

\item \exEN{\href{http://www.wordreference.com/enfr/thanks}{thanks}} qui s'écrit phonétiquement \href{https://en.oxforddictionaries.com/definition/thanks}{\exPH{θaŋks}}
  
  \begin{itemize}
  \item\exEN{We solved the \href{https://youtu.be/nQpCuYS41Oo}{problem} \href{https://youtu.be/hQiipuDbbxw}{thanks} to a concerted effort.}
  \item\exFR{Nous avons résolu le problème grâce à un effort
      concerté.}
  \end{itemize}

  \youglish{thanks}

\end{enumerate}

\newpage
\minitoc
\newpage

\chapter{Approximant (bloc d'air partiel, semblable à une voyelle)}
\label{chap:approx}

\speech{5}{consonnes \exFR{spirantes\CW{https://fr.wikipedia.org/wiki/Consonne_spirante}} (\exEN{approximant\CW{https://en.wikipedia.org/wiki/Approximant_consonant}})}

\newpage
\minitoc
\newpage

\section{\son{w}}\label{sec:w}

Ce \textcolor{red}{son} a pour nom
technique\dyse{voiced-labio-velar-approximant} :

\begin{itemize}
\item \exEN{Voiced labio-velar Approximant\CW{https://en.wikipedia.org/wiki/Voiced_labio-velar_approximant}.}
\item \exFR{Consonne spirante labio-vélaire voisée\CW{https://fr.wikipedia.org/wiki/Consonne_spirante_labio-v\%C3\%A9laire_vois\%C3\%A9e}.}
\end{itemize}

\indicsound

\properukus{https://youtu.be/4MpDb-gTipY}{https://youtu.be/RW94L6606DE}

\begin{enumerate}
\item \exEN{\href{http://www.wordreference.com/enfr/one}{one}} qui s'écrit phonétiquement \href{https://en.oxforddictionaries.com/definition/one}{\exPH{wʌn}}

  \begin{itemize}
  \item\exEN{\href{https://youtu.be/aSNJ00iAZ7I}{One} never knows where to begin, so let's start with
      the number \href{https://youtu.be/jHRXlK2SnQ8}{one}.}
  \item\exFR{On ne sait jamais par où commencer, alors commençons
      par le numéro un.}
  \end{itemize}

  \youglish{one}

\item \exEN{\href{http://www.wordreference.com/enfr/queen}{queen}} qui s'écrit phonétiquement \href{https://en.oxforddictionaries.com/definition/queen}{\exPH{kwiːn}}

  \begin{itemize}
  \item\exEN{The \href{https://youtu.be/Jmd4OLzhQw0}{queen} chooses her \href{https://youtu.be/VYhbUyzUUrA}{entourage} very carefully.}
  \item\exFR{La reine choisit \textcolor{red}{son} entourage très soigneusement.}
  \end{itemize}

  \youglish{queen}

\item \exEN{\href{http://www.wordreference.com/enfr/wall}{wall}} qui s'écrit phonétiquement \href{https://en.oxforddictionaries.com/definition/wall}{\exPH{wɔːl}}

  \begin{itemize}
  \item\exEN{The \href{https://youtu.be/SF_WqsmH-Lw}{shelf} is attached to the \href{https://youtu.be/BN5Z28Dfl7o}{wall}.}
  \item\exFR{L'étagère est fixée au mur.}
  \end{itemize}

  \youglish{wall}

\item \exEN{\href{http://www.wordreference.com/enfr/world}{world}} qui s'écrit phonétiquement \href{https://en.oxforddictionaries.com/definition/world}{\exPH{wəːld}}

  \begin{itemize}
  \item\exEN{The company entered the \href{https://youtu.be/fzDft0DZRUw}{world} market with great
      \href{https://youtu.be/KMp_EZLxAHc}{success}.}
  \item\exFR{L'entreprise est entrée sur le marché mondial avec
      grand succès.}
  \end{itemize}

  \youglish{world}

\end{enumerate}

\newpage

\section{\son{j}}\label{chap:j}

Ce \textcolor{red}{son} a pour nom
technique\dyse{palatal-approximant} :

\begin{itemize}
\item \exEN{Palatal Approximant\CW{https://en.wikipedia.org/wiki/Palatal_approximant}.}
\item \exFR{Consonne spirante palatale voisée\CW{https://fr.wikipedia.org/wiki/Consonne_spirante_palatale_vois\%C3\%A9e}.}
\end{itemize}

\indicsound

\properukus{https://youtu.be/XhqGU1WxOfc}{https://youtu.be/1Yo4BHIIBP8}

\begin{enumerate}
\item \exEN{\href{http://www.wordreference.com/enfr/beauty}{beauty}} qui s'écrit phonétiquement \href{https://en.oxforddictionaries.com/definition/beauty}{\exPH{ˈbjuːti}}

  \begin{itemize}
  \item\exEN{The actress is the \href{https://youtu.be/JYacDPOWsmE}{embodiment} of talent and \href{https://youtu.be/IzwWXNxFiyA}{beauty}.}
  \item\exFR{L'actrice est l'incarnation du talent et de la
      beauté.}
  \end{itemize}

  \youglish{beauty}

\item \exEN{\href{http://www.wordreference.com/enfr/few}{few}} qui s'écrit phonétiquement \href{https://en.oxforddictionaries.com/definition/few}{\exPH{fjuː}}

  \begin{itemize}
  \item\exEN{I gave my friend a \href{https://youtu.be/6E2hYDIFDIU}{few} tips to save \href{https://youtu.be/JkhX5W7JoWI}{money}.}
  \item\exFR{J'ai donné quelques conseils à mon ami pour
      économiser de l'argent.}
  \end{itemize}

  \youglish{few}

\item \exEN{\href{http://www.wordreference.com/enfr/usual}{usual}} qui s'écrit phonétiquement \href{https://en.oxforddictionaries.com/definition/usual}{\exPH{ˈjuːʒʊəl}}

  \begin{itemize}
  \item\exEN{My \href{https://youtu.be/IdlOj1n3vXA}{mother} made her \href{https://youtu.be/ThLRPCs8uzc}{usual} cake for my birthday.}
  \item\exFR{Ma mère a fait son gâteau traditionnel pour mon
      anniversaire.}
  \end{itemize}

  \youglish{usual}

\item \exEN{\href{http://www.wordreference.com/enfr/yellow}{yellow}} qui s'écrit phonétiquement \href{https://en.oxforddictionaries.com/definition/yellow}{\exPH{ˈjɛləʊ}}

  \begin{itemize}
  \item\exEN{We all live in a \href{https://youtu.be/m2uTFF\_3MaA}{yellow} \href{https://www.lacoccinelle.net/245633.html}{submarine}.}
  \item\exFR{Nous vivons tous dans un sous-marin jaune.}
  \end{itemize}

  \youglish{yellow}

\end{enumerate}

\newpage

\section{\son{r}}\label{sec:r}

Ce \textcolor{red}{son} a pour nom
technique\dyse{alveolar-approximant} :

\begin{itemize}
\item \exEN{Alveolar Approximant\CW{https://en.wikipedia.org/wiki/Alveolar_and_postalveolar_approximants}.}
\item \exFR{Consonne spirante alvéolaire voisée\CW{https://fr.wikipedia.org/wiki/Consonne_spirante_alv\%C3\%A9olaire_vois\%C3\%A9e}.}
\end{itemize}

\indicsound

\properukus{https://youtu.be/exUcpYrZotc}{https://youtu.be/q5a2-KuHkBU}

\begin{enumerate}
\item \exEN{\href{http://www.wordreference.com/enfr/arrange}{arrange}} qui s'écrit phonétiquement \href{https://en.oxforddictionaries.com/definition/arrange}{\exPH{əˈreɪn(d)ʒ}}

  \begin{itemize}
  \item\exEN{We can \href{https://youtu.be/oD5RzpwbrIc}{arrange} another meeting if \href{https://youtu.be/L3xXxXu1Kfc}{necessary}.}
  \item\exFR{Nous pouvons organiser une autre réunion si
      nécessaire.}
  \end{itemize}

  \youglish{arrange}

\item \exEN{\href{http://www.wordreference.com/enfr/road}{road}} qui s'écrit phonétiquement \href{https://en.oxforddictionaries.com/definition/road}{\exPH{rəʊd}}

  \begin{itemize}
  \item\exEN{The \href{https://youtu.be/bO2xMNU9bTw}{road} passes \href{https://youtu.be/9z1A1R8RQZs}{through} the forest.}
  \item\exFR{La route passe par la forêt.}
  \end{itemize}
  
\item \exEN{\href{http://www.wordreference.com/enfr/sorry}{sorry}} qui s'écrit phonétiquement \href{https://en.oxforddictionaries.com/definition/sorry}{\exPH{ˈsɒri}}

  \begin{itemize}
  \item\exEN{I am \href{https://youtu.be/ahCwKDyS5OE}{sorry} for any inconvenience I \href{https://youtu.be/QstrRR031XE}{may} have caused.}
  \item\exFR{Je suis désolé pour tout inconvénient que j'ai pu causer.}
  \end{itemize}

  \youglish{sorry}
  
\item \exEN{\href{http://www.wordreference.com/enfr/wrong}{wrong}} qui s'écrit phonétiquement \href{https://en.oxforddictionaries.com/definition/wrong}{\exPH{rɒŋ}}

  \begin{itemize}
  \item\exEN{There are no \href{https://youtu.be/a5e0z1\_uwHY}{wrong} answers to this \href{https://youtu.be/ENBv2i88g6Y}{question}.}
  \item\exFR{Il n'y a pas de mauvaises réponses à cette question.}
  \end{itemize}

  \youglish{wrong}

\end{enumerate}

\newpage

\section{\son{l}}\label{sec:l}

Ce \textcolor{red}{son} a pour nom
technique\dyse{alveolar-lateral-approximant} :

\begin{itemize}
\item \exEN{Alveolar Lateral Approximant\CW{https://en.wikipedia.org/wiki/Dental,_alveolar_and_postalveolar_lateral_approximants}.}
\item \exFR{Consonne spirante latérale alvéolaire voisée\CW{https://fr.wikipedia.org/wiki/Consonne_spirante_lat\%C3\%A9rale_alv\%C3\%A9olaire_vois\%C3\%A9e}.}
\end{itemize}

\indicsound

\properukus{https://youtu.be/sXKFT02-nKw}{https://youtu.be/JamM8TgB_AA}

\begin{enumerate}
\item \exEN{\href{http://www.wordreference.com/enfr/feel}{feel}} qui s'écrit phonétiquement \href{https://en.oxforddictionaries.com/definition/feel}{\exPH{fiːl}}
  
  \begin{itemize}
  \item\exEN{I \href{https://youtu.be/4k4SP01l6rY}{feel} \href{https://youtu.be/DuDeBcpLITQ}{good} because I \href{https://youtu.be/w_zhHE2CB3A}{slept} well.}
  \item\exFR{Je me sens bien car j'ai bien dormi.}
  \end{itemize}

  \youglish{feel}
  
\item \exEN{\href{http://www.wordreference.com/enfr/law}{law}} qui s'écrit phonétiquement \href{https://en.oxforddictionaries.com/definition/law}{\exPH{lɔː}}

  \begin{itemize}
  \item\exEN{I want to become a \href{https://youtu.be/mGbMwP8MDjg}{judge}, so I have to study \href{https://youtu.be/GEy6ThJwE3s}{law}.}
  \item\exFR{Je souhaite devenir juge, je dois donc étudier le droit.}
  \end{itemize}

  \youglish{law}
  
\item \exEN{\href{http://www.wordreference.com/enfr/light}{light}} qui s'écrit phonétiquement \href{https://en.oxforddictionaries.com/definition/light}{\exPH{lʌɪt}}

  \begin{itemize}
  \item\exEN{\href{https://youtu.be/ULHeRdgeT54}{Light} attracts \href{https://youtu.be/tSkCqj6T\_NQ}{moths}.}
  \item\exFR{La lumière attire les papillons de nuit.}
  \end{itemize}

  \youglish{light}

\item \exEN{\href{http://www.wordreference.com/enfr/valley}{valley}} qui s'écrit phonétiquement \href{https://en.oxforddictionaries.com/definition/valley}{\exPH{ˈvali}}

  \begin{itemize}
  \item\exEN{This beautiful \href{https://youtu.be/wQFgG\_HsI0w}{valley} is covered with \href{https://youtu.be/KzDl7nOCmyU}{flowers}.}
  \item\exFR{Cette superbe vallée est recouverte de fleurs.}
  \end{itemize}

  \youglish{valley}

\end{enumerate}

\newpage

\section{\son{ɫ}}\label{sec:chelou}

Ce \textcolor{red}{son} a pour nom
technique\dyse{dark-l} :

\begin{itemize}
\item \exEN{Dark L.}
\item \exFR{L sombre.}
\end{itemize}

\indicsound

\properukus{https://youtu.be/rlw6YbzETfk}{https://youtu.be/U4En7vG1wV4}

\begin{enumerate}
\item \exEN{\href{http://www.wordreference.com/enfr/full}{full}} qui s'écrit phonétiquement \href{https://home.cc.umanitoba.ca/\~krussll/phonetics/narrower/dark-l.html}{\exPH{fʊɫ} or \exPH{fɫ̩\}}}

  \begin{itemize}
  \item\exEN{His \href{https://youtu.be/TLmlgCteCMw}{full} name appears on his \href{https://youtu.be/Q7oF-t9Ybv8}{passport}.}
  \item\exFR{Son nom complet figure sur \textcolor{red}{son} passeport.}
  \end{itemize}

  \youglish{full}
  
\item \exEN{\href{http://www.wordreference.com/enfr/flail}{flail}} qui s'écrit phonétiquement \href{https://home.cc.umanitoba.ca/\~krussll/phonetics/narrower/dark-l.html}{\exPH{fleɫ}}
  
  \begin{itemize}
  \item\exEN{In Europe, a rope \href{https://youtu.be/AGf7n7iUF\_k}{flail} has been tried with some
      \href{https://youtu.be/nvQKfdQvEBk}{success}.}
  \item\exFR{En Europe, un fléau de corde a fait l'objet d'essais
      qui ont été raisonnablement satisfaisants.}
  \end{itemize}

  \youglish{flail}

\item \exEN{\href{http://www.wordreference.com/enfr/little}{little}} qui s'écrit phonétiquement \href{https://home.cc.umanitoba.ca/\~krussll/phonetics/narrower/dark-l.html}{\exPH{ˈlɪɾɫ̩\}}}

  \begin{itemize}
  \item\exEN{The car \href{https://youtu.be/r-61yYjKHHc}{salesman} offers many options at \href{https://youtu.be/FokJGK639R4}{little} cost.}
  \item\exFR{Le vendeur de voitures propose de nombreuses options à faible coût.}
  \end{itemize}

  \youglish{little}

\item \exEN{\href{http://www.wordreference.com/enfr/milk}{milk}} qui s'écrit phonétiquement \href{https://home.cc.umanitoba.ca/\~krussll/phonetics/narrower/dark-l.html}{\exPH{mɪɫk}}
  
  \begin{itemize}
  \item\exEN{The farmer \href{https://youtu.be/XBOE2CC0YYY}{milks} his cows every \href{https://youtu.be/pv7zbiLVBQw}{morning}.}
  \item\exFR{Le fermier trait ses vaches tous les matins.}
  \end{itemize}

  \youglish{milk}

  
\end{enumerate}

\newpage
\minitoc
