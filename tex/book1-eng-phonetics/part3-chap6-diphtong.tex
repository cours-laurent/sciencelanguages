\chapter{Diphthong Vowels}\label{chap:diphtong}

\begin{center}
  \begin{figure}[h]
    \centering
    \includegraphics[scale=.75]{../img/ogden/rp-cent-diph-ogden}
    \caption{Diphtongues centrales, source : \ogden }
    \label{fig:diph-cent}
  \end{figure}
\end{center}

\speech{8}{diphtongues\CW{https://fr.wikipedia.org/wiki/Diphtongue}}

\begin{center}
  \begin{figure}[h]
    \centering
    \includegraphics[scale=.75]{../img/ogden/rp-clos-diph-ogden}
    \caption{Diphtongues fermantes, source : \ogden }
    \label{fig:diph-ferm}
  \end{figure}
\end{center}


\newpage
\minitoc
\newpage

\section{Le \son \phon{eɪ} }\label{sec:ei}

\diph{e}{ɪ}{eɪ}{diphthong-1-7}{Le \son \phon{e} n'existe pas à l'état isolé en
  anglais et pour le \son \phon{ɪ} voir page~\pageref{sec:soni}.}{ei}
\flags
\properukus{https://youtu.be/oTAzk9xm5i8}{https://youtu.be/0RXzfRcjk-s}

\begin{enumerate}
\item \exEN{\href{http://www.wordreference.com/enfr/snake}{snake}} qui
  s'écrit phonétiquement
  \href{https://en.oxforddictionaries.com/definition/snake}{\phonm{sneɪk}}

  \begin{itemize}
  \item\exEN{\href{https://youtu.be/MOltIVdyAHQ}{Snakes} regularly shed their \href{https://youtu.be/6kjaagQcYkc}{skin}.}
  \item\exFR{Les serpents perdent régulièrement leur peau.}
  \end{itemize}

  \youglish{snake}
  
\item \exEN{\href{http://www.wordreference.com/enfr/pay}{pay}} qui s'écrit
  phonétiquement
  \href{https://en.oxforddictionaries.com/definition/pay}{\phonm{peɪ}}

  \begin{itemize}
  \item\exEN{How much \href{https://youtu.be/wRSNm3pr100}{would} you be able to \href{https://youtu.be/mBuLm5XeF44}{pay} for additional
      content?}
  \item\exFR{Combien seriez-vous capable de payer pour du contenu
      supplémentaire ?}
  \end{itemize}

  \youglish{pay}
  
\item \exEN{\href{http://www.wordreference.com/enfr/mail}{mail}} qui s'écrit
  phonétiquement
  \href{https://en.oxforddictionaries.com/definition/mail}{\phonm{meɪl}}
  
  \begin{itemize}
  \item\exEN{The \href{https://youtu.be/CFNaCU3sXh8}{post office} redirected the \href{https://youtu.be/KX1CSSZa1v0}{mail} to my new address.}
  \item\exFR{Le bureau de poste a fait suivre le courrier à ma
      nouvelle adresse.}
  \end{itemize}

  \youglish{mail}
  
\item \exEN{\href{http://www.wordreference.com/enfr/great}{great}} qui
  s'écrit phonétiquement
  \href{https://en.oxforddictionaries.com/definition/great}{\phonm{ɡreɪt}}
  
  \begin{itemize}
  \item\exEN{Your \href{https://youtu.be/dBnpr3pkFlk}{content} is \href{https://youtu.be/e0qM84DWXzA}{great}!}
  \item\exFR{Ton contenu est génial !}
  \end{itemize}

  \youglish{great}
  
\end{enumerate}

\newpage

\section{Le \son \phon{ɔɪ} }\label{sec:oouverti}

\diph{ɔː}{ɪ}{ɔɪ}{diphthong-2-7}{Le \son \phon{ɔː} a été étudié page~\pageref{sec:oouvert} et pour le \son \phon{ɪ} voir page~\pageref{sec:soni}.}{oi}
\flags
\properukus{https://youtu.be/M-8ZqxVJMf8}{https://youtu.be/ZfjPBN22mK8}

\begin{enumerate}
\item \exEN{\href{http://www.wordreference.com/enfr/toy}{toy}} qui s'écrit
  phonétiquement
  \href{https://en.oxforddictionaries.com/definition/toy}{\phonm{tɔɪ}}

  \begin{itemize}
  \item\exEN{The \href{https://youtu.be/UyzQMSlhQik}{little} boy was delighted with all his \href{https://youtu.be/1qbuZhVUj\_g}{toys}.}
  \item\exFR{Le petit garçon était enchanté par tous ses jouets.}
  \end{itemize}

  \youglish{toy}
  
\item \exEN{\href{http://www.wordreference.com/enfr/choice}{choice}} qui
  s'écrit phonétiquement
  \href{https://en.oxforddictionaries.com/definition/choice}{\phonm{tʃɔɪs}}

  \begin{itemize}
  \item\exEN{\href{https://youtu.be/2yKfkr2lqQM}{Looking} at my additional content is your \href{https://youtu.be/qBfeK\_IIHag}{choice}.}
  \item\exFR{Regarder mon contenu supplémentaire est votre choix.}
  \end{itemize}

  \youglish{choice}
  
\item \exEN{\href{http://www.wordreference.com/enfr/joy}{joy}} qui s'écrit
  phonétiquement
  \href{https://en.oxforddictionaries.com/definition/joy}{\phonm{dʒɔɪ}}

  \begin{itemize}
  \item\exEN{The music \href{https://youtu.be/jzUp8mdionM}{creates} a sensation of \href{https://youtu.be/-GjW1pSYgUk}{joy} and playfulness.}
  \item\exFR{La musique crée une sensation de joie et de gaieté.}
  \end{itemize}

  \youglish{joy}
  
\item \exEN{\href{http://www.wordreference.com/enfr/oyster}{oyster}} qui
  s'écrit phonétiquement
  \href{https://en.oxforddictionaries.com/definition/oyster}{\phonm{ˈɔɪstə}}

  \begin{itemize}
  \item\exEN{\href{https://youtu.be/arj7oStGLkU}{Inside} the \href{https://youtu.be/PVn6b9QQZeM}{oyster}, I found a pearl.}
  \item\exFR{À l'intérieur de l'huître, j'ai trouvé une perle.}
  \end{itemize}

  \youglish{oyster}
  
\end{enumerate}

\newpage

\section{Le \son \phon{aɪ} }\label{sec:ai}

\diph{aː}{ɪ}{aɪ}{diphthong-3-7}{Le \son \phon{aː} a été étudié
  page~\pageref{sec:sonalong} et pour le \son \phon{ɪ} voir
  page~\pageref{sec:soni}.}{ai}
\flags
\properukus{https://youtu.be/ub9ONgsThKc}{https://youtu.be/8uD-GuuSgyk}

\begin{enumerate}
\item \exEN{\href{http://www.wordreference.com/enfr/my}{my}} qui s'écrit
  phonétiquement
  \href{https://dictionary.cambridge.org/dictionary/english/my}{\phonm{maɪ}}

  \begin{itemize}
  \item\exEN{\href{https://youtu.be/nMRyjZ1rbB0}{My} content is made to help you \href{https://www.youtube.com/watch?v=m\_uWS6K-VF8\&list=PL0J5xb8JH3VukoRHgk86Yr9BSVeBewCuZ}{progress in English}.}
  \item\exFR{Mon contenu est fait pour vous aider à progresser en
      anglais.}
  \end{itemize}

  \youglish{my}
  
\item \exEN{\href{http://www.wordreference.com/enfr/while}{while}} qui
  s'écrit phonétiquement
  \href{https://dictionary.cambridge.org/dictionary/english/while}{\phonm{waɪl}}

  \begin{itemize}
  \item\exEN{\href{https://youtu.be/O040xuq2FR0}{She} partied \href{https://youtu.be/8q182kWAhiM}{while} I worked.}
  \item\exFR{Elle faisait la fête alors que je travaillais.}
  \end{itemize}

  \youglish{while}
  
\item \exEN{\href{http://www.wordreference.com/enfr/might}{might}} qui
  s'écrit phonétiquement
  \href{https://dictionary.cambridge.org/dictionary/english/might}{\phonm{maɪt}}

  \begin{itemize}
  \item\exEN{\href{https://youtu.be/gGMSfiH850o}{Hurricanes} show us the \href{https://youtu.be/Nqlr35WnqTk}{might} of nature.}
  \item\exFR{Les ouragans nous démontrent la puissance de la
      nature.}
  \end{itemize}

  \youglish{might}
  
\item \exEN{\href{http://www.wordreference.com/enfr/life}{life}} qui s'écrit
  phonétiquement
  \href{https://dictionary.cambridge.org/dictionary/english/life}{\phonm{laɪf}}. Exemple
  d'utilisation du mot :

  \begin{itemize}
  \item\exEN{The author \href{https://youtu.be/5x0ARyfNyGc}{withdrew} from public \href{https://youtu.be/zyKGKoGACVk}{life}.}
  \item\exFR{L'auteur s'est retiré de la vie publique.}
  \end{itemize}

  \youglish{life}
  
\end{enumerate}

\newpage

\section{Le \son \phon{əʊ} (UK) et \phon{oʊ} (US)}\label{sec:enenvomegaenv}

\RCLF{\phon{əʊ}}{https://youtu.be/Z-pZswbP0-g}{\phon{oʊ}}{https://youtu.be/Civ7UBZP99}

% \begin{tikzpicture}[remember picture,overlay]
% \node[anchor=east,inner sep=0pt] at (current page text
% area.east|-0,3cm) {\uks{https://youtu.be/Z-pZswbP0-g}\quad \uss{https://youtu.be/Civ7UBZP99}};
% \end{tikzpicture}

\notation

\begin{itemize}
\item Pour l'anglais britannique on a :
  
  \diph{ə}{ʊ}{əʊ}{diphthong-4-7}{Le \son \phon{ə} a été étudié
  page~\pageref{sec:sonenv} et pour le \son \phon{ʊ} voir
  page~\pageref{sec:omega}.}{eo}\uks{https://youtu.be/Z-pZswbP0-g}
\item \hypertarget{oohm}{Pour l'anglais américain} on a :
  \diph{o}{ʊ}{oʊ}{diphthong-4-7}{Le \son \phon{o} n'a pas
    véritablement été étudié parce que ce n'est pas un symbole
    phonétique dans la langue anglaise et pour le \son \phon{ʊ} voir
    page~\pageref{sec:omega}.}{ou}\uss{https://youtu.be/Civ7UBZP99}
\end{itemize}

\begin{enumerate}
\item \exEN{\href{http://www.wordreference.com/enfr/alone}{alone}} qui
  s'écrit phonétiquement
  \href{https://en.oxforddictionaries.com/definition/alone}{\phonm{əˈləʊn}}

  \begin{itemize}
  \item\exEN{I experience real \href{https://youtu.be/cnsk7iXFCtY}{joy} when I am alone in \href{https://youtu.be/9wAft7t8k2c}{nature}.}
  \item\exFR{Je ressens une joie réelle quand je suis seul dans la
      nature.}
  \end{itemize}

  \youglish{alone}
  
\item \exEN{\href{http://www.wordreference.com/enfr/goat}{goat}} qui s'écrit
  phonétiquement
  \href{https://en.oxforddictionaries.com/definition/goat}{\phonm{ɡəʊt}}

  \begin{itemize}
  \item\exEN{Behind a door there is a \href{https://youtu.be/VXw0XGOVQvw}{sports} car and behind each of
      the other two there is a \href{https://youtu.be/4Lb-6rxZxx0}{goat}.}
  \item\exFR{Derrière une porte il y a une voiture de sport et
      derrière chacune des deux autres il y a une chèvre.}
  \end{itemize}

  \youglish{goat}
  
\item \exEN{\href{http://www.wordreference.com/enfr/hope}{hope}} qui s'écrit
  phonétiquement
  \href{https://en.oxforddictionaries.com/definition/hope}{\phonm{həʊp}}

  \begin{itemize}
  \item\exEN{I \href{https://youtu.be/\_pKcv0Fml-A}{hope} you will \href{https://youtu.be/aGSKrC7dGcY}{enjoy} your stay.}
    \item\exFR{J'espère que vous apprécierez votre séjour.}
    \end{itemize}

    \youglish{hope}
    
\item \exEN{\href{http://www.wordreference.com/enfr/road}{road}} qui s'écrit
  phonétiquement
  \href{https://en.oxforddictionaries.com/definition/road}{\phonm{rəʊd}}

  \begin{itemize}
  \item\exEN{\href{https://youtu.be/jzmy6iUGDo8}{Body like a back road.}}
  \item\exFR{Un corps comme une route de retour.}
  \end{itemize}

  \youglish{road}
  
\end{enumerate}

\newpage

\section{Le \son \phon{ʊə} (UK) et juste \phon{ʊ} (US)}\label{sec:omegaenvenenv}

\RCLF{\phon{ʊə}}{https://youtu.be/feseqejnkL0}{\phon{ʊ}}{https://youtu.be/moLTR-dLQQY}

\notation

\begin{itemize}
\item Pour l'anglais britannique on a :
  
  \diph{ʊ}{ə}{ʊə}{diphthong-8}{Le \son \phon{ʊ} a été étudié
  page~\pageref{sec:omega} et pour le \son \phon{ə} voir
  page~\pageref{sec:sonenv}.}{oe}\uks{https://youtu.be/feseqejnkL0}
\item Pour l'anglais américain voir le \son \phon{ʊ} page~\pageref{sec:omega}.  \uss{https://youtu.be/moLTR-dLQQY}
\end{itemize}

\begin{enumerate}
\item \exEN{\href{http://www.wordreference.com/enfr/tourist}{tourist}} qui
  s'écrit phonétiquement
  \href{https://en.oxforddictionaries.com/definition/tourist}{\phonm{ˈtʊərɪst}}

  \begin{itemize}
  \item\exEN{As a \href{https://youtu.be/98H5AN_vfOY}{tourist} I like to go where \href{https://youtu.be/PCco_YBsU3w}{others} don't.}
  \item\exFR{En tant que touriste j'aime aller là où les autres ne
      vont pas.}
  \end{itemize}

  \youglish{tourist}
  
\item \exEN{\href{http://www.wordreference.com/enfr/boor}{boor}} qui s'écrit
  phonétiquement
  \href{https://en.oxforddictionaries.com/definition/boor}{\phonm{bʊə}}

  \begin{itemize}
  \item\exEN{Nowadays it is difficult to know how to \href{https://youtu.be/Gp2GMujyC58}{behave} with women
      in order to avoid to be considered as a \href{https://youtu.be/nHe5NgSRw3A}{boor} or a boring man.}
  \item\exFR{De nos jours il est difficile de savoir comment se
      comporter avec les femmes afin d'éviter d'être considéré comme
      un rustre ou un homme ennuyeux.}
  \end{itemize}

  \youglish{boor}
  
\item \exEN{\href{http://www.wordreference.com/enfr/pure}{pure}} qui s'écrit
  phonétiquement
  \href{https://en.oxforddictionaries.com/definition/pure}{\phonm{pjʊə}}

  \begin{itemize}
  \item\exEN{It is \href{https://youtu.be/kw7tpfeRHFI}{pure} chance if you
      are an English \href{https://youtu.be/ChZJ1Q3GSuI}{native} speaker or not because nobody choose his parents.}
    \item\exFR{C'est du pur hasard si vous êtes un anglophone natif ou
        pas parce que personne ne choisit ses parents.}
    \end{itemize}

    \youglish{pure}
    
\item \exEN{\href{http://www.wordreference.com/enfr/sure}{sure}} qui s'écrit
  phonétiquement
  \href{https://en.oxforddictionaries.com/definition/sure}{\phonm{ʃʊə}}

  \begin{itemize}
  \item\exEN{I'm \href{https://youtu.be/4HPR8WDwWb0}{sure} you will \href{https://youtu.be/QG2p53z67vk}{succeed}.}
  \item\exFR{Je suis sûr que tu vas réussir.}
  \end{itemize}

  \youglish{sure}
  
\end{enumerate}

\newpage

\section{Le \son \phon{aʊ} }\label{sec:aomega}

\diph{aː}{ʊ}{aʊ}{diphthong-5-7}{Le \son \phon{aː} a été étudié
  page~\pageref{sec:sonalong} et pour le \son \phon{ʊ} voir
  page~\pageref{sec:omega}.}{ao}
\flags
\properukus{https://youtu.be/JrWuLH_AYM4}{https://youtu.be/-V690OA75bA}

\begin{enumerate}
\item \exEN{\href{http://www.wordreference.com/enfr/now}{now}} qui s'écrit
  phonétiquement
  \href{https://en.oxforddictionaries.com/definition/now}{\phonm{naʊ}}

  \begin{itemize}
  \item\exEN{I am \href{https://youtu.be/xcpxjx2fy\_E}{now} completely free and \href{https://youtu.be/q86Z6Xw_C0w}{unencumbered}.}
  \item\exFR{Je suis désormais complètement libre et sans contrainte.}
  \end{itemize}

  \youglish{now}
  
\item \exEN{\href{http://www.wordreference.com/enfr/round}{round}} qui
  s'écrit phonétiquement
  \href{https://en.oxforddictionaries.com/definition/round}{\phonm{raʊnd}}

  \begin{itemize}
  \item\exEN{The \href{https://youtu.be/LV7JviaH-HU}{boxer} won the fight in the second \href{https://youtu.be/oGTBax-Cu4Q}{round}.}
  \item\exFR{Le boxeur a gagné le combat au deuxième round.}
  \end{itemize}

  \youglish{round}
  
\item \exEN{\href{http://www.wordreference.com/enfr/mouth}{mouth}} qui
  s'écrit phonétiquement
  \href{https://en.oxforddictionaries.com/definition/mouth}{\phonm{maʊθ}}

  \begin{itemize}
  \item\exEN{In \href{https://youtu.be/FYH8DsU2WCk}{order} to produce a vowel you need to open your
      \href{https://youtu.be/kkDHKSNrJ5g}{mouth}.}
  \item\exFR{Afin de produire une voyelle vous devez ouvrir votre
      bouche.}
  \end{itemize}

  \youglish{mouth}
  
\item \exEN{\href{http://www.wordreference.com/enfr/brown}{brown}} qui
  s'écrit phonétiquement
  \href{https://en.oxforddictionaries.com/definition/brown}{\phonm{braʊn}}

  \begin{itemize}
  \item\exEN{\href{https://youtu.be/OwTXBBU0JLo}{Brown} is just a \href{https://youtu.be/ufWtK3qizPA}{colour}.}
  \item\exFR{Le marron est juste une couleur.}
  \end{itemize}

  \youglish{brown}
  
\end{enumerate}

\newpage

\section{Le \son \phon{ɪə} }\label{sec:ieenv}

\RCLF{\phon{ɪə}}{https://www.youtube.com/watch?v=AnjNcqUKSsE}{\phon{ɪ}}{https://youtu.be/-km81q6DIlM}
\begin{itemize}
\item Pour l'anglais britannique on a :
  
  \diph{ɪ}{ə}{ɪə}{diphthong-6-7}{Le \son \phon{ɪ} a été étudié
  page~\pageref{sec:soni} et pour le \son \phon{ə} voir
  page~\pageref{sec:sonenv}.}{ie}
  \uks{https://www.youtube.com/watch?v=AnjNcqUKSsE}
\item Pour l'anglais américain ce n'est pas une diphtongue c'est tout
  simplement le \son \phon{ɪ} qui a été étudié page~\pageref{sec:soni}.
  \uss{https://youtu.be/-km81q6DIlM}
\end{itemize}

\begin{enumerate}
\item \exEN{\href{http://www.wordreference.com/enfr/weird}{weird}} qui s'écrit phonétiquement \href{https://en.oxforddictionaries.com/definition/weird}{\phonm{wɪəd}}

  \begin{itemize}
  \item\exEN{He always has \href{https://youtu.be/fcdUXnt87ng}{weird} dreams that \href{https://youtu.be/FikYhD7bXYE}{nobody} understands.}
  \item\exFR{Il fait toujours des rêves bizarres que personne ne
      comprend.}
  \end{itemize}

  \youglish{weird}
  
\item \exEN{\href{http://www.wordreference.com/enfr/beer}{beer}} qui s'écrit
  phonétiquement
  \href{https://en.oxforddictionaries.com/definition/beer}{\phonm{bɪə}}

  \begin{itemize}
  \item\exEN{\href{https://youtu.be/71du169Hn90}{Football} supporters usually drink \href{https://youtu.be/I1fsk4k-bOs}{beer}.}
  \item\exFR{Les supporters de foot boivent habituellement de la
      bière (attention à consommer avec modération).}
  \end{itemize}
  
\item \exEN{\href{http://www.wordreference.com/enfr/near}{near}} qui s'écrit
  phonétiquement
  \href{https://en.oxforddictionaries.com/definition/near}{\phonm{nɪə}}. 

  \begin{itemize}
  \item\exEN{\href{https://youtu.be/0TZg_a6T-Cw}{UK} is \href{https://youtu.be/xIS9K-bNt3M}{near} from France.}
  \item\exFR{Le Royaume-Uni est proche de la France.}
  \end{itemize}

  \youglish{near}
  
\item \exEN{\href{http://www.wordreference.com/enfr/steer}{steer}} qui
  s'écrit phonétiquement
  \href{https://en.oxforddictionaries.com/definition/steer}{\phonm{stɪə}}

  \begin{itemize}
  \item\exEN{The politician \href{https://youtu.be/z\_vSRFODAxU}{steered} the conversation to a different
      \href{https://youtu.be/pjYVTe0mcGg}{topic}.}
  \item\exFR{L'homme politique a orienté la conversation vers un autre sujet.}
  \end{itemize}

  \youglish{steer}
  
\end{enumerate}

\newpage

\section{Le \son \phon{eə} qu'on devrait écrire \phon{ɛə}}\label{sec:eeteenv}

\diph{ɛ}{ə}{ɛə}{diphthong-7-7}{Le \son \phon{ɛ} a été étudié
  page~\pageref{sec:sone} et pour le \son \phon{ə} voir
  page~\pageref{sec:sonenv}.}{ee}
\flags
  \uks{https://youtu.be/Ff-MqM6Zb4Q}

\begin{enumerate}
\item \exEN{\href{http://www.wordreference.com/enfr/bear}{bear}} qui s'écrit
  phonétiquement
  \href{https://dictionary.cambridge.org/dictionary/english/bear}{\phonm{bɛə}}

  \begin{itemize}
  \item\exEN{This \href{https://youtu.be/bo_GZ3q_Ays}{noise} is difficult to \href{https://youtu.be/NJ6jv\_lPBN8}{bear}.}
  \item\exFR{Ce bruit est difficile à supporter.}
  \end{itemize}

  \youglish{bear}
  
\item \exEN{\href{http://www.wordreference.com/enfr/rare}{rare}} qui s'écrit
  phonétiquement
  \href{https://dictionary.cambridge.org/dictionary/english/rare}{\phonm{rɛə}}

  \begin{itemize}
  \item\exEN{The \href{https://youtu.be/a7nrDN15NPE}{consultant} is an expert in \href{https://youtu.be/hPncU3924fU}{rare} illnesses.}
  \item\exFR{Le médecin spécialiste est expert en maladies rares.}
  \end{itemize}

  \youglish{rare}
  
\item \exEN{\href{http://www.wordreference.com/enfr/there}{there}} qui
  s'écrit phonétiquement
  \href{https://dictionary.cambridge.org/dictionary/english/there}{\phonm{ðeər}}

  \begin{itemize}
  \item\exEN{My friend is always \href{https://youtu.be/fg9pkAYvrSM}{there} for me when I \href{https://youtu.be/rdUeq09cGJ0}{need} her.}
  \item\exFR{Mon amie est toujours là pour moi quand j'ai besoin
      d'elle.}
  \end{itemize}

  \youglish{there}
  
\item \exEN{\href{http://www.wordreference.com/enfr/care}{care}} qui s'écrit
  phonétiquement
  \href{https://dictionary.cambridge.org/dictionary/english/care}{\phonm{keər}}

  \begin{itemize}
  \item\exEN{\href{https://youtu.be/y1KIVZw7Jxk}{Babies} need constant \href{https://youtu.be/ClrSEz\_tBZw}{care}.}
  \item\exFR{Les bébés ont besoin d'une attention constante.}
  \end{itemize}

  \youglish{care}
  
\end{enumerate}

\begin{center}
  \begin{figure}[h]
    \centering
    \includegraphics[scale=.5]{../img/rp-diph}
    \caption[\exEN{RP Diphthongs}]{\exEN{RP Diphthongs}, Source : Wikipédia}
    \label{fig:front-vowels-in-the-mouth}
  \end{figure}
\end{center}

\newpage
\minitoc
