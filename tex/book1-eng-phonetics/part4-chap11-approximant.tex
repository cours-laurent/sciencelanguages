\chapter{Approximant (bloc d'air partiel, semblable à une voyelle)}
\label{chap:approx}

\speech{5}{consonnes \exFR{spirantes\CW{https://fr.wikipedia.org/wiki/Consonne_spirante}} (\exEN{approximant\CW{https://en.wikipedia.org/wiki/Approximant_consonant}})}

\newpage
\minitoc
\newpage

\section{Le \son~\phon{w}}\label{sec:w}

Ce \son a pour nom
technique\dyse{voiced-labio-velar-approximant} :

\begin{itemize}
\item \exEN{Voiced labio-velar Approximant\CW{https://en.wikipedia.org/wiki/Voiced_labio-velar_approximant}.}
\item \exFR{Consonne spirante labio-vélaire voisée\CW{https://fr.wikipedia.org/wiki/Consonne_spirante_labio-v\%C3\%A9laire_vois\%C3\%A9e}.}
\end{itemize}

\indicsound

\properukus{https://youtu.be/4MpDb-gTipY}{https://youtu.be/RW94L6606DE}

\begin{enumerate}
\item \exEN{\href{http://www.wordreference.com/enfr/one}{one}} qui s'écrit phonétiquement \href{https://en.oxforddictionaries.com/definition/one}{\phonm{wʌn}}

  \begin{itemize}
  \item\exEN{\href{https://youtu.be/aSNJ00iAZ7I}{One} never knows where to begin, so let's start with
      the number \href{https://youtu.be/jHRXlK2SnQ8}{one}.}
  \item\exFR{On ne sait jamais par où commencer, alors commençons
      par le numéro un.}
  \end{itemize}

  \youglish{one}

\item \exEN{\href{http://www.wordreference.com/enfr/queen}{queen}} qui s'écrit phonétiquement \href{https://en.oxforddictionaries.com/definition/queen}{\phonm{kwiːn}}

  \begin{itemize}
  \item\exEN{The \href{https://youtu.be/Jmd4OLzhQw0}{queen} chooses her \href{https://youtu.be/VYhbUyzUUrA}{entourage} very carefully.}
  \item\exFR{La reine choisit \son entourage très soigneusement.}
  \end{itemize}

  \youglish{queen}

\item \exEN{\href{http://www.wordreference.com/enfr/wall}{wall}} qui s'écrit phonétiquement \href{https://en.oxforddictionaries.com/definition/wall}{\phonm{wɔːl}}

  \begin{itemize}
  \item\exEN{The \href{https://youtu.be/SF_WqsmH-Lw}{shelf} is attached to the \href{https://youtu.be/BN5Z28Dfl7o}{wall}.}
  \item\exFR{L'étagère est fixée au mur.}
  \end{itemize}

  \youglish{wall}

\item \exEN{\href{http://www.wordreference.com/enfr/world}{world}} qui s'écrit phonétiquement \href{https://en.oxforddictionaries.com/definition/world}{\phonm{wəːld}}

  \begin{itemize}
  \item\exEN{The company entered the \href{https://youtu.be/fzDft0DZRUw}{world} market with great
      \href{https://youtu.be/KMp_EZLxAHc}{success}.}
  \item\exFR{L'entreprise est entrée sur le marché mondial avec
      grand succès.}
  \end{itemize}

  \youglish{world}

\end{enumerate}

\newpage

\section{Le \son~\phon{j}}\label{chap:j}

Ce \son a pour nom
technique\dyse{palatal-approximant} :

\begin{itemize}
\item \exEN{Palatal Approximant\CW{https://en.wikipedia.org/wiki/Palatal_approximant}.}
\item \exFR{Consonne spirante palatale voisée\CW{https://fr.wikipedia.org/wiki/Consonne_spirante_palatale_vois\%C3\%A9e}.}
\end{itemize}

\begin{center}
  \begin{figure}[h]
    \centering
    \includegraphics[scale=.5]{../img/lodge/approximants-j-lodge}
    \caption{\exEN{Approximant j}, source :~\cite{lodge}}
    \label{fig:approximant-j}
  \end{figure}
\end{center}

\indicsound

\properukus{https://youtu.be/XhqGU1WxOfc}{https://youtu.be/1Yo4BHIIBP8}

\begin{enumerate}
\item \exEN{\href{http://www.wordreference.com/enfr/beauty}{beauty}} qui s'écrit phonétiquement \href{https://en.oxforddictionaries.com/definition/beauty}{\phonm{ˈbjuːti}}

  \begin{itemize}
  \item\exEN{The actress is the \href{https://youtu.be/JYacDPOWsmE}{embodiment} of talent and \href{https://youtu.be/IzwWXNxFiyA}{beauty}.}
  \item\exFR{L'actrice est l'incarnation du talent et de la
      beauté.}
  \end{itemize}

  \youglish{beauty}

\item \exEN{\href{http://www.wordreference.com/enfr/few}{few}} qui s'écrit phonétiquement \href{https://en.oxforddictionaries.com/definition/few}{\phonm{fjuː}}

  \begin{itemize}
  \item\exEN{I gave my friend a \href{https://youtu.be/6E2hYDIFDIU}{few} tips to save \href{https://youtu.be/JkhX5W7JoWI}{money}.}
  \item\exFR{J'ai donné quelques conseils à mon ami pour
      économiser de l'argent.}
  \end{itemize}

  \youglish{few}

\item \exEN{\href{http://www.wordreference.com/enfr/usual}{usual}} qui s'écrit phonétiquement \href{https://en.oxforddictionaries.com/definition/usual}{\phonm{ˈjuːʒʊəl}}

  \begin{itemize}
  \item\exEN{My \href{https://youtu.be/IdlOj1n3vXA}{mother} made her \href{https://youtu.be/ThLRPCs8uzc}{usual} cake for my birthday.}
  \item\exFR{Ma mère a fait son gâteau traditionnel pour mon
      anniversaire.}
  \end{itemize}

  \youglish{usual}

\item \exEN{\href{http://www.wordreference.com/enfr/yellow}{yellow}} qui s'écrit phonétiquement \href{https://en.oxforddictionaries.com/definition/yellow}{\phonm{ˈjɛləʊ}}

  \begin{itemize}
  \item\exEN{We all live in a \href{https://youtu.be/m2uTFF\_3MaA}{yellow} \href{https://www.lacoccinelle.net/245633.html}{submarine}.}
  \item\exFR{Nous vivons tous dans un sous-marin jaune.}
  \end{itemize}

  \youglish{yellow}

\end{enumerate}

\newpage

\section{Le \son~\phon{r} (UK) et noté \phon{ɹ} (US)}\label{sec:r}

\hypertarget{r}{Ce \son} a pour nom
technique\dyse{alveolar-approximant} :

\begin{itemize}
\item \exEN{Alveolar Approximant\CW{https://en.wikipedia.org/wiki/Alveolar_and_postalveolar_approximants}.}
\item \exFR{Consonne spirante alvéolaire voisée\CW{https://fr.wikipedia.org/wiki/Consonne_spirante_alv\%C3\%A9olaire_vois\%C3\%A9e}.}
\end{itemize}

\begin{center}
  \begin{figure}[h]
    \centering
    \includegraphics[scale=.5]{../img/cpp/american-r-cpp}
    \caption{\exEN{American ɹ}, source :~\cite{collins}}
    \label{fig:r-us}
  \end{figure}
\end{center}

\indicsound

\properukus{https://youtu.be/exUcpYrZotc}{https://youtu.be/q5a2-KuHkBU}

\begin{enumerate}
\item \exEN{\href{http://www.wordreference.com/enfr/arrange}{arrange}} qui s'écrit phonétiquement \href{https://en.oxforddictionaries.com/definition/arrange}{\phonm{əˈreɪn(d)ʒ}}

  \begin{itemize}
  \item\exEN{We can \href{https://youtu.be/oD5RzpwbrIc}{arrange} another meeting if \href{https://youtu.be/L3xXxXu1Kfc}{necessary}.}
  \item\exFR{Nous pouvons organiser une autre réunion si
      nécessaire.}
  \end{itemize}

  \youglish{arrange}

\item \exEN{\href{http://www.wordreference.com/enfr/road}{road}} qui s'écrit phonétiquement \href{https://en.oxforddictionaries.com/definition/road}{\phonm{rəʊd}}

  \begin{itemize}
  \item\exEN{The \href{https://youtu.be/bO2xMNU9bTw}{road} passes \href{https://youtu.be/9z1A1R8RQZs}{through} the forest.}
  \item\exFR{La route passe par la forêt.}
  \end{itemize}
  
\item \exEN{\href{http://www.wordreference.com/enfr/sorry}{sorry}} qui s'écrit phonétiquement \href{https://en.oxforddictionaries.com/definition/sorry}{\phonm{ˈsɒri}}

  \begin{itemize}
  \item\exEN{I am \href{https://youtu.be/ahCwKDyS5OE}{sorry} for any inconvenience I \href{https://youtu.be/QstrRR031XE}{may} have caused.}
  \item\exFR{Je suis désolé pour tout inconvénient que j'ai pu causer.}
  \end{itemize}

  \youglish{sorry}
  
\item \exEN{\href{http://www.wordreference.com/enfr/wrong}{wrong}} qui s'écrit phonétiquement \href{https://en.oxforddictionaries.com/definition/wrong}{\phonm{rɒŋ}}

  \begin{itemize}
  \item\exEN{There are no \href{https://youtu.be/a5e0z1\_uwHY}{wrong} answers to this \href{https://youtu.be/ENBv2i88g6Y}{question}.}
  \item\exFR{Il n'y a pas de mauvaises réponses à cette question.}
  \end{itemize}

  \youglish{wrong}

\end{enumerate}

\newpage

\section{Le \son~\phon{l}}\label{sec:l}

Ce \son a pour nom
technique\dyse{alveolar-lateral-approximant} :

\begin{itemize}
\item \exEN{Alveolar Lateral Approximant\CW{https://en.wikipedia.org/wiki/Dental,_alveolar_and_postalveolar_lateral_approximants}.}
\item \exFR{Consonne spirante latérale alvéolaire voisée\CW{https://fr.wikipedia.org/wiki/Consonne_spirante_lat\%C3\%A9rale_alv\%C3\%A9olaire_vois\%C3\%A9e}.}
\end{itemize}

\indicsound

\properukus{https://youtu.be/sXKFT02-nKw}{https://youtu.be/JamM8TgB_AA}

\begin{enumerate}
\item \exEN{\href{http://www.wordreference.com/enfr/feel}{feel}} qui s'écrit phonétiquement \href{https://en.oxforddictionaries.com/definition/feel}{\phonm{fiːl}}
  
  \begin{itemize}
  \item\exEN{I \href{https://youtu.be/4k4SP01l6rY}{feel} \href{https://youtu.be/DuDeBcpLITQ}{good} because I \href{https://youtu.be/w_zhHE2CB3A}{slept} well.}
  \item\exFR{Je me sens bien car j'ai bien dormi.}
  \end{itemize}

  \youglish{feel}
  
\item \exEN{\href{http://www.wordreference.com/enfr/law}{law}} qui s'écrit phonétiquement \href{https://en.oxforddictionaries.com/definition/law}{\phonm{lɔː}}

  \begin{itemize}
  \item\exEN{I want to become a \href{https://youtu.be/mGbMwP8MDjg}{judge}, so I have to study \href{https://youtu.be/GEy6ThJwE3s}{law}.}
  \item\exFR{Je souhaite devenir juge, je dois donc étudier le droit.}
  \end{itemize}

  \youglish{law}
  
\item \exEN{\href{http://www.wordreference.com/enfr/light}{light}} qui s'écrit phonétiquement \href{https://en.oxforddictionaries.com/definition/light}{\phonm{lʌɪt}}

  \begin{itemize}
  \item\exEN{\href{https://youtu.be/ULHeRdgeT54}{Light} attracts \href{https://youtu.be/tSkCqj6T\_NQ}{moths}.}
  \item\exFR{La lumière attire les papillons de nuit.}
  \end{itemize}

  \youglish{light}

\item \exEN{\href{http://www.wordreference.com/enfr/valley}{valley}} qui s'écrit phonétiquement \href{https://en.oxforddictionaries.com/definition/valley}{\phonm{ˈvali}}

  \begin{itemize}
  \item\exEN{This beautiful \href{https://youtu.be/wQFgG\_HsI0w}{valley} is covered with \href{https://youtu.be/KzDl7nOCmyU}{flowers}.}
  \item\exFR{Cette superbe vallée est recouverte de fleurs.}
  \end{itemize}

  \youglish{valley}

\end{enumerate}

\newpage

\section{Le \son~\phon{ɫ}}\label{sec:chelou}

\hypertarget{dl}{Ce \son} a pour nom
technique\dyse{dark-l} :

\begin{itemize}
\item \exEN{Dark L.}
\item \exFR{L sombre.}
\end{itemize}

\indicsound

\properukus{https://youtu.be/rlw6YbzETfk}{https://youtu.be/U4En7vG1wV4}

\begin{enumerate}
\item \exEN{\href{http://www.wordreference.com/enfr/full}{full}} qui
  s'écrit phonétiquement
  \href{https://home.cc.umanitoba.ca/\~krussll/phonetics/narrower/dark-l.html}{\phonm{fʊɫ}
    or \phonm{fɫ}}

  \begin{itemize}
  \item\exEN{His \href{https://youtu.be/TLmlgCteCMw}{full} name appears on his \href{https://youtu.be/Q7oF-t9Ybv8}{passport}.}
  \item\exFR{Son nom complet figure sur son passeport.}
  \end{itemize}

  \youglish{full}
  
\item \exEN{\href{http://www.wordreference.com/enfr/flail}{flail}} qui s'écrit phonétiquement \href{https://home.cc.umanitoba.ca/\~krussll/phonetics/narrower/dark-l.html}{\phonm{fleɫ}}
  
  \begin{itemize}
  \item\exEN{In Europe, a rope \href{https://youtu.be/AGf7n7iUF\_k}{flail} has been tried with some
      \href{https://youtu.be/nvQKfdQvEBk}{success}.}
  \item\exFR{En Europe, un fléau de corde a fait l'objet d'essais
      qui ont été raisonnablement satisfaisants.}
  \end{itemize}

  \youglish{flail}

\item \exEN{\href{http://www.wordreference.com/enfr/little}{little}}
  qui s'écrit phonétiquement
  \href{https://home.cc.umanitoba.ca/\~krussll/phonetics/narrower/dark-l.html}{\phonm{ˈlɪɾɫ}}

  \begin{itemize}
  \item\exEN{The car \href{https://youtu.be/r-61yYjKHHc}{salesman} offers many options at \href{https://youtu.be/FokJGK639R4}{little} cost.}
  \item\exFR{Le vendeur de voitures propose de nombreuses options à faible coût.}
  \end{itemize}

  \youglish{little}

\item \exEN{\href{http://www.wordreference.com/enfr/milk}{milk}} qui s'écrit phonétiquement \href{https://home.cc.umanitoba.ca/\~krussll/phonetics/narrower/dark-l.html}{\phonm{mɪɫk}}
  
  \begin{itemize}
  \item\exEN{The farmer \href{https://youtu.be/XBOE2CC0YYY}{milks} his cows every \href{https://youtu.be/pv7zbiLVBQw}{morning}.}
  \item\exFR{Le fermier trait ses vaches tous les matins.}
  \end{itemize}

  \youglish{milk}

  
\end{enumerate}

\newpage
\minitoc
