\chapter{Affricate (plosive + fricative)}\label{chap:affricative}

\speech{2}{consonnes \exFR{affriquées\CW{https://fr.wikipedia.org/wiki/Consonne_affriqu\%C3\%A9e}} (\exEN{africate\CW{https://en.wikipedia.org/wiki/Affricate_consonant}})}

\newpage
\minitoc
\newpage

\section{Le \son~\phon{tʃ} }\label{sec:tss}

\hypertarget{tch}{Ce \son} a pour nom technique\dyse{voiceless-postalveolar-affricate} :% #1: lien
                                % vers le blog
  %
  \begin{itemize}%
  \item \exEN{Voiceless postalveolar affricate \CW{https://en.wikipedia.org/wiki/Voiceless_postalveolar_affricate}.}% #2: sound name, #3: wiki EN
  \item \exFR{Consonne affriquée palato-alvéolaire sourde \CW{https://fr.wikipedia.org/wiki/Consonne_affriqu\%C3\%A9e_palato-alv\%C3\%A9olaire_sourde}.}% #4: nom du son, #5: wiki FR sinon blog voir
                         % package ifthen pour gérer ça
  \end{itemize}%
  %
  \indicsound%
  %
  \properukus{https://youtu.be/6SreswdXlAk}{https://youtu.be/unfuGPc3iXo}% #6: UK YT, #7: US YT

\begin{enumerate}
\item \exEN{\href{http://www.wordreference.com/enfr/cheese}{cheese}} qui
  s'écrit phonétiquement
  \href{https://en.oxforddictionaries.com/definition/cheese}{\phonm{tʃiːz}}
  
  \begin{itemize}
  \item\exEN{The \href{https://youtu.be/xYyP9o8wXtc}{cheese} had an \href{https://youtu.be/uxBg3KSeZO8}{awful} smell.}
  \item\exFR{Le fromage dégageait une odeur horrible.}
  \end{itemize}

  \youglish{cheese}
  
\item \exEN{\href{http://www.wordreference.com/enfr/match}{match}} qui
  s'écrit phonétiquement
  \href{https://en.oxforddictionaries.com/definition/match}{\phonm{matʃ}}
  
  \begin{itemize}
  \item\exEN{The password \href{https://youtu.be/-o\_IoZdtbWs}{matches} the one in the \href{https://youtu.be/FR4QIeZaPeM}{database}.}
  \item\exFR{Le mot de passe correspond à celui de la base de données.}
  \end{itemize}

  \youglish{match}
  
\item \exEN{\href{http://www.wordreference.com/enfr/nature}{nature}} qui
  s'écrit phonétiquement
  \href{https://en.oxforddictionaries.com/definition/nature}{\phonm{ˈneɪtʃə}}
  
  \begin{itemize}
  \item\exEN{Preserving \href{https://youtu.be/K\_jwPJM0QSc}{nature} is a \href{https://youtu.be/wyRy8kowyM8}{matter} of public concern.}
  \item\exFR{Préserver la nature est une question de responsabilité publique.}
  \end{itemize}

  \youglish{nature}
  
\item \exEN{\href{http://www.wordreference.com/enfr/watch}{watch}} qui
  s'écrit phonétiquement
  \href{https://en.oxforddictionaries.com/definition/watch}{\phonm{wɒtʃ}}
  
  \begin{itemize}
  \item\exEN{When my \href{https://youtu.be/jskG0yVDMLk}{parents} go out I have to \href{https://youtu.be/Eya0daHX-Fw}{watch} my little
      sister.}
  \item\exFR{Quand mes parents sortent je dois surveiller ma petite soeur.}
  \end{itemize}

  \youglish{watch}
  
\end{enumerate}

\newpage

\section{Le \son~\phon{dʒ} }\label{sec:dj}

\hypertarget{dj}{Ce \son} a pour nom technique\dyse{voiced-postalveolar-affricate} :% #1: lien
                                % vers le blog
  %
  \begin{itemize}%
  \item \exEN{Voiced postalveolar affricate \CW{https://en.wikipedia.org/wiki/Voiced_postalveolar_affricate}.}% #2: sound name, #3: wiki EN
  \item \exFR{Consonne affriquée palato-alvéolaire voisée \CW{https://fr.wikipedia.org/wiki/Consonne_affriqu\%C3\%A9e_palato-alv\%C3\%A9olaire_vois\%C3\%A9e}.}% #4: nom du son, #5: wiki FR sinon blog voir
                         % package ifthen pour gérer ça
  \end{itemize}%
  %
  \indicsound%
  %
  \properukus{https://youtu.be/vqL9ivPb09A}{https://youtu.be/unfuGPc3iXo}% #6: UK YT, #7: US YT
  
\begin{enumerate}
\item \exEN{\href{http://www.wordreference.com/enfr/age}{age}} qui s'écrit
  phonétiquement
  \href{https://en.oxforddictionaries.com/definition/age}{\phonm{eɪdʒ}}
  
  \begin{itemize}
  \item\exEN{\href{https://youtu.be/wKU5khnuY\_Y}{Age} and inactivity reduce joint \href{https://youtu.be/sYrIMdOBHkg}{mobility}.}
  \item\exFR{L'âge et l'inactivité réduisent la mobilité articulaire.}
  \end{itemize}

  \youglish{age}
  
\item \exEN{\href{http://www.wordreference.com/enfr/joy}{joy}} qui s'écrit
  phonétiquement
  \href{https://en.oxforddictionaries.com/definition/joy}{\phonm{dʒɔɪ}}
  
  \begin{itemize}
  \item\exEN{The \href{https://youtu.be/TyYIxGL2p6c}{joy} of \href{https://www.amazon.fr/gp/product/B013RQ72R2/ref=as\_li\_tl?ie=UTF8\&camp=1642\&creative=6746\&creativeASIN=B013RQ72R2\&linkCode=as2\&tag=wwwbecomefree-21\&linkId=e8ebecacb076d66dd3e5a435789050d5}{phonetics}.}
  \item\exFR{La joie de la phonétique.}
  \end{itemize}

  \youglish{joy}
  
\item \exEN{\href{http://www.wordreference.com/enfr/juggle}{juggle}} qui
  s'écrit phonétiquement
  \href{https://en.oxforddictionaries.com/definition/juggle}{\phonm{ˈdʒʌɡ(ə)l}}
  
  \begin{itemize}
  \item\exEN{He can \href{https://youtu.be/kCt1bmSASCI}{juggle} with five \href{https://youtu.be/wLxMbzWm5Es}{balls}.}
  \item\exFR{Il peut jongler avec cinq balles.}
  \end{itemize}

  \youglish{juggle}
  
\item \exEN{\href{http://www.wordreference.com/enfr/soldier}{soldier}} qui
  s'écrit phonétiquement
  \href{https://en.oxforddictionaries.com/definition/soldier}{\phonm{ˈsəʊldʒə}}
  
  \begin{itemize}
  \item\exEN{The \href{https://youtu.be/ucoSdNM2Atw}{soldier} defused the \href{https://youtu.be/cYyQcWPywHo}{bomb}.}
  \item\exFR{Le soldat a désamorcé la bombe.}
  \end{itemize}

  \youglish{soldier}
  
\end{enumerate}

\newpage
\minitoc
\newpage

