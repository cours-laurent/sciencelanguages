\chapter{Back Vowels (langue vers l'arrière)}\label{chap:backvow}

\speech{4}{voyelles postérieures\CW{https://fr.wikipedia.org/wiki/Voyelle_post\%C3\%A9rieure}}

\newpage
\minitoc
\newpage

\section{Le \son \phon{uː} }\label{sec:ulong}

\hypertarget{ulong}{Ce \son} a pour nom technique\dyse{close-back-rounded-vowel} :

\begin{itemize}
\item \exEN{Close Back Rounded Vowel\CW{https://en.wikipedia.org/wiki/Close_back_rounded_vowel}.}
\item \exFR{Voyelle fermée postérieure arrondie\CW{https://fr.wikipedia.org/wiki/Voyelle_ferm\%C3\%A9e_post\%C3\%A9rieure_arrondie}.}
\end{itemize}

\begin{center}
  \begin{figure}[h]
    \centering
    \includegraphics{../img/arrondies-wikipedia.png}
    \caption{Voyelles arrondies, source : Wikipédia}
    \label{fig:voy-arr}
  \end{figure}
\end{center}

\indicsound

\properukus{https://youtu.be/qPB0Ajjs7nE}{https://youtu.be/lkM6CKBM2ns}

\begin{enumerate}
\item \exEN{\href{http://www.wordreference.com/enfr/too}{too}} qui s'écrit
  phonétiquement
  \href{https://en.oxforddictionaries.com/definition/too}{\phonm{tuː}}
  
  \begin{itemize}
  \item\exEN{I like to speak \href{https://youtu.be/H3r9bOkYW9s}{English}, and you? Me \href{https://youtu.be/RaveinO4\_vs}{too}.}
  \item\exFR{J'aime parler Anglais, et toi ? Moi aussi.}
  \end{itemize}
  
\item \exEN{\href{http://www.wordreference.com/enfr/few}{few}} qui s'écrit
  phonétiquement
  \href{https://en.oxforddictionaries.com/definition/few}{\phonm{fjuː}}
  
  \begin{itemize}
  \item\exEN{\href{https://youtu.be/r3TaGhdqEiA}{Few} people \href{https://youtu.be/kIzFz9T5rhI}{understand} the key role of \href{https://www.youtube.com/watch?v=dtf8zGQj9GY&list=PLfLdA1jGDSu6exdSf9yQJWKgNqPviO4b4}{phonetics}.}
  \item\exFR{Peu de gens comprennent le rôle clé de la phonétique.}
  \end{itemize}
  
\item \exEN{\href{http://www.wordreference.com/enfr/rule}{rule}} qui s'écrit
  phonétiquement
  \href{https://en.oxforddictionaries.com/definition/rule}{\phonm{ruːl}}
  
  \begin{itemize}
  \item\exEN{\href{https://youtu.be/rStL7niR7gs}{Do you want} \href{https://amzn.to/2GneMiu}{to rule?}}
  \item\exFR{Voulez-vous diriger ?}
  \end{itemize}
  
\item \exEN{\href{http://www.wordreference.com/enfr/lose}{lose}} qui s'écrit
  phonétiquement
  \href{https://en.oxforddictionaries.com/definition/lose}{\phonm{luːz}}
  
  \begin{itemize}
  \item\exEN{You \href{https://youtu.be/UNcCTgA5lzo}{lose} \href{https://amzn.to/2GnOFbd}{the game} this time, do you want to \href{https://youtu.be/-FkvBA3U5lg}{try again}?}
  \item\exFR{Vous avez perdu la partie cette fois, voulez-vous
      essayer à nouveau ?}
  \end{itemize}
  
\end{enumerate}

\newpage

\section{Le \son \phon{ʊ} }\label{sec:omega}

\hypertarget{omega}{Ce \son} a pour nom technique\dyse{near-close-near-back-rounded-vowel} :

\begin{itemize}
\item \exEN{Near-Close Near-Back Rounded Vowel\CW{https://en.wikipedia.org/wiki/Near-close_near-back_rounded_vowel}.}
\item \exFR{Voyelle pré-fermée postérieure arrondie\CW{https://fr.wikipedia.org/wiki/Voyelle_pr\%C3\%A9-ferm\%C3\%A9e_post\%C3\%A9rieure_arrondie}.}
\end{itemize}

\begin{center}
  \begin{figure}[h]
    \centering
    \includegraphics{../img/arrondies-wikipedia.png}
    \caption{Voyelles arrondies, source : Wikipédia}
    \label{fig:voy-arr}
  \end{figure}
\end{center}

\indicsound

\properukus{https://youtu.be/5lOF-zRg8x0}{https://youtu.be/moLTR-dLQQY}

\begin{enumerate}
\item \exEN{\href{http://www.wordreference.com/enfr/good}{good}} qui
  s'écrit phonétiquement
  \href{https://en.oxforddictionaries.com/definition/good}{\phonm{ɡʊd}}
  
  \begin{itemize}
  \item\exEN{Your \href{https://youtu.be/GihybX7JyG4}{book} is \href{https://youtu.be/o3TQSaqHBtM}{good}.}
  \item\exFR{Votre le livre est bon.}
  \end{itemize}
  
\item \exEN{\href{http://www.wordreference.com/enfr/put}{put}} qui s'écrit
  phonétiquement
  \href{https://en.oxforddictionaries.com/definition/put}{\phonm{pʊt}}
  
  \begin{itemize}
  \item\exEN{\href{https://youtu.be/BSpoa7TsiD0}{Put} your energy in \href{https://youtu.be/bLMwe6kFFg0}{something you like}.}
  \item\exFR{Mettez votre énergie dans quelque chose que vous
      aimez.}
  \end{itemize}
  
\item \exEN{\href{http://www.wordreference.com/enfr/would}{would}} qui
  s'écrit phonétiquement
  \href{https://en.oxforddictionaries.com/definition/would}{\phonm{wʊd}}
  
  \begin{itemize}
  \item\exEN{\href{https://youtu.be/wRSNm3pr100}{Would} you like to \href{https://youtu.be/tiDvgH8yNhg}{drink} something?}
  \item\exFR{Voulez-vous boire quelque chose ?}
  \end{itemize}
  
\item \exEN{\href{http://www.wordreference.com/enfr/look}{look}} qui s'écrit
  phonétiquement
  \href{https://en.oxforddictionaries.com/definition/look}{\phonm{lʊk}}

  \begin{itemize}
  \item\exEN{\href{https://youtu.be/b4xcpMCPhfE}{Look} at \href{https://youtu.be/_SXm5nnzZJk}{this}!}
  \item\exFR{Regarde ça !}
  \end{itemize}
  
\end{enumerate}

\newpage

\section{Le \son \phon{ɔː} }\label{sec:oouvert}

\hypertarget{oouvert}{Ce \son} a pour nom technique\dyse{open-mid-back-rounded-vowel} :

\begin{itemize}
\item \exEN{Open-Mid Back Rounded Vowel\CW{https://en.wikipedia.org/wiki/Open-mid_back_rounded_vowel}.}
\item \exFR{Voyelle mi-ouverte postérieure arrondie\CW{https://fr.wikipedia.org/wiki/Voyelle_mi-ouverte_post\%C3\%A9rieure_arrondie}.}
\end{itemize}

\begin{center}
  \begin{figure}[h]
    \centering
    \includegraphics{../img/arrondies-wikipedia.png}
    \caption{Voyelles arrondies, source : Wikipédia}
    \label{fig:voy-arr}
  \end{figure}
\end{center}

\indicsound

\properukus{https://youtu.be/Bc1tCtP2ZSg}{https://youtu.be/pr_KAu-_Hmo}

\begin{enumerate}
\item \exEN{\href{http://www.wordreference.com/enfr/pork}{pork}} qui s'écrit
  phonétiquement
  \href{https://en.oxforddictionaries.com/definition/pork}{\phonm{pɔːk}}
  
  \begin{itemize}
  \item\exEN{Do you \href{https://youtu.be/ZJeI2VIEDY8}{eat} \href{https://youtu.be/WqTJbyfewzw}{pork}?}
  \item\exFR{Mangez-vous du porc ?}
  \end{itemize}

  \youglish{pork}
  
\item \exEN{\href{http://www.wordreference.com/enfr/law}{law}} qui s'écrit
  phonétiquement
  \href{https://en.oxforddictionaries.com/definition/law}{\phonm{lɔː}}

  \begin{itemize}
  \item\exEN{\href{https://youtu.be/us5CUAsH0U0}{Hackers like to say: code is law.}}
  \item\exFR{Les hackers aiment dire que le code est la loi.}
  \end{itemize}

  \youglish{law}
  
\item \exEN{\href{http://www.wordreference.com/enfr/taught}{taught}} qui
  s'écrit phonétiquement
  \href{https://en.oxforddictionaries.com/definition/taught}{\phonm{tɔːt}}
  
  \begin{itemize}
  \item\exEN{I \href{https://youtu.be/U2BG2\_K2fGk}{taught} you how to write \href{https://youtu.be/o8KppNXfx2k}{English phonetics} yesterday.}
  \item\exFR{Hier je t'ai enseigné comment écrire la phonétique anglaise.}
  \end{itemize}

  \youglish{taught}
  
\item \exEN{\href{http://www.wordreference.com/enfr/thought}{thought}} qui
  s'écrit phonétiquement
  \href{https://en.oxforddictionaries.com/definition/thought}{\phonm{θɔːt}}
  
  \begin{itemize}
  \item\exEN{\href{https://youtu.be/XO273RIGifY}{Tell me} your \href{https://youtu.be/8kR-GDbYHhc}{thoughts}.}
  \item\exFR{Raconte-moi tes pensées.}
  \end{itemize}

  \youglish{thought}
  
\end{enumerate}

\newpage

\section{Le \son \phon{ɒ}}\label{sec:oa}

\hypertarget{oa}{Ce \son} a pour nom technique\dyse{open-back-rounded-vowel} :

\begin{itemize}
\item \exEN{Open Back Rounded Vowel\CW{https://en.wikipedia.org/wiki/Open_back_rounded_vowel}.}
\item \exFR{Voyelle ouverte postérieure arrondie\CW{https://fr.wikipedia.org/wiki/Voyelle_ouverte_post\%C3\%A9rieure_arrondie}.}
\end{itemize}

\begin{center}
  \begin{figure}[h]
    \centering
    \includegraphics{../img/arrondies-wikipedia.png}
    \caption{Voyelles arrondies, source : Wikipédia}
    \label{fig:voy-arr}
  \end{figure}
\end{center}

\indicsound

\begin{center}
  \uks{https://youtu.be/A3l-yWQfIW4}
\end{center}


\begin{enumerate}
\item \exEN{\href{http://www.wordreference.com/enfr/got}{got}} qui s'écrit
  phonétiquement
  \href{https://en.oxforddictionaries.com/definition/got}{\phonm{ɡɒt}}

  \begin{itemize}
  \item\exEN{I \href{https://youtu.be/Bo09BiPb24Y}{got} you. (slang: \href{https://youtu.be/EWRaAbVUkjA}{Gotcha})}
  \item\exFR{Je t'ai eu.} (argot : Gotcha)
  \end{itemize}

  \youglish{got}
  
\item \exEN{\href{http://www.wordreference.com/enfr/watch}{watch}} qui
  s'écrit phonétiquement
  \href{https://en.oxforddictionaries.com/definition/watch}{\phonm{wɒtʃ}}
  
  \begin{itemize}
  \item\exEN{\href{https://youtu.be/qOs8MagOfwg}{Watch} this video \href{https://youtu.be/cnIanivwpSU}{carefully}.}
  \item\exFR{Regardez attentivement cette vidéo.}
  \end{itemize}

  \youglish{watch}
  
\item \exEN{\href{http://www.wordreference.com/enfr/rob}{rob}} qui s'écrit
  phonétiquement
  \href{https://en.oxforddictionaries.com/definition/rob}{\phonm{rɒb}}

  \begin{itemize}
  \item\exEN{Are you planning to \href{https://youtu.be/X3uZ0Gf104A}{rob} a bank? I \href{https://youtu.be/CILQJnsD128}{discourage} you to do
      that.}
  \item\exFR{Êtes-vous en train d'envisager de cambrioler une
      banque~? Je vous déconseille de faire ça.}
  \end{itemize}

  \youglish{rob}
  
\item \exEN{\href{http://www.wordreference.com/enfr/top}{top}} qui s'écrit
  phonétiquement
  \href{https://en.oxforddictionaries.com/definition/top}{\phonm{tɒp}}

  \begin{itemize}
  \item\exEN{\href{https://youtu.be/gPaD513xWOY}{Top} videos are sometime very \href{https://youtu.be/M9i2HAE-ZSw}{boring}.}
  \item\exFR{Les vidéos de top sont parfois très ennuyeuses.}
  \end{itemize}

  \youglish{top}
  
\end{enumerate}

\newpage
\minitoc
\newpage

