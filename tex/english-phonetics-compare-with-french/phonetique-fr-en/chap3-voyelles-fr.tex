\chapter{Voyelles}\label{chap:voy}

Que celui ou celle qui a dit <<~consonne~>> juste après avoir lu le
titre du chapitre sorte immédiatement\footnote{Toute référence à une
  émission de la télévision française serait purement fortuite, un
  coup du hasard si j'ose pousser légèrement la prose.}!\par
Plus sérieusement, lorsqu'on regroupe les voyelles\footnote{Puisque ce
  livre parle de phonétique on s'occupe des sons donc de l'oral et pas
  de l'écrit (donc on se fiche qu'il y ait six voyelles écrites).} on
découvre deux grandes catégories : les voyelles dites \exFR{Orales} et
les voyelles dites \exFR{Nasales}\footnote{Que les anglophones ont du
  mal à prononcer correcement.}.

On appelle voyelle un son produit par les vibrations des cordes
vocales qui s'échappe sans avoir été arrêté nul part dans le canal
vocal.  La langue française parlée compte
ainsi seize voyelles, 12 orales et 4 nasales.

\newpage
\minitoc
\newpage

\section{Voyelles orales}\label{sec:orales}

Les voyelles sont dites \underline{orales} lorsque le souffle qui les
produit s'échappe uniquement par la bouche :

\begin{center}
  \begin{tabular}[b]{|*{4}{c|}}
    \sonref{sona}{a}& \sonref{sonɑ}{ɑ}& \sonref{sone}{e}&
    \sonref{sonɛ}{ɛ}\\ \hline
    \sonref{sonə}{ə}& \sonref{soni}{i}& \sonref{sono}{o}&
    \sonref{sonɔ}{ɔ}\\ \hline
    \sonref{sonø}{ø}& \sonref{sonœ}{œ}& \sonref{sonu}{u}&  \sonref{sony}{y}
  \end{tabular}
\end{center}



\subsection{\son{a}}\label{subsec:afr}

\sonancre{sona}que l'on trouve dans les mots \exFR{arrêt} (~\exPH{aʀɛ}~),
\exFR{bar} (~\exPH{baʀ}~), \exFR{chat} (~\exPH{ʃa}~), et bien
d'autres. 

Observons un exemple pour chacun d'entre eux et voyons leurs
traductions anglaises :\par

\begin{enumerate}
\item \exFR{arrêt} (~\exPH{aʀɛ}~)
  \begin{itemize}
  \item \exFR{Ce matin j'ai attendu à l'arrêt de
      bus pendant dix  minutes.}
    \item \exEN{This morning I waited at the bus stop for ten minutes.}
    \end{itemize}
    Le mot \exEN{stop} est même utilisé dans la langue française
    au point que certaines personnes le conjuguent (~\textquote[][.]{\exFR{J'ai stoppé net quand j'ai vu la voiture arriver}}~).
    Cependant en anglais on
    prononce
    \href{https://en.oxforddictionaries.com/definition/stop}{\exPH{stɒp}}
    alors qu'en français on le prononce
    \href{http://www.wordreference.com/fren/stop}{\exPH{stɔp}}\footnote{Les
      sons [ɒ] et [ɔ] sont bien distincts}.
\item \exFR{bar} (~\exPH{baʀ}~)
  \begin{itemize}
  \item \exFR{Dans la rue où j'ai grandi il y
      avait un bar juste en bas de chez moi.}
  \item \exEN{In the street where I grew up there was a bar just down from my house.}
  \end{itemize}
    Attention ! Les deux mots sont homographes\footnote{Deux mots sont
      dits homographes s'ils s'écrivent de la même manière (avec les
      mêmes lettres).} mais ils ne sont pas homophones\footnote{Deux
      mots sont dits homophones s'ils se prononcent de la même façon
      (même son).} ! En effet, en anglais on prononce
    \href{https://en.oxforddictionaries.com/definition/bar}{\exPH{bɑː}}
    alors qu'en français on dit \href{http://www.wordreference.com/fren/bar}{\exPH{baʀ}}; voilà un exemple qui illustre parfaitement l'intérêt
    d'étudier la phonétique.
  \item \exFR{chat} (~\exPH{ʃa}~)
    \begin{itemize}
    \item \exFR{Le chat des voisins vient souvent nous rendre visite.}
    \item \exEN{The neighbor's cat often comes to visit us.}
    \end{itemize}
    Ici on a d'une part \exFR{chat} qui se prononce
    \href{http://www.wordreference.com/fren/chat}{\exPH{ʃa}} en français et
    \exEN{cat} en anglais qui se prononce
    \href{http://www.wordreference.com/enfr/cat}{\exPH{kæt}}. Attention !
    Le verbe anglais \exEN{to chat}\footnote{\exEN{To chat} signifie
      bavarder.} se prononce
    \href{https://en.oxforddictionaries.com/definition/chat}{\exPH{tʃæt}}. 
\end{enumerate}

\subsection{\son{ɑ}}\label{subsec:ɑ}

\sonancre{sonɑ}que l'on trouve dans les mots \exFR{bâton} (~\exPH{bɑtɔ̃ }~),
\exFR{mât} (~\exPH{mɑ}~), \exFR{pâte} (~\exPH{pɑt}~), et bien d'autres.

Observons un exemple pour chacun d'entre eux et voyons leurs
traductions anglaises :\par

\begin{enumerate}
\item \exFR{bâton} (~\exPH{bɑtɔ̃ }~)
  \begin{itemize}
  \item \exFR{Souvent dans la liste des fournitures scolaires on demande
      un bâton de colle.}
    \item \exEN{Often in the list of school supplies we ask for a stick of glue.}
    \end{itemize}
    Le mot \exEN{stick}\footnote{Le mot \exEN{stick} signifie \exFR{bâton} et
      peut aussi vouloir dire \exFR{matraque}.} se prononce en anglais \href{https://en.oxforddictionaries.com/definition/stick}{\exPH{stɪk}}.
\item \exFR{mât} (~\exPH{mɑ}~)
  \begin{itemize}
  \item \exFR{Avant l'invention du moteur les bateaux étaient tous à
      voile et par conséquent il fallait bien que quelqu'un grimpe au
      sommet du mât pour l'accrocher.}
  \item \exEN{Before the invention of the engine the boats were all sailing and therefore it was necessary that someone climbs to the top of the mast to hang it.}
  \end{itemize}
  En anglais le mot \exEN{mast} se prononce
  \href{https://en.oxforddictionaries.com/definition/mast}{\exPH{mɑːst}}. En
  quelques sortes on peut dire qu'il est une prolongation sonore de son
  équivalent français.

  \item \exFR{pâte} (~\exPH{pɑt}~)
    \begin{itemize}
    \item \exFR{Pour faire des crêpes il faut d'abord préparer la pâte.}
    \item \exEN{To make pancakes it is necessary to first prepare the dough.}
    \end{itemize}
    On notera que le mot français \exFR{pâte}\footnote{Vous pouvez consulter
      directement les traductions disponibles sur \url{http://www.wordreference.com/fren/p\%C3\%A2te}} admet de nombreux
    équivalents anglais \exEN{pastry} (pour les tartes ou
    pâtisseries), \exEN{dough} (pour le pain et les crêpes),
    \exEN{base} (pour les pizzas) et j'en passe et des meilleurs ! 
\end{enumerate}

\subsection{\son{e}}\label{subsec:efr}

\sonancre{sone}que l'on trouve par exemple dans les mots \exFR{créer}
(~\exPH{kʀee}~), \exFR{pré} (~\exPH{pʀe}~), \exFR{république} (~\exPH{ʀepyblik}~), et bien d'autres.

Observons un exemple pour chacun d'entre eux et voyons leurs
traductions anglaises :\par

\begin{enumerate}
\item \exFR{créer} (~\exPH{kʀee}~)
  \begin{itemize}
  \item \exFR{Créer est un acte souvent solitaire mais qui s'appuie
      également sur les travaux des autres.}
    \item \exEN{Creating is often a lonely act, but it is also based on the work of others.}
    \end{itemize}
    Le verbe anglais \exEN{create} se prononce
    \href{https://en.oxforddictionaries.com/definition/create}{\exPH{kriːˈeɪt}}.
\item \exFR{pré} (~\exPH{pʀe}~)
  \begin{itemize}
  \item \exFR{Dans le pré il y a les vaches qui broutent de l'herbe.}
  \item \exEN{In the meadow there are cows grazing grass.}
  \end{itemize}
  

\item \exFR{république} (~\exPH{ʀepyblik}~)
  \begin{itemize}
  \item \exFR{La France est une république.}
  \item \exEN{France is a republic.}
  \end{itemize}
  Même si les deux mots se ressemblent beaucoup graphiquement, ils
  ne se prononcent pas de la même manière. En effet, le mot
  \exEN{republic} se prononce \href{https://en.oxforddictionaries.com/definition/republic}{\exPH{rɪˈpʌblɪk}}. 
\end{enumerate}

  
\subsection{\son{ɛ}}\label{subsec:ɛfr}

\sonancre{ɛ}que l'on trouve par exemple dans les mots \exFR{carrière}
(~\exPH{kaʀjɛʀ}~), \exFR{mère} (~\exPH{mɛʀ}~), \exFR{taire} (~\exPH{tɛʀ}~), et bien d'autres.

Observons un exemple pour chacun d'entre eux et voyons leurs
traductions anglaises :\par

\begin{enumerate}
\item \exFR{carrière} (~\exPH{kaʀjɛʀ}~)
  \begin{itemize}
  \item \exFR{Généralement les sportifs de haut niveau finissent leur
      carrière avant 40 ans.}
    \item \exEN{Generally top athletes finish their career before 40 years.}
    \end{itemize}
    Le mot français \exFR{carrière} est polysémique\footnote{Il a
      plusieurs significations différentes.} mais en anglais chaque
    sens du mot français\footnote{Vous pouvez en consulter une liste
      ici \url{http://www.wordreference.com/fren/carri\%C3\%A8re}.}  correspond à un mot différent. Par exemple
    \exEN{quarry}
    (~\href{https://en.oxforddictionaries.com/definition/quarry}{
      \exPH{ˈkwɒrɪ}}~) pour les carrières de pierres ou encore \exEN{career}
    (~\href{https://en.oxforddictionaries.com/definition/career}{\exPH{kəˈrɪə}}~)
    pour la carrière professionnelle.
\item \exFR{mère} (~\exPH{mɛʀ}~)
  \begin{itemize}
  \item \exFR{C'est à toutes les mères que je dédicace ce livre.}
  \item \exEN{It is to all the mothers that I dedicate this book.}
  \end{itemize}

  
\item \exFR{taire} (~\exPH{tɛʀ}~)
  \begin{itemize}
  \item \exFR{Parfois pour plaire il vaut mieux se taire.}
  \item \exEN{Sometimes to please it is better to be quiet.}
  \end{itemize}
   
\end{enumerate}

\subsection{\son{ə}}\label{subsec:əfr}

\sonancre{sonə}que l'on trouve par exemple dans les mots
\exFR{appartement} (~\exPH{apaʀtəmɑ̃ }~), \exFR{chemin} (~\exPH{ʃ(ə)mɛ̃ }~), \exFR{venin} (~\exPH{vənɛ̃ }~), et bien d'autres.

Observons un exemple pour chacun d'entre eux et voyons leurs
traductions anglaises :\par

\begin{enumerate}
\item \exFR{appartement} (~\exPH{apaʀtəmɑ̃ }~)
  \begin{itemize}
  \item \exFR{En ce moment je loue un appartement.}
    \item \exEN{At the moment I am renting an apartment.}
    \end{itemize}
    
\item \exFR{chemin} (~\exPH{ʃ(ə)mɛ̃ }~)
  \begin{itemize}
  \item \exFR{Le chemin de gauche permet de mieux voir la nature.}
  \item \exEN{The path on the left makes it easier to see nature.}
  \end{itemize}

  
\item \exFR{venin} (~\exPH{vənɛ̃ }~)
  \begin{itemize}
  \item \exFR{Gare aux serpents qui ont du venin.}
  \item \exEN{Beware of snakes with venom.}
  \end{itemize}
   
\end{enumerate}

\subsection{\son{i}}\label{subsec:ifr}

\sonancre{soni}que l'on trouve par exemple dans les mots
\exFR{cri} (~\exPH{kʀi}~), \exFR{ravi} (~\exPH{ʀavi}~), \exFR{tri} (~\exPH{tʀi}~), et bien d'autres.

Observons un exemple pour chacun d'entre eux et voyons leurs
traductions anglaises :\par

\begin{enumerate}
\item \exFR{cri} (~\exPH{kʀi}~)
  \begin{itemize}
  \item \exFR{Je marchais seul dans la rue quand soudain j'entendis un
      cri !}
    \item \exEN{I walked alone in the street when suddenly I heard a scream!}
    \end{itemize}
    
\item \exFR{ravi} (~\exPH{ʀavi}~)
  \begin{itemize}
  \item \exFR{Je suis ravi de faire votre connaissance.}
  \item \exEN{I'm delighted to meet you.}
  \end{itemize}

  
\item \exFR{tri} (~\exPH{tʀi}~)
  \begin{itemize}
  \item \exFR{De temps en temps il est bon de faire du tri dans les
      fichiers et dossiers de l'ordinateur.}
  \item \exEN{From time to time it is good to sort through the files
      and folders of the computer.}
  \end{itemize}
   
\end{enumerate}

\subsection{\son{o}}\label{subsec:ofr}

\sonancre{sono}que l'on trouve par exemple dans les mots
 \exFR{arroser} (~\exPH{aʀoze}~), \exFR{prose} (~\exPH{pʀoz}~), \exFR{rose} (~\exPH{ʀoz}~), et bien d'autres.

Observons un exemple pour chacun d'entre eux et voyons leurs
traductions anglaises :\par

\begin{enumerate}
\item  \exFR{arroser} (~\exPH{aʀoze}~)
  \begin{itemize}
  \item \exFR{N'oublie pas d'arroser les plantes.}
    \item \exEN{Do not forget to water the plants.}
    \end{itemize}
    
\item \exFR{prose} (~\exPH{pʀoz}~)
  \begin{itemize}
  \item \exFR{La prose est un style littéraire.}
  \item \exEN{Prose is a literary style.}
  \end{itemize}

  
\item \exFR{rose} (~\exPH{ʀoz}~)
  \begin{itemize}
  \item \exFR{À Paris, aux terrasses de café, il y a souvent des vendeurs de roses qui viennent vous demander d'en acheter.}
  \item \exEN{In Paris, on the terraces of coffee, there are often vendors of roses who come to ask you to buy some.}
  \end{itemize}
   
\end{enumerate}


\subsection{\son{ɔ}}\label{subsec:ɔfr}

\sonancre{sonɔ} que l'on trouve par exemple dans les mots
\exFR{cote} (~\exPH{kɔt}~), \exFR{note} (~\exPH{nɔt}~), \exFR{vote} (~\exPH{vɔt}~), et bien d'autres.

Observons un exemple pour chacun d'entre eux et voyons leurs
traductions anglaises :\par

\begin{enumerate}
\item  \exFR{cote} (~\exPH{kɔt}~)
  \begin{itemize}
  \item \exFR{Le président n'a plus la même cote de popularité qu'auparavant.}
    \item \exEN{The president no longer has the same popularity rating as before.}
    \end{itemize}
    
\item \exFR{note} (~\exPH{nɔt}~)
  \begin{itemize}
  \item \exFR{La question de la note est souvent un sujet qui fâche.}
  \item \exEN{The question of the note is often an annoying subject.}
  \end{itemize}

\item \exFR{vote} (~\exPH{vɔt}~)
  \begin{itemize}
  \item \exFR{Une chose est sûre, le résultat du vote ne fera pas que
      des heureux.}
  \item \exEN{One thing is certain, the result of the vote will not only make people happy.}
  \end{itemize}
   
\end{enumerate}
         
\subsection{\son{ø}}\label{subsec:øfr}

\sonancre{sonø}que l'on trouve par exemple dans les mots
\exFR{cieux} (~\exPH{sjø}~), \exFR{lieu} (~\exPH{ljø}~), \exFR{pieu} (~\exPH{pjø}~), et bien d'autres.

Observons un exemple pour chacun d'entre eux et voyons leurs
traductions anglaises :\par

\begin{enumerate}
\item  \exFR{cieux} (~\exPH{sjø}~)
  \begin{itemize}
  \item \exFR{Les anciens regardaient vers les cieux dans l'espoir d'y
    trouver des réponses à leurs questionnements.}
    \item \exEN{The elders looked to the heavens in the hope of finding answers to their questions.}
    \end{itemize}
    
\item \exFR{lieu} (~\exPH{ljø}~)
  \begin{itemize}
  \item \exFR{Voici notre lieu de rendez-vous.}
  \item \exEN{Here is our meeting place.}
  \end{itemize}

\item \exFR{pieu} (~\exPH{pjø}~)
  \begin{itemize}
  \item \exFR{Dans les films de vampires on préconise de leur planter
      un pieu dans le c{\oe}ur pour s'en débarrasser.}
  \item \exEN{In vampire movies it is recommended to plant a stake in the heart to get rid of it.}
  \end{itemize}
   
\end{enumerate}         

\subsection{\son{œ}}\label{subsec:œfr}

\sonancre{sonœ}que l'on trouve par exemple dans les mots
\exFR{frayeur} (~\exPH{fʀɛjœʀ}~), \exFR{peur} (~\exPH{pœʀ}~), \exFR{tailleur} (~\exPH{tɑjœʀ}~), et bien d'autres.

Observons un exemple pour chacun d'entre eux et voyons leurs
traductions anglaises :\par

\begin{enumerate}
\item \exFR{frayeur} (~\exPH{fʀɛjœʀ}~)
  \begin{itemize}
  \item \exFR{Tu m'as fait une frayeur toute à l'heure quand tu as
      traversé la route sans regarder.}
    \item \exEN{You scared me earlier when you crossed the road without looking.}
    \end{itemize}
    
\item \exFR{peur} (~\exPH{pœʀ}~)
  \begin{itemize}
  \item \exFR{Est-ce que tu aimes avoir peur au cinéma ?}
  \item \exEN{Do you like to be afraid in the cinema?}
  \end{itemize}

\item \exFR{tailleur} (~\exPH{tɑjœʀ}~)
  \begin{itemize}
  \item \exFR{Pour méditer il est recommandé de s'asseoir en tailleur.}
  \item \exEN{To meditate it is recommended to sit cross-legged.}
  \end{itemize}
   
\end{enumerate}         
        
\subsection{\son{u}}\label{subsec:ufr}

\sonancre{sonu} que l'on trouve par exemple dans les mots
\exFR{nous} (~\exPH{nu}~), \exFR{sous} (~\exPH{su}~), \exFR{tous} (~\exPH{tus}~), et bien d'autres.

Observons un exemple pour chacun d'entre eux et voyons leurs
traductions anglaises :\par

\begin{enumerate}
\item \exFR{nous} (~\exPH{nu}~)
  \begin{itemize}
  \item \exFR{Nous avons plutôt bien avancé, je suis fier de vous.}
    \item \exEN{We have made quite good progress, I am proud of you.}
    \end{itemize}
    
\item \exFR{sous} (~\exPH{su}~)
  \begin{itemize}
  \item \exFR{Sous le soleil exactement.}
  \item \exEN{Exactly under the sun.}
  \end{itemize}

\item \exFR{tous} (~\exPH{tus}~)
  \begin{itemize}
  \item \exFR{Un pour tous et tous pour un.}
  \item \exEN{One for all and all for one.}
  \end{itemize}
   
\end{enumerate}         
        
\subsection{\son{y}}\label{subsec:yfr}

\sonancre{sony} que l'on trouve par exemple dans les mots
\exFR{fur} (~\exPH{fyʀ}~), \exFR{pur} (~\exPH{pyʀ}~), \exFR{saturer} (~\exPH{satyʀe}~), et bien d'autres.

Observons un exemple pour chacun d'entre eux et voyons leurs
traductions anglaises :\par

\begin{enumerate}
\item \exFR{fur} (~\exPH{fyʀ}~)
  \begin{itemize}
  \item \exFR{Au fur et à mesure de la pratique l'expérience
      s'acquiert patiemment.}
    \item \exEN{As the practice progresses, experience is patiently acquired.}
    \end{itemize}
\item \exFR{pur} (~\exPH{pyʀ}~)
  \begin{itemize}
  \item \exFR{Paradoxalement, un son pur\footnote{Je ne peux que vous
        recommander le livre \href{https://www.amazon.fr/gp/product/2841691527/ref=as_li_tl?ie=UTF8&camp=1642&creative=6746&creativeASIN=2841691527&linkCode=as2&tag=wwwbecomefree-21&linkId=bae3b8d9f1cf846547085459e99db652}{Histoire de l'acoustique
          musicale} qui explique très bien cela, bien mieux que je ne
        saurais le faire ici.} est très désagréable à l'oreille.}
  \item \exEN{Paradoxically, a pure sound is very unpleasant to the ear.}
  \end{itemize}

\item \exFR{saturer} (~\exPH{satyʀe}~)
  \begin{itemize}
  \item \exFR{C'est Jimi Hendrix qui a lancé la mode qui consiste à
      saturer le son de sa guitare électrique.}
  \item \exEN{Jimi Hendrix launched the fashion of saturating the sound of his electric guitar.}
  \end{itemize}
   
\end{enumerate}         


\section{Voyelles nasales}\label{sec:nasales}

Elles sont dites \underline{nasales} lorsque le souffle s'échappe par
le nez et par la bouche à la fois :

\begin{center}
  \begin{tabular}[b]{*{4}{c}}
    \sonref{sonɑ̃ }{ɑ̃ }& \sonref{sonɛ̃ɛ̃ }{ɛ̃ }&   \sonref{sonɔ̃ }{ɔ̃ }& \sonref{sonœ̃ }{œ̃ }
  \end{tabular}
\end{center}



\subsection{\son{ɑ̃}}\label{subsec:atfr}
\sonancre{sonɑ̃ }que l'on trouve par exemple dans les mots
\exFR{arranger} (~\exPH{aʀɑ̃ ʒe}~), \exFR{frange} (~\exPH{fʀɑ̃ ʒ}~), \exFR{manger} (~\exPH{mɑ̃ ʒe}~), et bien d'autres.

Observons un exemple pour chacun d'entre eux et voyons leurs
traductions anglaises :\par

\begin{enumerate}
\item \exFR{arranger} (~\exPH{aʀɑ̃ ʒe}~)
  \begin{itemize}
  \item \exFR{J'espère que les choses vont s'arranger pour vous.}
    \item \exEN{I hope things will work out for you.}
    \end{itemize}
\item \exFR{frange} (~\exPH{fʀɑ̃ ʒ}~)
  \begin{itemize}
  \item \exFR{Il y a des femmes pour lesquelles la frange est un bon choix.}
  \item \exEN{There are women for whom bangs are a good choice.}
  \end{itemize}
\item \exFR{manger} (~\exPH{mɑ̃ ʒe}~)
  \begin{itemize}
  \item \exFR{Il faut manger pour vivre et non vivre pour manger.}
  \item \exEN{You have to eat to live and not live to eat.}
  \end{itemize}
\end{enumerate}         

\subsection{\son{ɛ̃}}\label{subsec:etfr}
\sonancre{sonɛ̃ } que l'on trouve par exemple dans les mots
\exFR{larcin} (~\exPH{laʀsɛ̃ }~), \exFR{matin} (~\exPH{matɛ̃ }~), \exFR{pantin} (~\exPH{pɑ̃ tɛ̃ }~), et bien d'autres.

Observons un exemple pour chacun d'entre eux et voyons leurs
traductions anglaises :\par

\begin{enumerate}
\item \exFR{larcin} (~\exPH{laʀsɛ̃ }~)
  \begin{itemize}
  \item \exFR{Arsène Lupin était surnommé le roi du larcin.}
    \item \exEN{Arsene Lupine was nicknamed the king of larceny.}
    \end{itemize}
    Arsène Lupin est un personnage fictif qui a connu un certain
    succès au siècle dernier\footnote{En témoigne la
      \href{https://www.amazon.fr/gp/product/B01N7I8WHS/ref=as_li_tl?ie=UTF8&camp=1642&creative=6746&creativeASIN=B01N7I8WHS&linkCode=as2&tag=wwwbecomefree-21&linkId=b8a42c0dbe0bb02385db28679d7e46fd}{bibliographie}
      et le nombre d'exemplaires vendus.}.
\item \exFR{matin} (~\exPH{matɛ̃ }~)
  \begin{itemize}
  \item \exFR{Chaque matin de nouvelles opportuinités s'offrent à vous.}
  \item \exEN{Every morning new opportunities are available to you.}
  \end{itemize}
\item \exFR{pantin} (~\exPH{pɑ̃ tɛ̃ }~)
  \begin{itemize}
  \item \exFR{Pinocchio est probablement le pantin le plus célèbre.}
  \item \exEN{Pinocchio is probably the most famous puppet.}
  \end{itemize}
\end{enumerate}         

\subsection{\son{ɔ̃}}\label{subsec:ctfr}
\sonancre{sonɔ̃ } que l'on trouve par exemple dans les mots
\exFR{chaton} (~\exPH{ʃatɔ̃ }~), \exFR{maison} (~\exPH{mɛzɔ̃ }~), \exFR{saison} (~\exPH{sɛzɔ̃ }~), et bien d'autres.

Observons un exemple pour chacun d'entre eux et voyons leurs
traductions anglaises :\par

\begin{enumerate}
\item \exFR{chaton} (~\exPH{ʃatɔ̃ }~)
  \begin{itemize}
  \item \exFR{Le chaton aime ronronner sur les cuisses de sa maîtresse.}
    \item \exEN{The kitten likes to purr on the thighs of his mistress.}
    \end{itemize}
\item \exFR{maison} (~\exPH{mɛzɔ̃ }~)
  \begin{itemize}
  \item \exFR{Dans chaque maison il y a une porte d'entrée.}
  \item \exEN{In each house there is an entrance door.}
  \end{itemize}
\item \exFR{saison} (~\exPH{sɛzɔ̃ }~)
  \begin{itemize}
  \item \exFR{Dans les zones tempérées l'année se découpe en saisons.}
  \item \exEN{In temperate zones the year is divided into seasons.}
  \end{itemize}
\end{enumerate}         

\subsection{\son{œ̃}}\label{subsec:oetfr}
\sonancre{sonœ̃ } que l'on trouve par exemple dans les mots
\exFR{brun} (~\exPH{bʀœ̃ }~), \exFR{jeûn} (~\exPH{ʒœ̃ }~), \exFR{lundi} (~\exPH{lœ̃ di}~), et bien d'autres.

Observons un exemple pour chacun d'entre eux et voyons leurs
traductions anglaises :\par

\begin{enumerate}
\item \exFR{brun} (~\exPH{bʀœ̃ }~)
  \begin{itemize}
  \item \exFR{En Europe du sud il y a plus de bruns que de blonds.}
    \item \exEN{In southern Europe there are more browns than blondes.}
    \end{itemize}
\item \exFR{jeûn} (~\exPH{ʒœ̃ }~)
  \begin{itemize}
  \item \exFR{Au réveil on est à jeûn parce qu'on ne mange pas en dormant.}
  \item \exEN{When we wake up we are fasting because we do not eat while sleeping.}
  \end{itemize}
\item \exFR{lundi} (~\exPH{lœ̃ di}~)
  \begin{itemize}
  \item \exFR{Comment ça va ? Comme un lundi.}
  \item \exEN{How is it going ? Like a monday.}
  \end{itemize}
\end{enumerate}         

\newpage
\minitoc
