\chapter{Les outils concrets pour améliorer votre phonétique}

Dans cette partie je vais vous donner une liste (forcément non
exhaustive) d'outils concrets pour améliorer votre pratique. Alors
c'est parti !

\begin{enumerate}
\item \href{http://upodn.com/}{upodn} ce site permet de convertir du
  texte en transcription phonétique.
\item \href{https://tophonetics.com/}{tophonetics} il fait la même
  chose que le précédent sauf qu'il a (au moins) trois avantages. Tout
  d'abord il permet de changer la langue de l'interface. Ensuite il
  permet de traduire en anglais britannique ou en anglais américain ce
  qui est très pratique. Enfin il existe une application mobile
  disponible sur l'App Store et Google Play.
\item \href{https://easypronunciation.com/fr/}{easypronunciation} ce
  site présente encore plus d'avantages que le précédent parce qu'il
  permet aussi de transcrire dans d'autres langues que l'anglais.
\item \href{http://www.photransedit.com/}{photransedit} le site est
  totalement en anglais. Par contre il a (au moins) trois atouts
  considérables. Le premier est qu'il dispose d'une version de bureau
  que vous pouvez installer sur votre ordinateur. Le deuxième est que
  sa version \exEN{online} se déciline en trois parties : texte,
  clavier et bibliothèque de textes déjà transcrits\footnote{Et rien
    que ça, ça vaut de l'or !}. Enfin son troisième atout est qu'il
  dispose de nombreuses ressources vers des blogs et sites web
  spécialisés dans la phonétique.
\item \href{http://www.phonemicchart.com/}{phonemicchart} propose
  moins d'options. En revanche il propose un tableau bien pratique
  lorsque qu'on a besoin d'écrire les symboles phonétiques qui ne sont
  pas forcément aisés à taper au clavier. De plus il dispose d'un
  moteur de transcription (uniquement pour des mots) et de nombreuses
  ressources complémentaires (notamment des \exEN{flashcards}).
\end{enumerate}

\newpage

