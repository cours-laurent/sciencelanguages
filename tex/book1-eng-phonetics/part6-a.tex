\chapter{Transcriptions phonétiques en 4 parties}

Voici les deux seules et uniques pages données dans la méthode nommée \ad. Elles
regroupent les sons en 4 catégories :
\begin{enumerate}
\item Voyelles brèves
\item Voyelles allongées
\item Voyelles doubles (diphtongues)
\item Consonnes  
\end{enumerate}

Tous les éléments présentés dans les tableaux suivants sont cliquables
et permettent ainsi de retrouver la section consacré au son pour plus
de détails.

\notation

Lorsque certains choix ont été fait, comme par exemple le
problématique \phon{e}, j'ai conservé celles qui sont dans le livre
pour montrer le problème concret évoqué plus haut dans mon livre.

\newpage

\section{Voyelles brèves}

\begin{center}
  \begin{table}[h]
    \centering
    \begin{tabular}[t]{ccc}
      API                         & Mot exemple  & Transcription \\ \\
      \hyperlink{soni}{\phon{ɪ}}  & \oxford{sit} & \wordref{sit}{sɪt}\\ \\
      \hyperlink{sonae}{\phon{æ}} & \oxford{cat} & \wordref{cat}{kæt}\\ \\
      \hyperlink{oa}{\phon{ɒ}}    & \oxford{shop}& \wordref{shop}{ʃɒp}\\ \\
      \hyperlink{omega}{\phon{ʊ}} & \oxford{put} & \wordref{put}{pʊt}\\ \\
      \hyperlink{sone}{\phon{e}}  & \oxford{ten} & \wordref{ten}{ten}\\ \\
      \hyperlink{sonup}{\phon{ʌ}} & \oxford{cup} & \wordref{cup}{kʌp}\\ \\
      \hyperlink{sonenv}{\phon{ə}}& \oxford{ago} & \wordref{ago}{əgəʊ}\\ \\
    \end{tabular}
    \caption{Voyelles brèves}
    \label{fig:voybrev}
  \end{table}
\end{center}

\newpage

\section{Voyelles allongées}

\begin{center}
  \begin{table}[h]
    \centering
    \begin{tabular}[t]{ccc}
      API                                & Mot exemple   & Transcription \\ \\
      \hyperlink{ilong}{\phon{iː}}       & \oxford{tea}  & \wordref{tea}{tiː}\\ \\
      \hyperlink{sonalong}{\phon{aː}}    & \oxford{car}  & \wordref{car}{kaː}\\ \\
      \hyperlink{oouvert}{\phon{ɔː}}     & \oxford{ball} & \wordref{ball}{bɔːl}\\ \\
      \hyperlink{ulong}{\phon{uː}}       & \oxford{boot} & \wordref{boot}{buːt}\\ \\
      \hyperlink{sonenvlong}{\phon{ɜː}}  & \oxford{bird} & \wordref{bird}{bɜːd}\\ \\
    \end{tabular}
    \caption{Voyelles allongées}
    \label{fig:voylong}
  \end{table}
\end{center}

\newpage

\section{Voyelles doubles (diphtongues)}

\begin{center}
  \begin{table}[h]
    \centering
    \begin{tabular}[t]{ccc}
      API                       & Mot exemple    & Transcription \\\\
      \hyperlink{ai}{\phon{aɪ}} & \oxford{buy}   & \wordref{buy}{baɪ}\\\\
      \hyperlink{ei}{\phon{eɪ}} & \oxford{day}   & \wordref{day}{deɪ}\\\\
      \hyperlink{oi}{\phon{ɔɪ}} & \oxford{boy}   & \wordref{boy}{ɔɪ}\\\\
      \hyperlink{ao}{\phon{aʊ}} & \oxford{brown} & \wordref{brown}{braʊn}\\\\
      \hyperlink{eo}{\phon{əʊ}} & \oxford{no}    & \wordref{no}{nəʊ}\\\\
      \hyperlink{ie}{\phon{ɪə}} & \oxford{beer}  & \wordref{beer}{bɪə}\\\\
      \hyperlink{oe}{\phon{ʊə}} & \oxford{tour}  & \wordref{beer}{tʊə}\\\\
      \hyperlink{ee}{\phon{eə}} & \oxford{air}   & \wordref{air}{eə}\\\\
    \end{tabular}
    \caption{Diphtongues}
    \label{fig:diphtong}
  \end{table}
\end{center}

\newpage

\section{Consonnes}

\begin{center}
  \begin{table}[h]
    \centering
    \begin{tabular}[t]{ccc}
      API                       & Mot exemple     & Transcription \\\\
      \hyperlink{th}{\phon{ð}}  & \oxford{this}   & \wordref{this}{ðɪs}\\\\
      \hyperlink{ss}{\phon{θ}}  & \oxford{thin}   & \wordref{thin}{θin}\\\\
      \hyperlink{ing}{\phon{ŋ}} & \oxford{sing}   & \wordref{sing}{siŋ}\\\\
      \hyperlink{ez}{\phon{ʒ}}  &\oxford{pleasure}&\wordref{pleasure}{pleʒə}\\\\
      \hyperlink{dj}{\phon{dʒ}} & \oxford{jam}    & \wordref{jam}{dʒam}\\\\
      \hyperlink{ch}{\phon{ʃ}}  & \oxford{shoe}   & \wordref{shoe}{ʃuː}\\\\
      \hyperlink{tch}{\phon{tʃ}}& \oxford{chips}  & \wordref{chips}{tʃips}\\\\
      \hyperlink{h}{\phon{h}}   & \oxford{hat}    & \wordref{hat}{hat}\\\\
    \end{tabular}
    \caption{Consonnes}
    \label{fig:cons}
  \end{table}
\end{center}

